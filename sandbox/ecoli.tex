% Author        : PokMan Ho pok.ho19@imperial.ac.uk
% Script        : ecoli.tex
% Desc          : reasons for including E. coli as electricigens
% Input         : none
% Output        : pdf report in same directory
% Arguments     : 0
% Date          : Feb 2020

\documentclass[a4paper,11pt]{article}
\usepackage[margin=2cm]{geometry}
\usepackage[english]{babel}
\usepackage{graphicx, longtable, amsmath, amssymb, csquotes}
\graphicspath{{graph/}{graph/}}

\usepackage{xcolor,colortbl}
\definecolor{green}{rgb}{0,.4,0}

\usepackage{hyperref}
\hypersetup{
	colorlinks=true,
	linkcolor=green,
	filecolor=red,      
	urlcolor=blue,
	citecolor=orange
}

%% citation
\usepackage[%
autocite 	= superscript,
backend 	= bibtex,
sortcites 	= true,
style 		= nature
]{biblatex}
\bibliography{ecoli.bib}

\title{\textit{E. coli} in Eco-Bioelectric Cell}
\author{PokMan Ho}
\date{08 Feb 2020}

\begin{document}
    \maketitle
    \tableofcontents

    \section{lineage description}
    \href{https://microbewiki.kenyon.edu/index.php/Escherichia_coli}{\textit{Escherichia coli}} is a facultative Gram-negative rod-shaped bacteria.  It has adhesive pilli but not able to sporulate.  It has one circular chromosome and one plasmid.  Being mesophiles, this lineage can live in different aquatic situations but achieve its maximum ability at 37$^o$C in lower gut environments.  Hence this is a common species on egestive materials and an indicator of water pollution by human faeces.
    
    \section{assumptions}
    \begin{itemize}
        \item cytochrome protein abundance on cell membrane have positive linear correlation with power output
        \item population density has positive correlation with power output
        \item no internal loss of electricity in cell culture once produced
    \end{itemize}
    
    \section{electricigen potential}
    Membrane-bounded cytochrome (cyt) proteins were long described\autocite{gennis1987cytochromes}, with associated midpoint potential of milli-volts (mV).  These cyt are from terminal oxidase families, and here we'll focus on only the aerobic group.  cytD class contains the highest midpoint potential (+180mV\autocite{gennis1987cytochromes} or +255mV\autocite{lorence1984effects}), meaning that the protein is net lack of electrons and hence the cells can gain electrons from anodes.  Midpoint potentials are highest in slightly acidic environment (pH6.0) and about halved in seawater pH\autocite{lorence1984effects}.  This cyt class are induced at low oxygen tension\autocite{gennis1987cytochromes}, means they present on the \ec\ cytoplasmic membranes in ordinary aquatic environment.
    
    \section{digestion efficiency to metabolic energy}
    
    \section{conclusion}
    Combining the above information, a high organic matter content aquatic pool would be best for \ec\ to live and produce electricity.  However this may also enhance electricity internal loss, which a microbial nanowire organism should be involved as counter-measure.
    
    \nocite{*}\printbibliography
\end{document}