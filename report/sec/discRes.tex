% 20200713, 20200720
\documentclass[../thesis.tex]{subfiles} %% use packages & commands as this main file
\begin{document}
\section{Discussion}
The model shows a general reality of possibilities in using a \pbs\ of \bac\ and \phy\ to continuously harvest carbon from the atmosphere with a biological meaningful quantity.  We define eight biological parameters in this general model and use published experimental measurements to root the basis of the parameter ranges (Table \ref{t:ranges}).  Then we use percentage ranges from data collected for some parameters to apply on data deficient parameters of similar biological meaning.  The model captures a few commonly-known important interactions between microbes and its carbon source (Fig.\ref{f:model}).  With minimal parameters and minimal mathematical complexity, this ordinary differential equation (ODE) model should reflect a biological pattern with the use of real-life parameter values.

With real-life scenarios, we explored the effect of having \bac\ coexisting with \phy\ and continuous/destructive harvest of carbon from the system.  The result shows daily carbon density yield achieves high by using the right combination of \phy\ and \bac.  On a contrary, \phy-only systems are more effective than \pbs\ for its linear accumulation of harvestable carbon (Fig.\ref{f:destCarbon}A).  Although \phy-only systems are easier to be established and generally having higher yield than the \pbs\ (Fig.\ref{f:ydDaily}), our model shows that among the few right combinations, much less external harvesting effort are needed to get a general high return (low harvest interval/rate comparisons in Fig.\ref{f:ydByHarv}).

Industrial production of biomolecules is a general interest in production maximisation.  Past studies on bacterial cellulose production showed material production by a pure bacterial culture can be maximised.  In the past few years a study established the experimental maximum to around 284 \dxdt\ \autocite{aytekin2016statistical}.  In the study they feed glucose to the \bac, replace the bacterial culture and seed a new batch for material extraction.  If we adopt a \phy\ recommendation from this model, theoretically they can have the same yield with only daily removal of organic matter (i.e. dead cells) from the system.  In that case, the material cost for them can be cut by quite a fraction with similar material output.  By knowing the biological properties of the \bac\ of interest, this model can also provide information on biological properties of the optimal \phy\ choice and optimal extraction frequency within seconds of calculation.

For this study, we want to investigate whether \bac\ can coexist with \phy\ without significantly changing the system’s maximum carbon harvest ability.  Our result shows that if we are able to get the optimal combination of \phy\ and \bac, the \PBH\ system is competitive with \phy-only systems even with low external effort of harvesting (Fig.\ref{f:ydByHarv}).  However in terms of absolute maximum carbon harvest, addition of \bac\ is always an disadvantage to the system.  The reason is because carbon fixed into organic form by \phy\ can be released back to the atmosphere through bacterial respiration (Fig.\ref{f:model}).  Hence in short, an optimised \PBH\ system is better suited than others because users can engineer the \bac\ to serve other functions.

However, the model also shows that \bac l invasion into \phy-only systems destroy productivity.  In the model, the \phy-only system resembles the \PoN\ system in our model.  People put up the system for some time and replace the saturated system with a new one.  \Bac l invasion resembles a transition from a \PoN\ to \PBN\ system.  In Fig.\ref{f:ydByHarv}\&\ref{f:ydByPara}, expectancy of productivity (i.e. median) drops from a positive number to zero.  Even with the resultant invaded community is the best-fit combination, productivity still halved, from around 350 to 250 \dxdt (Fig.\ref{f:ydByPara}).  On a different perspective, if the \phy\ culture is designed to incorporate external harvest, which resembles the \PoH\ system, then \bac l invasion is almost always be ecologically unfeasible.  Unfeasible \PBH\ means \bac\ will be naturally eliminated and hence the system is ``resilient" towards invasions.  This situation shows the importance of considering external harvest in the initial design of ecological carbon capture devices.

Also, we want to explore for a \pbs, whether destructive or continuous harvest can maximise the carbon harvest.  If you want absolute maximisation, our result support continuous harvest.  Yet if you want a balance between feasibility and maximum yield, then we suggest destructive harvest.  The reason is because a \PBH\ system can produce a high yield for specific \bac l biological combinations (Fig.\ref{f:ydByPara}E-H) but with a small feasibility limit (0.3\% success).  \PBN\ systems however have a smaller maximum (around 40 \dxdt\ lower) with chance of success around fifty-fifty.  Logically speaking, continuous harvest is a daily stable output from a mature carbon cycle within the \pbs.  On a contrary, biomass needs time to build up in a destructive system before the system gets replaced \autocite{aytekin2016statistical}.  Also in continuous harvest systems, the rate of cellular biochemistry is in its full theoretical potential.  Therefore to reach high productivity in destructive systems, grow rates (Fig.\ref{f:ydByPara}C,G) and density-independent death rate (Fig.\ref{f:ydByPara}H) have to be higher than that in continuous systems.

On the last point we are curious about whether biological features of \phy\ and/or \bac\ can significantly affect the system’s carbon harvest function.  Our result shows that biological features of \phy\ affect the carbon harvest yield flux.  $\ePR$, $\eP$ and $\gP$ have positive influence but $\aP$ has high negative impact.  \Phy\ intraspecific interference is strictly limited to the minimal value within its defined parameter range.  The effects are inconclusive for \bac l features (Fig.\ref{f:ydByPara}E-H) due to the high number of unfeasible scenarios.  That means a fast-growing carbon-efficient \phy\ insensitive to population density is highly preferred in any systems regardless of the types we had defined but the \bac\ choice is highly case-specific.

Among all the nine parameters, death-related parameters ($\aP$ \& $\mB$) are hardly changeable under the current technology -- they can only be changed by choosing appropriate microbial candidates.  On a contrary, harvest rate ($x$) is the easiest to change according to needs because it is an external effort working on the system.  \Phy\ growth rate ($\gP$) and bacterial clearance rate ($\gB$) are changeable with a number of associated factors \autocite{park2020potential}.  These factors may relate to many environmental and genetic factors such as temperature, media acidity, light intensities, metabolites concentration, carbon use efficiencies (i.e. $\ePR$ \& $\eP$ for \phy; $\eBR$ \& $\eB$ for \bac) and resource allocation to other cellular activities instead of biomass (such as environmental sensing).  For the group of factor parameters, they can be changed by genetic modification \autocite{moniruzzaman1996ethanol} but hard to be changed via growth media because of phenotypic plasticity \autocite{j1989respiration,bratbak1985phytoplankton,samejima1958heterotrophic} and interspecies coordinations \autocite{beliaev2014inference,amin2012interactions}.

\end{document}