% Author        : PokMan Ho pok.ho19@imperial.ac.uk
% Script        : candPlay.tex
% Desc          : find candidates in model
% Input         : none
% Output        : pdf report in same directory
% Arguments     : 0
% Date          : Feb 2020

\documentclass[a4paper,11pt]{article}
\usepackage[margin=2cm]{geometry}
\usepackage[english]{babel}
\usepackage{graphicx, longtable, amsmath, amssymb, csquotes}
\graphicspath{{graph/}{graph/}}

\usepackage{xcolor,colortbl}
\definecolor{green}{rgb}{0,.4,0}

\usepackage{hyperref}
\hypersetup{
	colorlinks=true,
	linkcolor=green,
	filecolor=red,      
	urlcolor=blue,
	citecolor=orange
}

%% citation
\usepackage[%
%autocite 	= superscript,
backend 	= bibtex,
sortcites 	= true,
style 		= nature
]{biblatex}
\bibliography{candPlay.bib}

\title{Searching for suitable candidate players in the Autotroph-Heterotroph coupled Eco-Bioelectric Cell}
\author{PokMan Ho}
\date{06-07 Feb 2020}

\begin{document}
    \maketitle
    \tableofcontents
    
    \section{Main message}
    Look like I have to use my very limited theoretical modeling knowledge to make an autotroph-heterotroph interaction model
    
    \section{candidates}
    A \href{https://www.nature.com/articles/nrmicro2113}{list} of \href{https://en.wikipedia.org/wiki/Exoelectrogen}{species} and extra (start from \#22)
    
    \begin{enumerate}
            \item Shewanella oneidensis MR-1
            \item Shewanella putrefaciens IR-1
            \item Clostridium butyricum
            \item Desulfuromonas acetoxidans
            \item Geobacter metallireducens
            \item Geobacter sulfurreducens
            \item Rhodoferax ferrireducens
            \item Aeromonas hydrophilia (A3)
            \item Pseudomonas aeruginosa
            \item Desulfobulbus propionicus
            \item Geopsychrobacter electrodiphilus
            \item Geothrix fermentans
            \item Shewanella oneidensis DSP10
            \item \href{https://microbewiki.kenyon.edu/index.php/Escherichia_coli}{\textbf{Escherichia coli}}: \href{https://www.greenoptimistic.com/bio-battery-e-coli-20130718/}{existing battery}
            \item Rhodopseudomonas palustris
            \item Ochrobactrum anthropic YZ-1
            \item Desulfovibrio desulfuricans
            \item Acidiphilium sp.3.2Sup5
            \item Klebsiella pneumoniae L17
            \item Thermincola sp.strain JR
            \item Pichia anomala
            \item \href{https://pubs.acs.org/doi/pdf/10.1021/es2020007}{\textbf{Bacillus stratosphericus}}
        \end{enumerate}
    
    \subsection{Electricigens: microbial powerhouse}
    \href{https://microbewiki.kenyon.edu/index.php/Escherichia_coli}{\textit{Escherichia coli}}
    \begin{itemize}
        \item easy accessible
        \item common, lots of researches on living condition
        \item existing working \href{https://www.greenoptimistic.com/bio-battery-e-coli-20130718/}{battery}
        \item it can help food breakdown and digestion - potential simulation of gut environment will work already?
    \end{itemize}
    \href{https://microbewiki.kenyon.edu/index.php/Bacillus_stratosphericus#Applications_to_Biotechnology}{\textit{Bacillus stratosphericus}}
    \begin{itemize}
        \item can be found in extreme environments
        \item can release energy to cathode in anaerobic environment
        \item not too preferred in our situation as it should be a oxic environment
    \end{itemize}
    
    \subsection{Exoelectrogen: microbial wires}
    \href{https://microbewiki.kenyon.edu/index.php/Geobacter}{\textit{Geobacter sulfurreducens}}\autocite{reguera2005extracellular}\\
    Reasons:
    \begin{itemize}
        \item this genus is highly electro-conductive
        \item this species do not have highly-mobile pilli
        \item this species can attach to electron source head-to-tail and let electron run through
        \item (disadvantage) need Fe(II) ions in culture as nutrient source
    \end{itemize}
    
    \subsection{Autotroph: food source}
    \href{https://www.researchgate.net/publication/259495650_Spirulina_Cultivation_A_Review}{Spirulina}\autocite{usharani2012spirulina}
    \begin{itemize}
        \item as currently a type of superfood (E.coli should have no problem in ingesting)
        \item easily cultivable (no fancy, hence expensive and leave heavy C-footprint, materials required)
        \item \href{https://microbewiki.kenyon.edu/index.php/Arthrospira_platensis}{common} organism
    \end{itemize}
    
    \section{papers on electricigen cyanobacteria}
    \href{https://www.nature.com/articles/s41467-017-01084-4/}{printed on paper}\\
    \href{sci-hub.se/10.1021/acs.nanolett.8b02642}{printed on mushrooms}
    
    \section{Useful reference quotes}
    model-related articles: 1985\autocite{bratbak1985phytoplankton}, 1992\autocite{sanders1992relationships}\\
    parameter-related articles: 2014\autocite{beliaev2014inference}, 1990\autocite{currie1990large}\\
    application-related articles: 2018\autocite{light2018flavin}, 2005\autocite{reguera2005extracellular}, 2013\autocite{xie2013microbial}

    \subsection{Interactions between Diatoms and Bacteria}
    conclusion\autocite{amin2012interactions}: not too useful unless choose seawater and diatoms as media
    Diatoms
    \begin{itemize}
        \item phycosphere: region of enhanced bacterial growth due to extracellular products of alga cell/colony
        \item total flux of molecules proportional to cellular surface area (i.e. cell size-related)
        \item diatoms release ``transparent exopolymer particles" (TEP) - either actively or upon cell lysis, often colonized by bacteria
        \item signaling likely serving:
        \begin{itemize}
            \item nurture specific bacteria
            \item facilitate bacterial attachment
        \end{itemize}
        \item provide dissolved organic carbon (DOC)
    \end{itemize}
    Bacteria
    \begin{itemize}
        \item average velocity 30$\mu ms^{-1}$ (very slow, \textit{E. coli})
        \item run-and-tumble mechanism, \textit{E. coli}
        \item release exopolysaccharies (EPS) respond to presence of phytoplankton to initiate attachment
        \item attach to diatoms for vitamins
    \end{itemize}
    
    \subsection{Inference of interactions in cyanobacterial– heterotrophic co-cultures via transcriptome sequencing}
    conclusion\autocite{beliaev2014inference}: can use as parameter reference source
    \begin{itemize}
        \item photoautotroph-heterotroph interactions
        \item cyanobacteria: Synechococcus sp. PCC 7002
        \item marine facultative aerobe Shewanella putrefaciens W3-18-1
        \item given artificial culture medium recipe
        \item baseline cultivation conditions and physiological outputs of cyanobacteria \& co-cultured bacteria: C-source, dilution rate, irradiance ($\mu$mol photons), DOT, OD730, biomass
        \item metabolic coupling leading to co-culture growth on either inorganic / organic carbon
        \item excreted metabolite table
        \item functional assignment pie charts (only on enhanced expression genes $>$200\%, mRNA abundance as indicator)
        \item lists of biochemical pathway enhancement with p-value
    \end{itemize}
    
    \subsection{Phytoplankton-bacteria interactions: an apparant paradox? Analysis of a model system with both competition and commensalism}
    conclusion\autocite{bratbak1985phytoplankton}: can use a preliminary model reference
    \begin{itemize}
        \item compare simple mathematical model to biological model system
        \item biological model syste: diatom \textit{Skeletonema costatum} \& marine bacterium strain with limited phosphate
        \item paradox of enrichment, predicted by model
        \item enrich chemostat content by remove water
    \end{itemize}
    
    \subsection{Interactions between bacteria and algae in aquatic ecosystems}
    conclusion\autocite{cole1982interactions}: not useful
    \begin{itemize}
        \item positive correlation between primary production and heterotrophic bacterial abundance
        \item heavily based on lab culture results
        \item support on phycosphere concept
    \end{itemize}
    
    \subsection{Large-scale variability and interactions among phytoplankton, bacterioplankton, and phosphorus}
    conclusion\autocite{currie1990large}: maybe preliminary parameter ideas
    \begin{itemize}
        \item freshwater ecosystem
        \item phosphate limited environment
        \item chemical and/or temperature limit ecosystem development
        \item linearized log equations, using R$^2$ as fit indicator
        \item statistical test on empirical model parameters
    \end{itemize}
    
    \subsection{Phytoplankton-bacteria coupling under elevated CO2 levels: A stable isotope labelling study}
    conclusion\autocite{de2010phytoplankton}: parameter value if need to consider CO$_2$ effect
    \begin{itemize}
        \item lab simulated algal bloom
        \item CO$_2$ has no significant effect on C transfer to bacteria during algal bloom
        \item CO$_2$ effect has most pronounced effect in post-bloom phase, under nutrient limitation
        \item simple C-13 source-sink isotope ratio model
        \item CO$_2$ effect on biomass table
        \item empirical model + statistical test validation
    \end{itemize}
    
    \subsection{Recognition Cascade and Metabolite Transfer in a Marine Bacteria-Phytoplankton Model System}
    conclusion\autocite{durham2017recognition}: nothing related
    \begin{itemize}
        \item metabolites relationship towards bacterial growth mediation
        \item potential insight on oceanic carbon transformations
        \item gene regulations and biochemical pathways
    \end{itemize}
    
    \subsection{Dynamics of bacterial abundance and the related environmental factors in large shallow eutrophic Lake Taihu}
    conclusion\autocite{gong2017dynamics}: unrelated
    \begin{itemize}
        \item optimizing bacterial cell counting method in field
        \item environmental parameters for eutrophied lakes
    \end{itemize}
    
    \subsection{The release of micro-algal photosynthate and associated bacterial uptake and heterotrophic growth}
    conclusion\autocite{jones1986release}: not related
    \begin{itemize}
        \item summary table of measured C-14 uptake by algae
        \item biomass energy content
        \item empirical measurement-based
    \end{itemize}
    
    \subsection{An Experimental Study of Microbial Fuel Cells for Electricity Generating: Performance Characterization and Capacity Improvement}
    conclusion\autocite{li2013experimental}: not useful
    \begin{itemize}
        \item benthic mud sample anaerobic electricity generation
        \item heteroorgannotrophic microbes
        \item max electricity generation (mV) at day 0, linearly dropping until day 10 (voltage =0)
        \item aquarium pump needed - C-footprint
    \end{itemize}
    
    \subsection{A flavin-based extracellular electron transfer mechanism in diverse Gram-positive bacteria}
    conclusion\autocite{light2018flavin}: application-citable
    \begin{itemize}
        \item extracellular electron transfer (EET) constructed a novel electron transport chain that supports growth on extracellular electron acceptors
        \item mutation-related study
    \end{itemize}
    
    \subsection{Extracellular electron transfer via microbial nanowires}
    conclusion\autocite{reguera2005extracellular}: potential ebc-player
    \begin{itemize}
        \item for bioengineering of novel conductive materials
        \item bacteria that intake and excrete electrons
    \end{itemize}
    
    \subsection{Relationships between bacteria and heterotrophic nanoplankton in marine and fresh waters: an inter-ecosystem comparison}
    conclusion\autocite{sanders1992relationships}: primary model structure, not related to autotrophs
    \begin{itemize}
        \item chemosensitivity of bacteria and heterotrophic nanoplankton
        \item density-dependent Michaelis-Menten (Monod) type relationships
        \item simple predator-prey model
    \end{itemize}
    
    \subsection{Cultivation of an Obligate Fe(II)-Oxidizing Lithoautotrophic Bacterium Using Electrodes}
    conclusion\autocite{summers2013cultivation}: not related
    \begin{itemize}
        \item metal-consuming bacteria generate current
        \item \textit{M. ferrooxydans} PV-1, are capable of accepting electrons from an external source
    \end{itemize}
    
    \subsection{Microbial battery for efficient energy recovery}
    conclusion\autocite{xie2013microbial}: not related to model but ok for discussion
    \begin{itemize}
        \item coupling electricigens as microbial fuel cells using reversible redox solid state electrode as bridges
        \item 0.01-1mAcm$^{-2}$
        \item energy loss reason and percentages
        \item voltage drop seriously within hours
    \end{itemize}
    
    \nocite{*}\printbibliography
\end{document}
