% Author: PokMan Ho pok.ho19@imperial.ac.uk
% Script: intro.tex
% Desc: MRes thesis introduction section
% Input: none
% Output: none
% Arguments: 0
% Date: Apr 2020

\documentclass[../thesis.tex]{subfiles} %% use packages & commands as this main file

\begin{document}
\section{Introduction}

Climate destabilisation is happening.\autocite{notz2016observed,schuur2015climate}  Anthropogenic emission is affecting the environment\autocite{notz2016observed} but the society still depends on fossil fuels to power the economy.\autocite{lotfalipour2010economic}  Since economy has to be developed by using electricity,\autocite{ferguson2000electricity} how can we minimise our impacts to the globe?

In the past decades, scientists and engineers have proposed numerous methods and measures to capture and sequester carbon.\autocite{farrelly2013carbon,yang2008progress}  There are three main approaches: abiotic (chemical or mechanical) capture and isolation, artificial fertilization and plantation.  From the aspect of life-cycle assessment (LCA), the only way to achieve a net negative carbon footprint is to minimize anthropogenic intervention.  Most of the methods reviewed cannot logically achieve expected effect because they fail to address carbon emitted from associated materials they use.  For example, chemicals, adsorbents and fertilizers used for carbon sequestration require electricity to produce.  The use of this material can only be justified being carbon negative if electricity is produced in a (at most) carbon-neutral way under LCA.  Hence it is nearly unfeasible to use artificial non-waste materials as ingredients for carbon capture.  One exceptional case is the cultivation of fast-growing photosynthetic organisms,\autocite{farrelly2013carbon} if they can survive without human cultivation effort.  This shines light on ecology, which building a sustainable artificial microbial ecosystem is logically feasible.

Aquatic photosynthetic communities are guesstimated as powerful as terrestrial ones in carbon capture ability.\autocite{SCHLESINGER2013341}  Since aquatic phytoplanktons are small sized but fast-growing organisms, their communities can be bottled into different forms, such as panels and bulbs\autocite{evanson_2019}.  If this advantage can be better utilised, carbon footprints from cities and industries can potentially be decreased.  Given the fact that bacterial decomposers are everywhere, it is illogical to neglect their existence (and potential impacts) in an open system (i.e. materials can be exchanged between the set-up and the outside world) in open space.  With the above idea in mind, empirical approaches become too complicated because 1. fit phytoplankton candidate identities are unknown; 2. lineages of bacterial decomposers exist in a system are random and their performances are under-investigated; and 3. too many potential biochemical interactions between individuals, populations and lineages are simultaneously happening with many interactions are yet to be discovered.  Hence a mathematical model is useful to generalise the reality and contain the system to a few defined interactions for detailed investigations.

Many interactive models are built for different purposes, mainly grouped by their growth media (aquatic or terrestrial) and number of players (multi-lineages or multi-organs within an individual).  These models are specific to their respective environmental settings and nutrient limitations.  By comparing model features, a new general model should be made (Table \ref{modComp}).  Apart from the above phenotypical models, there are two models published, a community model based on whole genome\autocite{harcombe2014metabolic} and a specific biofuel\autocite{kirthiga2014mathematical} one.  These two models are too specific because they require prior knowledge and data regarding to the microbes of interest.  Hence this new ordinary differential equation (ODE) model is aimed at visualising a minimal ecosystem as a whole, which accommodates known and unknown microbes as long as they fit into either the phytoplankton or bacterial decomposer categories.

Features of this model/system are fitted between the previous two clusters of models.  It is a three-parts system modeling carbon density at equilibrium position.  The open system is modeled under the research question ``how much carbon can a minimal artificial ecosystem sequester?"  Under this motivation, three hypotheses are proposed:
\begin{itemize}
    \item there are biological constraint ranges for a lineage being considered as a ``suitable candidate" for the system;
\end{itemize}

\begin{itemize}
    \item there are a few important factors bringing system stability; and
\end{itemize}

\begin{itemize}
    \item this system can bring biologically significant effect on carbon sequestration
\end{itemize} %% need to be more specific, can change in later stage of project (20200423)

\end{document}