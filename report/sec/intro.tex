% 20200814, 20200815
\documentclass[../thesis.tex]{subfiles} %% use packages & commands as this main file
\begin{document}

Climate change is affecting not only biosphere, but also human society \autocite{notz2016observed,schuur2015climate}.  Extreme weather, seasonal regime shifts and tropical disease spread towards temperate regions have intensified.  Although we know carbon emissions are major contributors of climate change \autocite{notz2016observed}, we still need fossil fuels to power our cities and industries \autocite{ferguson2000electricity}.  Minimising or even reverting the damages to our climate is an urgent task to complete.

For decades, experts have proposed geo-engineering plans for climate mitigation and adaptation \autocite{farrelly2013carbon,yang2008progress,boyd2008ranking,boettcher2019high,vaughan2011review}.  Two major categories were distinguished \autocite{boyd2008ranking}, one on carbon capture and storage and the other one on solar radiation reduction.  Debates and reviews are still ongoing because we do not have a standard scheme and/or model to judge which proposal(s) work(s) or not \autocite{boyd2008ranking,boettcher2019high,boyd2016development,oschlies2017indicators,gattuso2018ocean}.  More importantly, interfering natural processes may cause unintended consequences.

One intervention example is large-scale fertilisation of open ocean \autocite{gnanadesikan2008export,lovelock2007ocean,lawrence2014efficiency} or intertidal autotrophs \autocite{duarte2017can,johannessen2016geoengineering,krause2016substantial}.  Fertilisation can be done using artificial fertilisers \autocite{gnanadesikan2008export,lawrence2014efficiency,trick2010iron} or piping deep ocean water \autocite{kwiatkowski2015atmospheric,lovelock2007ocean}.  However the expensive fertilisation may have lower carbon yield than expected \autocite{boyd2008implications,gnanadesikan2008export,oschlies2010side} while risking regional acidification \autocite{oschlies2010side}, unbalanced ecological resource competition \autocite{chung2011using,thiele2012microbial,batten2014did} and atmospheric pressure anomaly \autocite{kwiatkowski2015atmospheric}.  Fertilising high-nitrate-low-chlorophyll water may also trigger toxic algal blooms which harms the regional ecosystem \autocite{trick2010iron}.

Another example is large scale reflective particles injection to high atmosphere as sun-shields \autocite{latham2008global,stjern2018response,xia2016stratospheric}.  The expensive sun-shields \autocite{boyd2008implications} can theoretically pause greenhouse effect by blocking sunlight \autocite{williamson2012impacts}.  Sunlight reduction, however, can collapse thermal-sensitive food webs \autocite{williamson2012impacts}, causing extreme weather events even more extreme \autocite{jones2009climate} and weakening the deep ocean thermal circulation for a long time \autocite{lauvset2017climate,williamson2012impacts}.  Salt particles and soluble aerosols can also fall back to Earth through precipitations and cause damages to ecosystems \autocite{bala2011albedo}.  The proposal thus put rainforests at risk \autocite{muri2015tropical}.

Geo-engineering plans are expensive because they require lots of power and chemicals \autocite{boyd2008ranking,boyd2008implications,mcclellan2012cost}.  Power generation and chemical extraction currently rely on the supply of fossil fuels, which is the main contributor for climate change.  Calculation of the overall benefits, carbon footprints and environmental damages of a proposal is a ``life-cycle assessment" (LCA).  Many geo-engineering proposals are unsustainable under LCA \autocite{abdussalam2020green}, such as fertilisation and particle injection plans.

A potential sustainable solution for carbon capture and storage is using \phy\ \autocite{farrelly2013carbon} or co-cultures of \phy\ and \bac\ \autocite{fuentes2016impact}, using nature to cure natural problems.  Several \phy-\bac\ co-cultures have been found suitable for biomass and biomolecules production in the last decade \autocite{fuentes2016impact,santos2014microalgal}.  Although carbon yield production from these co-cultures are high, publications did not investigate the underlying mechanistic reasons, such as life history traits.  The popular investigation direction is on biochemical interactions between \phy\ and \bac, including chemical/genetics \autocite{amin2009photolysis,durham2015cryptic,leyva2014accumulation,rivas2010interactions,seyedsayamdost2011roseobacticides} or phenomenological \autocite{choix2012enhanced1,choix2012enhanced2,kazamia2012mutualistic,santos2014microalgal} aspects.  However these information is unable to explain why these combinations are special.

Siderophores are popular chemical mediators produced by \bac\ enhancing \phy\ growth and carbon yield \autocite{fuentes2016impact}.  Yet what bio-mechanical pathways are regulated by siderophores causing an enhanced \phy\ growth are not mentioned.  Biological importance of such biochemical regulations on bio-mechanical pathways are rarely discussed, and there is very limited evidence showing siderophores are necessary to achieve high yield scenarios.  This study aims to address this research gap by exploring the mechanistic reasons through the importance of life history trait values of \phy\ and \bac.  In addition, the method of carbon harvest is investigated in this study.  It plays a crucial role in the success of deploying of a co-culture system and its highly dependent on human-related factors in the ecosystem.

In the past few decades, dozens of \phy-\bac\ related mechanistic models are published but unsuitable for this study.  They are based on site-specific nutrient (nitrogen and/or phosphorus) cycles (Table \ref{modComp}).  They used zooplankton, functional responses on grazing pressure and biomass batch removal by seasonal thermocline depth shifts \autocite{anderson2015empower,kidston2013phytoplankton,llebot2010role}
with multiple chemical resource pools \autocite{llebot2010role,mitra2009closure,findlay2006modelling}.  They are unsuitable because I focus on a single resource (carbon dioxide) pool in a co-culture.  A new mechanistic model is developed, simulating how carbon flows in a \pbs\ with one \phy, one \bacm\ and artificial organic carbon harvest.

Using the new model, I address \Rn{1}) feasibility of \pbs s, \Rn{2}) harvest preferences on different \phy\ and \phy-\bac\ \pbs s and \Rn{3}) importance of life history traits of \phy\ and \bac\ on the organic carbon yield.

\textbf{How common is a \pbs\ being feasible under different harvest modes?}  Feasibility is defined as finite non-negative carbon densities in each system component in a scenario.  Harvest mode is defined as either ``destructive” or ``continuous”.  Destructive harvest extracts all carbon from a system after letting it run for some time.  Continuous harvest extracts carbon at a steady rate when the system is running.  Knowing the feasibility limits can expand the knowledge on the fragmented understanding \autocite{fuentes2016impact} on \pbs s.  The information can show whether \pbs s is an effective method for carbon capture and storage.

\textbf{In what conditions does a \pbs\ prefer destructive or continuous harvest?}  Destructive harvest is a common practice in farms and is extended to aquacultures of macroalgae \autocite{duarte2017can} and \phy\  \autocite{evanson_2019}.  Continuous harvest is a growing concept for yield enhancement in various microbial cultures \autocite{aytekin2016statistical,fuentes2016impact}.  At present, there is insufficient information to conclude which harvesting method is superior than the other under what conditions.  Evaluating both harvest methods quantitatively can give direction on designing protocols to construct, deploy and maintain \pbs\ as a carbon harvest engine.

\textbf{To what extent do life history traits influence organic carbon yield from \phy\ cultures and \pbs?}  At present, experimental searching effort on possible co-cultures in the past decade has resulted in a handful of examples \autocite{fuentes2016impact,santos2014microalgal}.  Because \pbs s are rare, published methodologies in these \pbs\ investigations are case-specific.  Criteria for searching and validating a possible \phy\/\bacm\ candidate are unclear.  Co-culture carbon yields are typically optimised by growth media optimisation \autocite{aytekin2016statistical,fuentes2016impact} and interspecies signalling \autocite{fuentes2016impact}.  The optimisations can provide indirect inference on how high the carbon yield can be but have insufficient mechanistic information showing how these yields are achieved.  Directly addressing the lack of mechanistic information is necessary to show the key indicators contributing to the success in carbon harvest using \phy-\bac\ \pbs s.

The new model resembles an open system (material exchange across system boundary is allowed) of carbon capture engineered with one \phy\ and one \bacm.  Prediction on organic carbon yield at different times and stable positions of the systems are possible.  The predictions provide quantitative guides on which life history trait values of \phy\ and \bac\ should be considered and what type of harvesting method is optimal for the selected \phy-\bac\ couple.

\end{document}