% Author: PokMan Ho
% Script: disc.tex
% Desc: MRes thesis discussion section on the result
% Input: none
% Output: none
% Arguments: 0
% Date: Jul 2020

\documentclass[../thesis.tex]{subfiles} %% use packages & commands as this main file

\begin{document}
\section{Discussion}
\subsection{Overview}
The result shows that P+B systems can be more productive the P-only system in total carbon content (Fig.\ref{fig:totC}) but not in yield flux (Fig.\ref{fig:yield}).  Destructive harvest also significantly higher than both total system carbon (Fig.\ref{fig:totC}) and yield flux (Fig.\ref{fig:yield}).  In terms of biological components (Fig.\ref{fig:v2}), $\ePR$, $\eP$, $\gP$ and $\eBR$ seems not having directional effect on both total carbon and yield flux on feasible PBH systems.  $\aP$, $\eB$ and $\gB$ are having negative effect on both carbon and yield aspects of feasible PBH systems.  $\mB$ is a unique parameter, which higher bacterial death rates bring down the total system carbon but raise the yield flux.  At the high $\mB$ extreme the PBH systems have the 95\% confidence interval of total carbon largely overlap with that of PoH systems.  Low $\mB$ is having high total carbon in systems but very low yield value in PBH.

If we want to construct an efficient carbon harvest device with other bacterial function, the theoretical optimal strategy would be using a phytoplankton strain with low density sensitivity (low $\aP$) with a low growth efficiency (low $\eB$) K-selected (low $\gB$ and low $\mB$) bacteria.  This device is being constantly replaced/flushed by new culture medium once the total system carbon density reaches equilibrium.  In short, a theoretical PBN system is preferred as a carbon harvest device, which various genes can be edited into the bacterial component for designated industrial functions.  Experimentally, the PBN system resembles a semi-continuous growth media replacement system.\autocite{aytekin2016statistical}

\end{document}