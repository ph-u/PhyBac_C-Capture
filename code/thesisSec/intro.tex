% Author: PokMan Ho
% Script: intro.tex
% Desc: MRes thesis introduction section
% Input: none
% Output: none
% Arguments: 0
% Date: Apr 2020

\documentclass[../thesis.tex]{subfiles} %% use packages & commands as this main file

\begin{document}
\section{Introduction}

Climate destabilisation is happening.\autocite{notz2016observed,schuur2015climate}  Anthropogenic emission is affecting the environment\autocite{notz2016observed} but the society still depends on fossil fuels to power the economy.\autocite{lotfalipour2010economic}  Since economy has to be developed by using electricity,\autocite{ferguson2000electricity} how can we minimise our impacts to the globe?

In the past decades, scientists and engineers have proposed numerous methods and measures to capture and sequester carbon.\autocite{farrelly2013carbon,yang2008progress}  There are three main approaches: abiotic (chemical or mechanical) capture and isolation, artificial fertilization and plantation.  From the aspect of life-cycle assessment (LCA), the only way to achieve a net negative carbon footprint is to minimize anthropogenic intervention.  Most of the methods reviewed cannot logically achieve expected effect because they fail to address carbon emitted from associated materials they use.  For example, chemicals, adsorbents and fertilizers used for carbon sequestration require electricity to produce.  The use of this material can only be justified being carbon negative if electricity is produced in a (at most) carbon-neutral way under LCA.  Hence it is nearly unfeasible to use artificial non-waste materials as ingredients for carbon sequestration.  One exceptional case is the cultivation of fast-growing photosynthetic organisms,\autocite{farrelly2013carbon} if they can survive without human cultivation effort.  This shines light on ecology, which building a sustainable artificial microbial ecosystem is logically feasible.

Aquatic photosynthetic communities are guesstimated as powerful as terrestrial ones in carbon capture ability.\autocite{SCHLESINGER2013341}  Since aquatic phytoplanktons are small sized but fast-growing organisms, their communities can be bottled into different forms, such as panels and bulbs\autocite{evanson_2019}.  If this advantage can be better utilised, carbon footprints from cities and industries can potentially be reduced.  Given the fact that bacterial decomposers are everywhere, it is illogical to neglect their existence (and potential impacts) in an open system (i.e. materials can be exchanged between the set-up and the outside world) in open space.  With the above idea in mind, empirical approaches become too complicated because 1. many phytoplanktons are unknown, hence optimal candidates are unlikely to be identified and listed; 2. lineages of bacterial decomposers exist in a system are random and their performances are under-investigated; and 3. too many known and unknown biochemical interactions are happening simultaneously between individuals, populations and lineages in real life.  Hence a mathematical model is useful to generalise the reality and contain the system to a few defined interactions for detailed investigations.

Many interactive models are built for different purposes, mainly grouped by their growth media (aquatic or terrestrial) and number of players (multi-lineages or multi-organs within an individual).  These models are specific to their respective environmental settings and nutrient limitations.  By comparing model features (Table \ref{modComp}), a new general model should be constructed.  Apart from the above phenotypical models, there are two models published, a community model based on whole genomes\autocite{harcombe2014metabolic} and a specific model for biofuel\autocite{kirthiga2014mathematical}.  These two models are too specific because they require prior knowledge and data regarding to the microbes of interest.  Hence this new ordinary differential equation (ODE) model is aimed at visualising a minimal ecosystem as a whole, which accommodates known and unknown microbes as long as they fit into either the phytoplankton or bacterial decomposer categories.

Features of this model/system are fitted between the previous two known clusters of models (i.e. aquatic nutrient and terrestrial carbon-based).  It is a three-part system modeling carbon density from start to equilibrium.  The open system is modeled under the research question ``how fast can we remove atmospheric CO$_2$ using a simple microbial ecosystem?"  Under this motivation, three hypotheses are proposed:
\begin{itemize}
    \item \textbf{human intervention in the microbial system destabilizes the internal balance}
%    \item \textbf{presence of artificial removal of organic carbon shifts optimal biological feature values of phytoplanktons and bacterial decomposers}
    ;
\end{itemize}

The carbon removal term is an alien term in the ODE system (Eq.\ref{eq:ode}).  This specific term depend only on the instantaneous density of organic carbon.  It also has no interaction with any other carbon pools.  It will disappear under ``no intervention" scenario (removal rate is set zero).  Hence presence of the external perturbation can potentially shift the internal balance, shifting optimal biological features (e.g. growth rates, intraspecific interference, death rates and carbon handling efficiencies) to other values.

\begin{itemize}
    \item \textbf{human intervention benefits carbon sequestration performance of the system}
%    \item \textbf{carbon sequestration systems under respective optimum conditions perform better with higher rate of artificial organic carbon removal}
    ; and
\end{itemize}

It is based on the idea that carbon sequestration ability of grasslands (i.e. a frequently-disturbed ecosystem by wild fire) may be more reliable than forests (i.e. a stable ecosystem).\autocite{dass2018grasslands}  A possible reason is because young individuals maximize growth (i.e. carbon accumulation for biomass) to win over competitors.  Under this idea, the carbon cycle in the ecosystem can potentially be sped up by taking carbon out from the system and conserve this momentum of growth.

\begin{itemize}
    \item \textbf{with a given set of microbial features, human intervention brings up the total removed carbon at a linear manner}
%    \item \textbf{artificial removal rate of organic carbon exponentially-correlates with total collected carbon density at the end of system run-time under the same set of biological features}
    .
\end{itemize}

From Eq.\ref{eq:ode}, the alien term of carbon removal is linearly-linked with the instantaneous density of organic carbon.  Hence increase of the removal rate is expected to bring a linear effect on the total amount of carbon sequestered.

\end{document}
