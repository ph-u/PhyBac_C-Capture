% Author: PokMan Ho pok.ho19@imperial.ac.uk
% Script: intro.tex
% Desc: MRes thesis introduction section
% Input: none
% Output: none
% Arguments: 0
% Date: Apr 2020

\documentclass[../thesis.tex]{subfiles} %% use packages & commands as this main file

\begin{document}
\section{Introduction}

%20200410
Climate destabilization has alerted many and plankton farming is one of the ideas attempting to do carbon sequestration service.  However, using the trial-and-error approach to seek for the best candidate(s) can be time consuming and cost ineffective due to the huge microbial diversity.  By using a forward-time ordinary differential equation (ODE) model, simulations of these artificial ecosystems can potentially give insights on key biological features (i.e. parameters, e.g. growth rate, death rate, carbon-handling efficiencies) for potential candidates.  Then by matching wild microbial lineages with the bio-feature ranges, candidates can be filtered out easier.  Given the setting on plankton carbon sequestration, one should use an ODE model with the following features: aquatic, homogeneous environment, carbon density-based and position between theory and generalized reality.

ODE models are popular among fields studying interactive systems.  Oceanic mixed layer models are highly specific in locations and organisms.[REF]  Parameters are set according to the data collected from field sites.  It is known that ocean ecosystems are usually depleted in nitrogen and/or phosphorous.[REF]  Hence these models are built using nitrogen and/or phosphorous concentration(s) with organisms at the site locations.  For terrestrial ecosystems, ODE models are often focus on one individual autotroph.[REF]  The main reason is because atmospheric carbon and nitrogen nutrients are obtained at different parts of plants.  Hence material transportation within the individual limits the growth of the individual and hence the energy available to higher trophic levels.  By comparing the criteria of this study's target system and available models, the plankton-farm situation lies between the aquatic and terrestrial model groups.  Hence a new ODE model addressing the carbon cycle of an artificial plankton open system would be preferred.

Several assumptions are made, including: 1. living conditions in the system is homogeneous spatially, nutritionally, carbon and light availability; 2. nutrient supply is continuous and unlimited; and 3. the effect of high carbon density blocking light within the system is negligible.  In short, this model has only one environmental limitation -- living space for photocells.

This model aims at addressing the question of ``what are the parameter ranges for an artificial ecosystem with high carbon sequestration service?" with the following hypotheses:

\begin{enumerate}
    \item The idea of an ``artificial ecosystem" is theoretically self-sustainable;
    \item There are specific boundaries for each parameter; and
    \item Continuous carbon removal from the system can be achieved in a sustainable system.
\end{enumerate}

\end{document}