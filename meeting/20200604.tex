% Author        : PokMan Ho
% Script        : 20200604_progRep.tex
% Desc          : MRes progress report tex
% Input         : none
% Output        : pdf report in same directory
% Arguments     : 0
% Date          : May 2020

\documentclass[a4paper,11pt]{article}
\usepackage[margin=2cm]{geometry}
\usepackage[english]{babel}
\usepackage{graphicx, hyperref, longtable, amsmath, amssymb}
\graphicspath{{figure/}{figure/}}

\hypersetup{
	colorlinks=true,
	linkcolor=blue,
	filecolor=blue,      
	urlcolor=blue,
	citecolor=blue
}

\title{Thursday mornings MRes project progress report}
\author{PokMan Ho}
\date{}

\begin{document}
    \maketitle
    
    \begin{tabular}{rl}
        Date Time: & 04-Jun-2020 09:00 \\
        Location: & MS Teams \\
        People: & Samraat, Emma, PokMan \\
    \end{tabular}
    
    \section{Topics discussed}
    \begin{enumerate}
        \item result of analytical parameter scan
        \item conceptually what result would be worth investigation in the thesis
        \item use of ``maximum sustainable yield" (MSY, $xC$, unit gC/day) concept as target of investigation based on rate of carbon removal ($x$, unit day$^{-1}$)
    \end{enumerate}
    
    \section{Consensus}
    \begin{enumerate}
        \item take analytical scanned result and compare whether bacteria brings benefit to system of carbon sequestration:
        \begin{itemize}
            \item within the same parameter space, how many simulations would have MSY using pure phytoplankton (eqm position 2) vs co-culture (eqm position 4)
            \item probability of feasibility (fraction of situation having MSY) VS MSY
        \end{itemize}
        \item draw a cartoon illustration on how \& what extraction method the system is constructed in real life (IRL)
        \item able to layout ``why feasibility is important" in the model's context
    \end{enumerate}
    
    \section{notes}
    \begin{itemize}
        \item feasibility VS stability
        \begin{itemize}
            \item feasibility: max extraction, regardless of population as long as it is $>$ 0
            \item stability: optimal extraction with system sustainability on all resources
        \end{itemize}
        \item later may investigate recovery rate $\rightarrow$ use integration
    \end{itemize}
    
\end{document}
