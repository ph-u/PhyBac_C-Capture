
% Author: PokMan Ho pok.ho19@imperial.ac.uk
% Script: proposal.tex
% Desc: `LaTex` proposal for MRes 9-months project ver 2
% Input: none
% Output: pdf proposal in same directory
% Arguments: 0
% Date: Nov 2019

\documentclass[a4paper, 11pt]{article}

%% packages
\usepackage[margin=2cm]{geometry}

\usepackage{graphicx}
\graphicspath{{graphics/}}

\usepackage{fontspec}
\setmainfont{Arial}

\usepackage{hyperref, setspace, lineno, longtable}

%% hyperlinks
\hypersetup{
	colorlinks=true,
	linkcolor=blue,
	filecolor=blue,      
	urlcolor=blue,
	citecolor=blue
}

%% test insert variables
\newcommand{\ReportTitle}{Proposal for MRes CMEE Dissertation Project}
\newcommand{\ReportAuthor}{PokMan HO (CID: 01786076)}
\newcommand{\ReportAffil}{Department of Life Sciences, Faculty of Natural Sciences,\\Imperial College London}
\newcommand{\Disclaim}{\textbf{A Proposal submitted in partial fulfilment of the requirements for the degree of Master of Research at Imperial College London\\\vspace{1cm}
		Formatted in the journal style of the \textit{Nature} Journal\\
		Submitted for the MRes in Computational Methods of Ecology and Evolution}}
%% A Proposal submitted in partial fulfilment of the requirements for the degree of Master of Research at Imperial College London\\\\Formatted in the journal style of the \textit{Nature} Journal\\Submitted for the MRes in Computational Methods of Ecology and Evolution -- not accepting 4 backslashes

%% tabular colouring <https://tex.stackexchange.com/questions/94799/how-do-i-color-table-columns>
\usepackage{xcolor, colortbl}
\definecolor{grey90}{rgb}{.9,.9,.9}
\definecolor{lorange}{rgb}{1,.8,0}

%% 90-deg tabular tab rotation
%\usepackage{rotating, arraycols, booktabs, array}
%\newcolumntype{R}{1}{>{\begin{turn}{90}\begin{minipage}{#1}\scriptsize}l
%			<{\end{minipage}\end{turn}}}}

\title{\ReportTitle}
\author{\ReportAuthor}
\date{}

%% citation
\usepackage[%
autocite    = superscript,
backend     = bibtex,
sortcites   = true,
style       = nature
]{biblatex}
\bibliography{../reference/00_2019ICLMResProposal.bib}

%% set as required
\onehalfspacing %% <https://tex.stackexchange.com/questions/30073/why-is-the-linespread-factor-as-it-is>
\linenumbers

\begin{document}
	\begin{center}
		\Huge\textbf{\ReportTitle}\\
		\LARGE\ReportAuthor\\
		\Large\ReportAffil\\
		\Large{Last modified: Nov 2019}
	\end{center}
	\begin{figure}[h]
		\centering\includegraphics[width=\linewidth]{icl.jpg}
	\end{figure}
Primary supervisor:\\
\indent Dr. James Rosindell\\
\indent Department of Life Sciences (Silwood Park), Faculty of Natural Sciences, Imperial College London\\
\indent email: \href{mailto:j.rosindell@imperial.ac.uk}{j.rosindell@imperial.ac.uk}\\
Secondary supervisor:\\
\indent Dr. Samraat Pawar\\
\indent Department of Life Sciences (Silwood Park), Faculty of Natural Sciences, Imperial College London\\
\indent email: \href{mailto:s.pawar@imperial.ac.uk}{s.pawar@imperial.ac.uk}
%External adviser:\\
%\indent Prof. Wei Huang\\
%\indent Department of Engineering Science, University of Oxford, Parks Road, OX1 3PJ Oxford, United Kingdom\\
%\indent email: \href{mailto:wei.huang@eng.ox.ac.uk}{wei.huang@eng.ox.ac.uk}
\begin{center}
	\Large\Disclaim
\end{center}
\clearpage
\section{Keywords}
continuous-time model, ecological photovoltaics, carbon sequestration, co-existence, nutrient cycling
\section{Introduction}
Humans are destabilizing our climate\autocite{schuur2015climate}.  It is easy to create atmospheric carbons but hard to reverse\autocite{yang2008progress}.  Biological electricity generator is available\autocite{mccormick2015biophotovoltaics} yet still lagging behind the global energy thirst\autocite{ferguson2000electricity}.  This project aims at modelling an ecosystem which can extract energy from multiple solar bands.\\
Proposed research question: What is the recipe for making a multi-banded self-standing ecological solar panel?
\section{Proposed methods}
A continuous-time model will be scripted in R (ver 3.6.0).  It would be run in coarse and fine resolutions with different raw material combinations.
%Monitored chemical species and bio-component players will be listed in columns with each row as one time step.  Each chemical and biological components will be interlinked with one another using mole/unit ratios.  The model will be run for logical designated time-steps which could be translated to actual Earth time.  Short course-resolution models will be run as trials to get an overall picture on accuracy of achieving the global minimum of initial ingredients and maximum of carbon sequestration efficiency.  Fine-resolution models will be run in batches later after refinement of time steps to screen for possibilities in greater details.\\
\section{Anticipated outputs and outcomes}
Initial parameter(s) for simulations achieving dynamic equilibria would be exported as csv and cluster graphs.  Respective biochemical and population flux would be shown in networks.
%Expected Model output: csv listing all chemical and biological players population fluctuation at every time step for every simulation\\
%\begin{itemize}
%	\item a huge csv listing chemical species (i.e. resource) fluctuation at every time step for every simulation
%	\item a long csv listing population of each bio-players (i.e. consumers) at every time step for every simulation
%\end{itemize}
%Expected Analysis output:find out the optimal ingredient-to-efficiency combination\\
%\begin{itemize}
%	\item csv showing possible combinations of chemical species resulting in ecosystem equilibria
%	\item animated network graphs showing fluctuations before achieving equilibria from the ecosystem requiring the least amount of initial chemical species and highest carbon sequestration efficiency
%	\item cluster graphs showing carbon sequestration efficiency against raw ingredients
%	\item any possible pattern analysis results
%\end{itemize}
%Expected outcomes: know what and how much raw ingredients is needed to start an artificial carbon sequestering ecosystem
%\begin{itemize}
%	\item find the right ingredients for possible artificial minimal ecosystem with high carbon sequestration efficiency\\
%	\item find the correct amount of initial ingredients to start an artificial minimal ecosystem\\
%	\item calculate the theoretical carbon sequestration efficiency for the best artificial minimal ecosystem\\
%\end{itemize}

\section{Project feasibility}
Most details on the nutrient conditions for producers\autocite{duarte2009microbial,markou2014microalgal} and consumers\autocite{neff1957purification,mooshammer2014stoichiometric} were published.  Successful cyanobacteria biovoltaics panel details were also published\autocite{mccormick2015biophotovoltaics}.  Hence the project would be successful once data was included correctly in the model.\\
Key:
\begin{tabular}{lll}
	section & target & action\\\hline
	1 = abstract & m = discrete-time model & s = scripting\\
	2 = introduction & a = analysis & d = debug\\
	3 = methodology & p = parameters & v = validation\\
	4 = results && \cellcolor{grey90}work\\
	5 = discussion && \cellcolor{lorange}buffer / if any\\
\end{tabular}\\
Gantt chart:\\
\begin{longtable}{p{.1\linewidth}p{.1\linewidth}|p{.1\linewidth}|p{.1\linewidth}|p{.1\linewidth}|p{.1\linewidth}|p{.1\linewidth}|}
	%Month & Week & \multicolumn{1}{R{1cm}}{writing} & \multicolumn{1}{R{1cm}}{writing} & \multicolumn{1}{R{1cm}}{writing} & \multicolumn{1}{R{1cm}}{writing} & \multicolumn{1}{R{1cm}}{writing} & \multicolumn{1}{R{1cm}}{writing}\\\hline
	Month & Week & write & design & script & \begin{tabular}{c}MSc\\lecture\end{tabular} & \begin{tabular}{c}model\\run\end{tabular}\\\hline
	Dec & 2 && \cellcolor{grey90}p,a &&&\\
	& 3 && \cellcolor{grey90}p,m & \cellcolor{grey90}ms & \cellcolor{lorange} &\\
	& 4 && \cellcolor{grey90} & \cellcolor{grey90} & \cellcolor{lorange} &\\
	Dec-Jan & 5/1 && \cellcolor{grey90} & \cellcolor{grey90}md & \cellcolor{lorange} &\\
	& 2 && \cellcolor{grey90}m & \cellcolor{grey90} & \cellcolor{lorange} & \cellcolor{lorange}trial \\
	& 3 & \cellcolor{grey90}3 & \cellcolor{grey90} & \cellcolor{grey90}mv & \cellcolor{lorange} & \cellcolor{grey90}1st batch \\
	& 4 & \cellcolor{grey90} & \cellcolor{grey90}a && \cellcolor{lorange} & \cellcolor{grey90} \\
	& 5 & \cellcolor{grey90} & \cellcolor{grey90} && \cellcolor{lorange} & \cellcolor{grey90} \\
	Feb & 1 & \cellcolor{grey90}2 &&& \cellcolor{lorange} & \cellcolor{grey90} \\
	& 2 & \cellcolor{grey90} &&& \cellcolor{lorange} & \cellcolor{grey90} \\
	& 3 &&&& \cellcolor{lorange} & \cellcolor{grey90} \\
	& 4 &&&& \cellcolor{lorange} & \cellcolor{grey90} \\
	Mar & 1 &&&& \cellcolor{lorange} & \cellcolor{grey90} \\
	& 2 &&&& \cellcolor{lorange} & \cellcolor{grey90} \\
	& 3 & \cellcolor{lorange}3 & \cellcolor{lorange}a & \cellcolor{grey90}as & \cellcolor{lorange} & \cellcolor{grey90}2nd batch \\
	& 4 & \cellcolor{lorange} & \cellcolor{lorange} & \cellcolor{grey90} & \cellcolor{lorange} & \cellcolor{grey90} \\
	Mar-Apr & 5/1 & \cellcolor{grey90}45 && \cellcolor{grey90}a(trial) && \cellcolor{grey90} \\
	Apr & 2 & \cellcolor{grey90}5 && \cellcolor{grey90}a(1st) && \cellcolor{grey90} \\
	& 3 & \cellcolor{grey90} && \cellcolor{grey90} && \cellcolor{grey90} \\
	& 4 & \cellcolor{grey90} && \cellcolor{grey90} && \cellcolor{grey90} \\
	& 5 & \cellcolor{grey90} && \cellcolor{grey90} && \cellcolor{grey90} \\
	May & 1 & \cellcolor{grey90}45 && \cellcolor{grey90}a(2nd) && \cellcolor{grey90} \\
	& 2 & \cellcolor{grey90}5 && \cellcolor{grey90} && \cellcolor{lorange} \\
	& 3 & \cellcolor{grey90} && \cellcolor{grey90} && \cellcolor{lorange} \\
	& 4 & \cellcolor{grey90} && \cellcolor{grey90} && \cellcolor{lorange} \\
	Jun & 1 & \cellcolor{grey90} && \cellcolor{lorange} && \cellcolor{lorange} \\
	& 2 & \cellcolor{lorange} && \cellcolor{lorange} && \cellcolor{lorange}3rd batch \\
	& 3 & \cellcolor{lorange} &&&& \cellcolor{lorange} \\
	& 4 & \cellcolor{lorange} &&&& \cellcolor{lorange} \\
	Jun-Jul & 5/1 & \cellcolor{grey90}1 &&&& \cellcolor{lorange} \\
	Jul & 2 & \cellcolor{grey90} &&&& \cellcolor{lorange} \\
	& 3 & \cellcolor{grey90} &&&& \cellcolor{lorange} \\
	& 4 & \cellcolor{grey90} &&&& \cellcolor{lorange} \\
	& 5 & \cellcolor{lorange}any && \cellcolor{lorange}a(3rd) && \cellcolor{lorange} \\
	Aug & 1 & \cellcolor{lorange} && \cellcolor{lorange} && \cellcolor{lorange} \\
	& 2 & \cellcolor{lorange} && \cellcolor{lorange} && \cellcolor{lorange} \\
	& 3 & \cellcolor{lorange} &&&& \cellcolor{lorange} \\
	& 4 & \cellcolor{lorange} &&&&\\
\end{longtable}
\section{Budget}
This project is free
%\begin{tabular}{r|r@{}lc|l}
%	item & \multicolumn{2}{c}{cost (GBP)} && justification\\\hline
%	Crucial 8GB external RAM & 27&.90 && insufficient computing power on the personal computer\\
%\end{tabular}
\section*{Acknowledgement}
The author would like to thank \href{mailto:wei.huang@eng.ox.ac.uk}{Prof. Wei Huang} for agreeing to be the external adviser for this project.
\clearpage
\nocite{*}\printbibliography
\end{document}