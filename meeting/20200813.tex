\documentclass[a4paper,11pt]{article}
\usepackage[margin=2cm]{geometry}
\usepackage[english]{babel}
\usepackage{graphicx, hyperref, longtable, amsmath, amssymb}
\graphicspath{{figure/}{../result/}{../report/media/}}

\hypersetup{
	colorlinks=true,
	linkcolor=blue,
	filecolor=blue,      
	urlcolor=blue,
	citecolor=blue
}

\title{Progress report on Thursdays}
\author{PokMan Ho}
\date{}

\begin{document}
    \maketitle
    
    \begin{tabular}{rl}
        Date Time: & 13-Aug-2020 09:00 \\
        Location: & MS Teams \\
        People: & Samraat, Emma, PokMan; James \& PokMan on 12-Aug \\
    \end{tabular}
    
    \begin{longtable}{p{.2\linewidth}p{.7\linewidth}}\hline
    Content & Things to be done\\\hline
    project direction of the thesis & \begin{itemize}
        \item a ``proof-of-concept" study showing feasibility of a phytoplankton-bacteria co-culture and potential applications
        \item bacterial invasion: may have implications; need further investigation on invasion analysis
    \end{itemize}\\
    results & interpretation/explanation on legend \& text\\
    abstract & \begin{itemize}
        \item general framework of ``carbon capture and storage" (CCS) field
        \item how much is known in sequestration and biofuel $\rightarrow$ main limitations
        \item bacteria not in xxx applications, although they are used in xxx fields
        \item small Roman numbers for listing points
        \item I do ... The result shows ... [thesis use ``I", papers use ``we"]
    \end{itemize}\\
    introduction & expansion on abstract background part; on CCS and geo-engineering fields\\
    discussion [James, 20200812] & \begin{itemize}
        \item engineering with bio-component have less potential harm against geo-engineering: no interventions/perturbations on natural processes and feedback systems
        \item My result shows there's not a single systems serves all purposes; system choice depends on what users want to achieve and whether we can control bacterial life history traits
        \item limitations: cost, difficulty to scale up, potential problems, may able to find a fit bacteria from the nature but if not, genetic modification can be a way
        \item LCA brief outline: co-culture construction, maintenance, materials
        \item bacteria doesn't change construction cost, only bringing benefit if optimal candidate is chosen -- beneficial from a LCA calculation
        \item continuous/destructive harvest bring difference on system implementation (i.e. the way of harvest carbon), account for future work
    \end{itemize}\\
    \hline\end{longtable}
    
\end{document}