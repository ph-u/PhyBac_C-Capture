% 20200817
\documentclass[../thesis.tex]{subfiles} %% use packages & commands as this main file
\begin{document}

\subsection{Strengths and Constraints}
This model is spatial, temporal independent and ignores temperature variation; it can be applied across latitudes and longitudes where \phy\ and \bac\ can thrive.  The results provide a general overview on carbon yields within the feasibility limitations.  This provides a framework on how the model data can be interpreted and suggests potential criteria on selecting the optimal harvest parameters and life history traits of \phy\ and \bac.

Enhanced co-cultures were reported resilient towards \bac l invasion experimentally \autocite{fuentes2016impact,seyedsayamdost2011roseobacticides}.  The model in this study may have implications on invasion tolerance for \pbs s; the low feasibility (0.3\%) may indicate but not directly address this issue.  An invasion analysis is needed for confirmation this \pbs\ property.

The model can be engineered as a panel of co-culture with an optimal \phy-\bac\ pair; this design is applicable in urban area and on rooftops as a carbon capture device.  To construct such device, a life cycle assessment (LCA) should be needed.  Important factors in the LCA should include resource footprints of materials, logistics, construction, maintenance and disposal.  Carbon debt of the co-culture carbon capture panel includes the logistics for raw materials (\phy, \bac, chemicals and disposable laboratory apparatus), genetic modification materials and power used (if needed), co-culture assembly, device manufacture, delivery of the device to the installation location, installing the device, consistence use of machinery and maintenance for carbon harvest and finally disposal of the device.  It is worth noting that using \bac\ in the design has minor impact on the overall carbon footprint but a huge benefit on the carbon yield (Fig. \ref{f:ydDaily}).  Carbon benefit includes the total carbon yield harvest from the system during the service life and the carbon content in the disposed device.  However organic carbon can be consumed and respired by decomposers in nature and release back to the atmosphere.  Only carbon that are not easily consumable by organisms is the ``carbon benefit" of the device.

Method of carbon harvest is an important factor contributing to the carbon footprint \autocite{fuentes2016impact}.  Continuous harvest can reduce as much as a quarter of the carbon benefit from a feasible co-culture \autocite{mata2010microalgae}.  Existing harvest methods are using ultrafiltration membranes \autocite{zhang2010harvesting}, electrocoagulation and centrifugation (currently not deployable at large scale) \autocite{wijffels2010outlook} and carbon adsorption \autocite{mata2010microalgae,wang2012novel,lee2014repeated}.  These methods should be incorporated in an LCA and a cost-benefit analysis.  Although automated harvesting may be costly in funds and carbon footprint, we can always design and encourage harvesting by hand.  Harvesting by hand can reduce this harvest method cost to zero (because human effort is negligible in an LCA); the only requirement is to construct an invasion-resistant co-culture which would not genetically pollute the environment.  The co-culture should be invasion-resistant because the tools harvesting carbon by hand is likely to be contaminated by environmental microorganisms.  Co-culture resistance to \bac l invasion is a key to preserve the expected carbon yield and some co-cultures might already have this ecological property \autocite{fuentes2016impact,seyedsayamdost2011roseobacticides}.  Handling the co-culture by hand will also bring genetic materials out from the artificial co-culture which are not native to the area of installation.  If non-native genetic material is leaked from the co-culture to the environment during carbon harvest or container leakage, native microorganisms may have access to these environmental DNA and may fuse these material in their genomes.  If these recombined organisms win over native microorganisms, the native microorganism balance may shift and affect the higher level organisms.  The leakage of genetic material is called a ``genetic pollution".  It is an important concern if genetic modified \phy\ or \bac\ is used in the co-culture.

Broad assumptions are the main constraints for this model.  For example, unlimited nutrient may not always be valid because some \phy\ have specific nutrient requirements \autocite{kazamia2012mutualistic}.  Nutrient specificity of \phy\ may require specific \bacm\ or artificial fertilisers.  These additional criteria may make a \pbs\ using sub-optimal \bac, or even lowers the overall carbon yield because production of artificial fertilisers require energy (hence translate to carbon footprints in an LCA analysis).  Genetic modification may provide a solution but there is a risk of genetic pollution to local microbiomes if there is any leakage from the co-cultures in practice.

\end{document}