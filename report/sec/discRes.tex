% 20200816
\documentclass[../thesis.tex]{subfiles} %% use packages & commands as this main file
\begin{document}

Engineering with bio-components have less potential harm comparing with geo-engineering proposals because the former does not perturb nature.  Since the engineered devices with bio-components can be installed in cities and on rooftops, natural processes are kept untouched.  Hence natural balances and ecosystem feedback systems are not affected and the engineering with bio-components bear small risks of having large scale unintended consequences.  This study proposed a new model on laying out artificial and life history traits constrains when we want to engineer these carbon capture devices with living components.  The model quantitatively shows the importance of choosing the optimal \phy, \bac, harvest mode and harvest interval/rate when engineering an efficient carbon capture device.  The result shows there is not a single systems serves all purposes; the choice of system depends on what the users want to achieve and whether we can control the life history traits of \phy\ and \bac.

\subsection{Feasibility of \pbs s}
Feasibility of \pbs s are always lower than that of \phy-only systems (Table \ref{t:feasDist}).  The lower feasibility means a wild search of a \phy-\bac\ pair has a probability of system establishment.  The result shows the probability is 0.3\% using a continuous harvest mode and 97.2\% using a destructive harvest mode (Table \ref{t:feasDist}).  The small probability can potentially explain the few success cases in the experimental search of high carbon yield \phy-\bac\ pairs \autocite{fuentes2016impact}.  Life history traits of the \phy\ and \bac\ can be measured for these published pairs in co-cultures.  Such measurements can confirm whether these existing efficient co-cultures bear the trait values within the feasible ranges quantified in this study (Table \ref{t:ranges}).  The measurements can also test the hypothesis of whether chemical mediators from \phy\ and/or \bac\ are important for an effective carbon harvest engine in general \autocite{rivas2010interactions,amin2009photolysis,fuentes2016impact}.

\subsection{Harvest preferences of \phy\ and \phy-\bac\ \pbs s}
The result shows no harvest preferences for \phy-only systems (Fig.\ref{f:ydDaily}).  The reason is because \phy\ is a carbon pump in the model (Fig.\ref{f:model}).  The rate of carbon capture by \phy\ is depending on rate of photosynthesis, respiration, leakage, biomass growth and intraspecific interference.  These rates in the model are either fixed values (resource allocation, biomass growth and interspecific interference) or a combination of fixed parameters (photosynthesis, respiration and leakage rates).  Hence equilibrium positions of \phy\ biomass density should be similar between destructive and continuous harvest modes (Fig.\ref{f:ydByHarv}\&\ref{f:harvPo}).  The result cannot falsify that continuous harvest is a growing concept as an effective carbon harvest method \autocite{fuentes2016impact} but in practice aquaculture and geo-engineering proposals adopt the destructive harvest mode by default \autocite{lawrence2014efficiency,krause2016substantial}.

Continuous harvest mode is necessary for \phy-\bac\ \pbs s.  The result shows continuous harvest is significantly higher than destructive harvest among the feasible scenarios across all harvest intervals/rates (Fig.\ref{f:ydDaily}).  The effect of high product yield is also observed from the production of hydrogen \autocite{kim2008anaerobic}, carbohydrates \autocite{choix2012enhanced1,choix2012enhanced2} and fatty acids \autocite{leyva2014accumulation}.  The reason for the lack of consensus support is because of the small feasibility of \pbs s (Table \ref{t:feasDist}).  In publications reporting high yields, chemical analysis of the \pbs s is a popular methodology \autocite{santos2014microalgal,rivas2010interactions,leyva2014accumulation,amin2009photolysis}.  Without indications on life history traits of the organisms they have selected, it is hard to construct an effective carbon capture engine with living components.  This pilot study hence shows directly and quantitatively how continuous harvest is superior under defined ranges of \phy\ and \bac\ life history trait values.

\subsection{Importance of life history traits on the organic carbon yield}
Life history traits of \phy\ are important for \phy-only systems. An optimised \phy\ bring a significant higher average/daily organic carbon yield (Fig.\ref{f:bacEffect}).  In the model, a \phy-only system contains only the \phy\ half of Fig.\ref{f:model}.  Destructive carbon harvest both organic carbon ($C$) and \phy\ biomass ($P$) pool after days of accumulation while continuous harvest daily extract only the $C$ pool.  However content of $P$ is non-significant when compared to the $C$ pool (Fig.\ref{f:ydByHarv}).  Yield maximisation hence focuses on expanding the $C$ pool, which can be achieved through \Rn{1}) increase the rate of photosynthesis and \Rn{2}) reduce the rate of respiration.  Rate of photosynthesis can be increased by having more chlorophyll content for carbon capture.  This is achieved by applying fertilisers in geo-engineering proposals \autocite{gnanadesikan2008export,lawrence2014efficiency,trick2010iron,kwiatkowski2015atmospheric,lovelock2007ocean} or using a \phy\ with higher carbon-to-biomass ratio (Fig.\ref{f:bacEffect}B) or higher growth rate (Fig.\ref{f:bacEffect}C).  Rate of respiration can be reduced by lowering the \phy\ respiration carbon fraction (Fig.\ref{f:bacEffect}A), which is a factor currently lack of studies.  Intraspecific interference ($\aP$) is beneficial in smaller values (Fig.\ref{f:bacEffect}D).  In destructive system a low $\aP$ has pure benefit because carbon in the system are used to absorb more carbon into the system.  However in continuous harvest systems a high $\aP$ brings direct benefit to the $C$ pool (as \phy\ biomass contribute to the $C$ pool upon death) while a low $\aP$ only bring indirect benefit by contributing to \phy\ biomass.  More \phy\ can absorb more carbon into the system for either respiration, leakage (carbon directly into the $C$ pool) or into \phy\ biomass (further increasing photosynthesis rate).  The result shows that \phy\ biomass is functionally more important than its literal carbon content (Fig.\ref{f:ydByHarv}).

Life history traits of \phy\ and \bac\ are vital for feasible \pbs s.  Destructive harvest mode brings most carbon yield down to near zero (Table \ref{t:feasDist}), hence it has no functional importance in this study.  Continuous harvest mode can have high carbon yield with suitable \phy\ and \bac\ combinations (Fig.\ref{f:harvPB}), and has the highest expected yield among all systems in this study.  \PBH\ systems are Fig.\ref{f:model}, which the $C$ pool is supplied with leakages and dead cells from both \phy\ and \bac.  The result shows the best \phy-\bac\ \pbs\ maximises \phy\ and minimises \bac l carbon use efficiencies.  Model parameters can then be regrouped into two functional categories: demand and supply.

``Demand" parameters reduce the \pbs's future carbon capture ability which should be minimised.  $\aP$ is the short-term increasing $C$ pool at the expense of future carbon capture (Fig.\ref{f:harvPB}D).  $\eBR$ and $\eB$ are the \bac l carbon consumption parameters which directly reduce the carbon density in $C$ pool (Fig.\ref{f:harvPB}E-F).  ``Supply" parameters enhance the \pbs's future carbon capture ability which should be maximised.  $\mB$ is the short-term increase of $C$ pool and hinder the future \bac\ consumption on organic carbon; yet only a low death rate can support a sustainable population of \bac\ in a \pbs. $\ePR$ and $\eP$ are the \phy\ carbon capture parameters which directly enhances the carbon content in the \pbs s.  $\gP$ and $\gB$ are two special parameters which conditionally belong to ``demand" or ``supply".  Values lower than the optimal have positive effects (the ``supply" group) while those higher than the optimal have negative effects (the ``demand" group).  Reasons for $\gP$ having positive and $\gB$ having negative effects are logical because of their functional roles -- \phy\ is the producer while \bac\ is the consumer.  $\gP$ higher than optimal may analogous to a \phy\ bloom.  A bloom in the model system brings more carbon into the system but space is limited.  Hence the extra \phy\ cannot contribute to the $C$ pool and hence cannot be harvested.  This situation may provide a reason for a lower-than-expected yield in ocean fertilisation geo-engineering proposals \autocite{boyd2008implications,gnanadesikan2008export,oschlies2010side} because space limiting the maximum capacity of \phy.  $\gB$ lower than optimal may suggest a reason for the importance of \bac\ enhances carbon yield in \pbs s \autocite{fuentes2016impact,santos2014microalgal}.

\end{document}