% 20200813
\documentclass[../thesis.tex]{subfiles} %% use packages & commands as this main file
\begin{document}

Climate is changing due to the rocketing atmospheric carbon content and we need an effective solution to capture and store these carbon.  Many carbon capture solutions were proposed but we lack a consensus on which proposals truly work.  One of these plans is to use a \phy-\bac\ co-culture to capture atmospheric carbon.  However only a handful of co-cultures are found suitable.  To facilitate the search for carbon capture and storage, I have simulated the co-culture with a new mathematical model.  The model simulates the carbon flow in a \pbs\ with one \phy\ and one \bacm\ using several life history traits under two different yield harvesting methods -- destructive and continuous.  Destructive harvest collects all carbon content in the system at once; continuous harvest filters out some organic carbon from the system daily.  The result shows \Rn{1}) feasible \pbs s have specific \phy-\bac\ requirements; \Rn{2}) continuous carbon harvesting makes feasible \pbs s superior in carbon capture ability; \Rn{3}) feasible \pbs s maximise carbon yield by maximising \phy\ and minimising \bac\ carbon-use efficiencies.  This study uses life history trait values to address the published observations of enhanced carbon capture ability in a few of \phy-\bac\ co-cultures.  The result also provides a quantitative guide for future research on finding the optimal strategy, \phy\ and \bac\ for an effective carbon capture device to combat the current human-induced climate change.

\end{document}