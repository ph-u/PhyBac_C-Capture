% Author: PokMan Ho
% Script: conclu.tex
% Desc: MRes thesis conclusion section
% Input: none
% Output: none
% Arguments: 0
% Date: Jan 2020

\documentclass[../thesis.tex]{subfiles} %% use packages & commands as this main file

\begin{document}
\section{Conclusion}
%this is a conclusion
Carbon harvest ability of a \phy-only or \pbs\ can be quantitatively assess by our new model.  \Phy\ biology determines the carbon capture efficiency of a organic carbon harvest system while \bac\ biology determines feasibility of such system.  Destructive harvest is superior only when we cannot control over \bac l biology.  Continuous harvest gives resilience to \phy\ cultures towards \bac l invasion.  Carbon yield can largely be preserved in a \pbs\ when the optimal \bac\ was used.  Compared with \phy-only systems, \pbs s are having higher application potential in current industrial and commercial systems due to the multi-functional flexibility from \bac.

\end{document}
