% 20200713
\documentclass[../thesis.tex]{subfiles} %% use packages & commands as this main file
\begin{document}
\section{Introduction}

Climate change is affecting not only the ecosystem, but also the human society \autocite{notz2016observed,schuur2015climate}.  Extreme weather, season regime shifts and tropical disease spread towards temperate regions have intensified at a terrified rate.  Although we know carbon emissions are major contributors \autocite{notz2016observed}, we need fossil fuels to power our cities and industries \autocite{ferguson2000electricity}.  How can we minimise or even revert the damages to our climate?

For decades, experts have proposed geo-engineering plans for climate mitigation and adaptation \autocite{farrelly2013carbon,yang2008progress}.  Two major categories are distinguished \autocite{}, one on carbon capture and storage and the other one on solar radiation reduction.  Some suggested fertilising planktons \autocite{} or large autotrophs \autocite{}, some proposed releases of particles in high atmosphere as sunshield \autocite{}, some planned for fixing carbon using rocks \autocite{} or ocean \autocite{} and some put up ideas of filtering atmospheric carbon dioxide directly \autocite{}.  Debates and reviews are going on because we do not have a standard scheme and/or model to judge which method(s) work or not \autocite{}.  We are also afraid that some ideas may trigger unintended consequences \autocite{} and/or indirectly impact other Earth systems \autocite{}.  Proposals are also facing challenges from feasibility and social acceptability problems \autocite{}.  Expected efficiencies in some initial publications are also found exaggerated \autocite{} or having potential harms \autocite{}.  Hence in recent years scientists are still urging for emergency actions for our climate \autocite{}.

Although the situation is urgent, geo-engineering plans usually require a lot of power and raw materials \autocite{}.  Eventually, electricity and material extraction requirements symbolise the extensive use of fossil fuels.  The calculation method to estimate the overall benefits, carbon footprints and environmental damages within the lifetime of a device is called ``life-cycle assessment” (LCA).  Through LCA, not only the above mitigation measures do not have expected benefits, but also many renewable energies are not truly sustainable \autocite{abdussalam2020green}.  For example, manufacturing solar panels requires different chemicals and rare earth metals.  Material extraction, material delivery, panel assembly and all associated machines use a large amount of materials and power.  Hence the expected lifespan of the solar panel cannot cover all the environmental debt \autocite{martinopoulos2020rooftop}.  For a proper LCA, chemical pollution caused by the disposal of solar panels should be also taken into consideration.  Hence the only sustainable solution is to use the natural methods, which shines light on the use of \phy\ \autocite{farrelly2013carbon}.

\Phy\ can capture around double the amount of land plants \autocite{SCHLESINGER2013341} because plankton grow fast.  Plankton can also be bottled into different shapes \autocite{evanson_2019} due to their small size.  Although the solution sounds ideal, the \phy\ carbon capture device is exposed in air and \bac\ can invade into the system.  Hence we have to consider and mitigate the bacterial damages.  On the other hand, \bac\ is useful.  Different types of \bac\ are currently used in industries for pollution control \autocite{dash2013marine,naik2013lead}, medicine production \autocite{huang2012industrial} and electricity generation \autocite{songera2012electricity}.  If we can couple \phy\ with \bac, we can not only use the system for harvesting carbon but also to produce useful products for our society.  Coupling \phy\ and \bac\ is hence bear potentials for further investigations.

However, where to start?  Billions of \phy\ and \bac\ exist around the world and hence trial-and-error would take forever and be unaffordable.  In this case, we can seek for a simpler alternative by using mathematical models.  There are dozens of ecosystem models published in the past few decades.  Yet they are constructed based on nutrient cycles (nitrogen and/or phosphorus) in their respective project sites (Table \ref{modComp}).  So these published models do not fit our purpose because they cannot represent general \phy-\bac\ relationships and/or do not resemble the carbon flow within the system.  So to answer our question, we have proposed a new mathematical model.  The model simulates how carbon flows and recycles in a simple \pbs\ when we harvest carbon from it.

Using the model, we want to address three aims listed below.

Can \bac\ coexist with \phy\ without significantly changing the system’s maximum carbon harvest ability?  Carbon harvest in this context can be interpreted as either ``destructive harvest” or ``continuous harvest”.  Destructive harvest means we wait for some time, take down and replace the whole \pbs\ and extract all carbon content (total carbon) from the system.  Continuous harvest means we use some methods (e.g. filtration) to take a portion of organic carbon out from the ongoing system in regular time intervals (yield flux).  The question is targeted on organic carbon capture devices.  In those devices, a pure \phy\ culture is used \autocite{evanson_2019} and hence bacterial invasion can potentially cause unexpected malfunctioning.  We would also want to apply the result in industrial bacterial cultures, like those for production of medicine and electricity.  The main reason is because this model can suggest whether adding \phy\ as a carbon source for these \bac\ would be beneficial for the environment.  If this is true, adding \phy\ also means the factories do not need to buy sugar for the \bac\ and thus bringing down their production costs.

Does a \pbs\ prefer destructive or continuous harvest?  As defined above, destructive harvest is taking all carbon content out from the stabilised system and replacing it with a fresh culture.  This method is experimentally supported by the production of bacterial cellulose, which a semi-continuous growth media replacement increases the product yield within the same period of time \autocite{aytekin2016statistical}.  However the production of harvestable dead cellular content in a stable \pbs\ is theoretically high because the death rate for both \phy\ and \bac\ are at their maxima.  So we don’t need to wait for a new system to grow to its stable position before destroying it for carbon.

Are biological features of \phy\ and/or \bac\ significantly affecting the system’s carbon harvest function?  For this question we choose one of the carbon harvesting systems with \phy\ and \bac\ as our model setting to look into how the biological features influence the carbon output from it.  In which we will see how theoretically resource usage efficiencies, growth rates and death rates influence the model’s ability to harvest carbon.  At the same time, we compare the effect of having \bac\ in the system to estimate the expected impact or benefit by not excluding \bac\ from \phy\ cultures.  The result can also reflect on our current approach on choosing the ``right” \bac\ for designated functions.  Currently industries usually design a recombinant \bac\ and optimise its growth media for high productivity \autocite{dash2013marine,naik2013lead,huang2012industrial,evanson_2019}.  However the choice of \bac\ is often the one we understand the most \autocite{huang2012industrial}, which may be not the best candidate for the job.  Understanding the effect of biological features might provide a window for further optimisation and hence levelling up the productivity by modifying its resource efficiency and/or other features with indirect influence.

%Our literature review on relevant fields (i.e. ecosystem modeling, biofuel engineering and carbon sequestration science) reveal an empty field of research addressed by our exploratory questions.  Publications on ecosystem modeling are site-specific as discussed above (Table \ref{modComp}).  Because of their specificity, it is impossible to use the same models in other situations.  Apart from the above, there is a published microbial community model and a biofuel model.  With the community model, users can predict microbial community successions in laboratory settings using whole genome data \autocite{harcombe2014metabolic}.  Using the biofuel model, users can estimate the continuous biofuel extraction rate by inputting molecular features of a microbial strain \autocite{kirthiga2014mathematical}.  Both models are not suitable for our questions because they focus on well-known species while we want to explore the unknowns.  A brief search in Google Scholar on our focus (with keyphrase ``carbon sequestration" OR ``biofuel" ``yield" AND ``model" AND ``ordinary differential" AND ``equation" AND ``ecological" -``policy" in incognito mode excluding patents and citations from all time) resulted in 139 entries.  Yet none was adopting our method of using an ecological viewpoint to script a model simulating carbon capture.  Our focus is currently an unexplored interdisciplinary topic.

In this study, we have developed a general \phy-\bac\ model that resembles a simple real-life situation.  This model is a set of ordinary differential equations (ODE) describing an open system (matter can flow into and out from the defined system).  Integrating the model can simulate a forward-time prediction of carbon densities stored in the system while solving the equation set analytically can pinpoint the stable position of carbon densities in the system.  The mathematical form and a graphical representation of the model is shown in the Methods section.

\end{document}