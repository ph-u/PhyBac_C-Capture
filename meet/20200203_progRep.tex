% Author        : PokMan Ho
% Script        : 20200203_progRep.tex
% Desc          : MRes Proj progress report tex
% Input         : none
% Output        : pdf report in same directory
% Arguments     : 0
% Date          : Feb 2020

\documentclass[a4paper,11pt]{article}
\usepackage[margin=2cm]{geometry}
\usepackage[utf8]{inputenc}
\usepackage[english]{babel}
\usepackage{graphicx, hyperref, longtable, amsmath, amssymb}
\graphicspath{{result/}{../result/}}

\hypersetup{
	colorlinks=true,
	linkcolor=blue,
	filecolor=blue,      
	urlcolor=blue,
	citecolor=blue
}

\title{Progress report 01}
\author{PokMan Ho}
\date{03-Feb-2020}

\begin{document}
    \maketitle
    
    \begin{tabular}{rl}
        Date Time: & 03-Feb-2020 13:50 \\
        Location: & Boardroom, Silwood Park ICL \\
        People: & Dr. James Rosindell, PokMan Ho \\
    \end{tabular}
    \section{Follow-up items}
    None
    \section{Topics discussed}
    \begin{enumerate}
        \item Global solar data hourly time series (1947-2019) data handling
        \item Model exploration
    \end{enumerate}
    \section{Consensus}
    \begin{enumerate}
        \item Test limits of each model parameter on Jyputer
        \item Plot graphs of model behaviour and summarize model limit results
        \item Also get global temperature hourly data and incorporate it into model as a ``energy loss/expenses" term
        \item test model stability when population gets partial destruction
    \end{enumerate}
    \section{Work to do (expected finishing date)}
    \begin{itemize}
        \item Expand and incorporate temperature as energy loss term into current working equation system (before Wednesday, 05-Feb)
        \item Test limits for individual, multiple parameters; graph them with explanations in a Jyputer notebook (before Wednesday, 05-Feb)
        \item download temperature data and fuse with geo-mapped solar irradiation data (after Wednesday, 05-Feb)
    \end{itemize}
\end{document}
