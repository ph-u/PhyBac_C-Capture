
% Author: PokMan Ho pok.ho19@imperial.ac.uk
% Script: proposal.tex
% Desc: `LaTex` proposal for MRes 9-months project ver 3
% Input: none
% Output: pdf proposal in same directory
% Arguments: 0
% Date: Dec 2019

\documentclass[a4paper, 11pt]{article}

%% packages
\usepackage[margin=2cm]{geometry}

\usepackage{graphicx}
\graphicspath{{graphics/}}

\usepackage{fontspec}
\setmainfont{Arial}

\usepackage{hyperref, setspace, lineno, longtable, amsmath, amssymb}

%% hyperlinks
\hypersetup{
	colorlinks=true,
	linkcolor=blue,
	filecolor=blue,      
	urlcolor=blue,
	citecolor=blue
}

%% test insert variables
\newcommand{\ReportTitle}{Proposal for MRes CMEE Dissertation Project}
\newcommand{\ReportAuthor}{PokMan HO (CID: 01786076)}
\newcommand{\ReportAffil}{Department of Life Sciences, Faculty of Natural Sciences,\\Imperial College London}
\newcommand{\Disclaim}{\textbf{A Proposal submitted in partial fulfilment of the requirements for the degree of Master of Research at Imperial College London\\\vspace{1cm}
		Formatted in the journal style of the \textit{Nature} Journal\\
		Submitted for the MRes in Computational Methods of Ecology and Evolution}}
\newcommand{\As}{\textit{Anabaena sp.}}
\newcommand{\Ct}{\textit{Chroococcidiopsis thermalis}}
\newcommand{\Cs}{\textit{Chroococcidiopsis sp.}}

%% tabular colouring <https://tex.stackexchange.com/questions/94799/how-do-i-color-table-columns>
\usepackage{xcolor, colortbl}
\definecolor{grey90}{rgb}{.9,.9,.9}
\definecolor{lorange}{rgb}{1,.8,0}

%% 90-deg tabular tab rotation
%\usepackage{rotating, arraycols, booktabs, array}
%\newcolumntype{R}{1}{>{\begin{turn}{90}\begin{minipage}{#1}\scriptsize}l
%			<{\end{minipage}\end{turn}}}}

\title{\ReportTitle}
\author{\ReportAuthor}
\date{}

%% citation
\usepackage[%
%autocite 	= superscript,
backend 	= bibtex,
sortcites 	= true,
style 		= authoryear %nature
]{biblatex}
\bibliography{../reference/proposal.bib}

%% set as required
\onehalfspacing %% <https://tex.stackexchange.com/questions/30073/why-is-the-linespread-factor-as-it-is>
\linenumbers

\begin{document}
	\begin{center}
		\Huge\textbf{\ReportTitle}\\
		\LARGE\ReportAuthor\\
		\Large\ReportAffil\\
		\Large{Last modified: Dec 2019}
	\end{center}
	\begin{figure}[h]
		\centering\includegraphics[width=\linewidth]{icl.jpg}
	\end{figure}
Primary supervisor:\\
\indent Dr. James Rosindell\\
\indent Department of Life Sciences (Silwood Park), Faculty of Natural Sciences, Imperial College London\\
\indent email: \href{mailto:j.rosindell@imperial.ac.uk}{j.rosindell@imperial.ac.uk}\\
Secondary supervisor:\\
\indent Dr. Samraat Pawar\\
\indent Department of Life Sciences (Silwood Park), Faculty of Natural Sciences, Imperial College London\\
\indent email: \href{mailto:s.pawar@imperial.ac.uk}{s.pawar@imperial.ac.uk}
\clearpage
\section{Keywords}
ODE model, ecological photovoltaics, electricity, carbon sequestration, co-existence, nutrient cycling
\section{Introduction}
Both \textit{Synechocystis sp.} on carbon nanotube-coated paper \autocite{sawa2017electricity} and \As on button mushrooms \autocite{joshi2018bacterial} are found generating currents under the sun (micro- and nano-ampere respectively).  Under the threat of climate change \autocite{schuur2015climate}, we need electricity and carbon sequestration services happening together fast.  Augmenting photosynthesis by inserting carbon nanotubes inside phytocells \autocite{giraldo2014plant} is possible but not environmentally friendly in terms of material life-cycles.  Primitive cyanobacteria \Cs\ were found thermal resistant \autocite{baque2013boss} and \Ct\ was confirmed utilizing far red solar spectra for photosynthesis \autocite{nurnberg2018photochemistry}.  Nature is probably having the materials to solve our big problems.\\

With ecology in mind, this project aims at finding the theoretical recipe to construct a multi-spectra bio-solar panel with carbon sequestration ability.  The raw idea is using mushrooms' umbrella-shaped pileus to immobilize and stimulating photocurrents generation from photocell co-existing colonies of \As\ and \Cs.  By using a separate Lactobacillus fermentation chamber, old mushrooms can be digested to irrespirable lactates \autocite{senthuran1997lactic} for carbon sequestration while new mushrooms keep replacing old ones with newly-established photocell colonies.  The photocell ecosystem can be assumed using domestic wastewater as an all-rounded fertilizer \autocite{markou2014microalgal}.\\

With this system in mind, the research questions are 1. When would the coexistence system be stabilized after initialisation; 2. How do \As\ and \Ct\ interact; 3. What is the expected performance on electricity generation based on this ecosystem; and 4. What is the expected performance on carbon sequestration based on this ecosystem (if time allows).
%\begin{itemize}
%	\item When would the coexistence system be stabilized after initialisation?
%	\item How do \As\ and \Ct\ interact?
%	\item What is the expected performance on electricity generation based on this ecosystem?
%	\item What is the expected performance on carbon sequestration based on this ecosystem? (if time allows)
%\end{itemize}

\section{Proposed methods}
A set of ordinary differential equations (ODEs) would be implemented as a Lotka-Volterra model (LVM).  Main equations were:\\
%% eq 1,2 	: rate = growth - natural death - competition - substrate death
%% eq 3 	: rate = growth - natural death
For \As:
\begin{equation}
	dA/dt = [r_A A] - [k_{A1} A] - [k_{A2} A C] - [K_M]
\end{equation}
For \Ct:
\begin{equation}
	dC/dt = [r_C C] - [k_{C1} C] - [k_{C2} A C] - [K_M]
\end{equation}
For button mushroom:
\begin{equation}
	dM/dt = [r_M M] - [k_M M]
\end{equation}

$dA/dt, dC/dt$ and $dM/dt$ were instantaneous growth rates.  $[r_A A], [r_C C]$ and $[r_M M]$ were natural growth rates.  $[k_{A1} A], [k_{C1} C]$ and $[k_M M]$ were natural death rates.  $[k_{A2} A C]$ and $[k_{C2} A C]$ were growth hinder terms on spatial competition.  $[K_M]$ was cohort removals when host mushrooms died.  This model would be validated by having population cycles following the insolation cycles.\\

Assuming both cyanobacteria lineages only compete by growth, competition coefficients $k_{A2}$ \& $k_{C2}$ can be estimated by their differences in light harvesting ability (a fraction).  The fraction would be positively depending on their growth rates:
\begin{tabular}{cc}
	$k_{A2} = \dfrac{J_{Ct}}{J_{As} + J_{Ct}}\cdot r_A$ \&& $k_{C2} = \dfrac{J_{At}}{J_{As} + J_{Ct}}\cdot r_C$
\end{tabular}, which energy harvested by \As\ per cell per unit time was $J_{As}$ and that by \Ct\ was $J_{Ct}$.\\
Expected current size can be estimated by ratio, assuming the printed \As\ layer was one-cell thick in the reference paper \autocite{joshi2018bacterial}.  By known cell size of \As, one can estimate current generated per cell by estimating population size on one pileus.  Bringing the solar spectrum under consideration, solar to current energy conversion ratio can be obtained.  On the other hand, carbon sequestration ability could also be estimated through biomass growth \autocite{markou2014microalgal} and Lactobacillus fermentation efficiency \autocite{senthuran1997lactic}.  This numeric description would be carried out if time allows.

\section{Anticipated outputs and outcomes}
From the simulation of successful coexistence, there would be a graph of population sizes for all ecological parties upon solving the equations.  Model will be validated if population fluctuations within the ecosystem followed (daily,) monthly, seasonal and annual insolation cycles for chosen locations.

\section{Project feasibility}
Parameter data was available in literature.  Hence completing the project within time limit is considered positive.

Key:
\begin{tabular}{lll}
	section & target & action\\\hline
	1 = abstract & m = model & s = scripting\\
	2 = introduction & a = analysis & d = debug\\
	3 = methodology & p = parameters & v = validation\\
	4 = results && \cellcolor{grey90}work\\
	5 = discussion && \cellcolor{lorange}buffer / if any\\
\end{tabular}\\
Gantt chart:
\begin{tabular}{r|ccccccccc}
	Month		&Dec						&Jan						&Feb					&Mar					&Apr					&May				&Jun					&Jul					&Aug				\\\hline
	Write		&							&\cellcolor{grey90}3		&\cellcolor{grey90}2	&\cellcolor{lorange}3	&\cellcolor{grey90}4,5	&\cellcolor{grey90}	&\cellcolor{grey90}		&\cellcolor{grey90}1	&\cellcolor{lorange}\\
	Design		&\cellcolor{grey90}m,a,p	&\cellcolor{grey90}m		&\cellcolor{grey90}a	&\cellcolor{lorange}	&						&					&						&						&					\\
	Script		&\cellcolor{grey90}m,s		&\cellcolor{grey90}m,d,v	&						&\cellcolor{grey90}a,s	&\cellcolor{grey90}		&\cellcolor{grey90}	&\cellcolor{lorange}	&						&					\\
	MSc lecture	&\cellcolor{lorange}		&\cellcolor{lorange}		&\cellcolor{lorange}	&\cellcolor{lorange}	&						&					&						&						&					\\
	Model run 	&							&\cellcolor{grey90}			&\cellcolor{grey90}		&\cellcolor{grey90}		&\cellcolor{grey90}		&\cellcolor{grey90}	&\cellcolor{lorange}	&\cellcolor{lorange}	&\cellcolor{lorange}\\
\end{tabular}
\section{Budget}
This project does not require any compulsory funds.
\clearpage
\section*{Acknowledgement}
The author would like to thank \href{mailto:wei.huang@eng.ox.ac.uk}{Prof. Wei Huang} for agreeing to be the external adviser for this project.
\nocite{*}\printbibliography
\end{document}
