% 20200813
\documentclass[../thesis.tex]{subfiles} %% use packages & commands as this main file
\begin{document}

Climate is changing due to the rocketing atmospheric carbon content and we need an effective solution to capture and store these carbon.  Many carbon capture solutions were proposed but we lack a consensus on which proposals truly work.  One of these plans is to use a \phy-\bac\ co-culture to capture atmospheric carbon as biofuel.  However only a handful of co-cultures are found suitable.  To facilitate the future search for appropriate \phy\ and \bac\ for carbon capture and storage, I have simulated the co-culture with a new mathematical model.  The model simulates the carbon flow in a \pbs\ with one \phy\ and one \bac\ using several life history traits under two different yield harvesting methods -- destructive and continuous.  Destructive harvest is collecting all carbon content in the system at once; continuous harvest is filtering out some organic carbon from the system daily.  The result shows \Rn{1}) destructive harvest is superior when \bac l life history traits are uncontrollable; \Rn{2}) continuous harvest makes most \pbs s unfeasible but the feasible scenarios have an expected yield higher than pure \phy\ systems; \Rn{3}) high yield is possible when an optimal \bac\ is co-cultured with \phy.  This proof-of-concept study shows that \bac\ is beneficial to \phy\ cultures when the \bac\ has the life history traits with optimal functional values.  Further research are needed on finding the optimal \bac, how to construct an effective carbon harvest apparatus for the co-culture and how high is the potential benefit.  If the co-culture has a bigger carbon capture and storage benefit than geo-engineering methods, these optimal co-cultures can be proposed as an effective solution to combat the current human-induced climate change.

\end{document}