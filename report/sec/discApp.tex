% 20200817
\documentclass[../thesis.tex]{subfiles} %% use packages & commands as this main file
\begin{document}

\subsection{Strengths, Constrains and Future directions}
Selection of \phy, \bac, harvest mode and harvest rates are important factors in the engineering of an artificial carbon capture device.  \Phy\ living space is a major limitation hindering carbon yield under very high \phy\ growth rate.  This model is a pilot quantitative study addressing the observations of enhanced carbon yield by \phy-\bac\ co-cultures \autocite{fuentes2016impact,santos2014microalgal}.

This model is spatial, temporal and temperature independent.  It can be applied to anywhere on Earth with daily solar supply and appropriate parameter values.  The result suggested that chemical mediators might not be as important as previously thought \autocite{fuentes2016impact}; an optimal pair of \phy\ and \bac\ should also achieve the same enhanced carbon capture effect as reported in published co-cultures \autocite{amin2009photolysis,rivas2010interactions,seyedsayamdost2011roseobacticides}.

Enhanced co-cultures were reported resilient towards \bac l invasion experimentally \autocite{fuentes2016impact,seyedsayamdost2011roseobacticides}.  The model in this study may have implications on invasion tolerance for \pbs s; the small number of feasible scenarios may hint the feasibility but the model does not directly address the invasion tolerance issue.  An invasion analysis should be done to confirm this speculation.

Blanketed assumptions are the main constrains for this model.  For example, unlimited nutrient is a board term with little details.  Some \phy\ have specific nutrient requirements \autocite{kazamia2012mutualistic} which require specific \bac l strains or artificial fertilisers.  These additional criteria may make a \pbs\ using sub-optimal \bac, or even lowers the overall carbon yield because production of artificial fertilisers require energy (hence translate to carbon footprints).  Genetic modification may provide a solution but there is a risk of genetic pollution to local microbiomes if there is a culture leakage from the functioning co-cultures in practice.

The model can be engineered as a co-culture panel with an optimal \phy-\bac\ pair; this design is applicable in urban area and on rooftops as a carbon capture device.  To construct such device for environmental purpose, a life cycle assessment (LCA) should be needed.  Important factors in the LCA should include resource footprints of materials, logistics, construction, maintenance and disposal.  Carbon debt of the co-culture carbon capture panel includes the delivery of \phy\ (and \bac) strains in laboratory, genetic modification materials and power used (if needed), co-culture assembly, device manufacture, delivery of the device to installation location, installing the device, machinery to harvest carbon from the system along the service life and finally the disposal logistics.  Note that \bac\ does not change much of the oveall carbon footprint; addition of an optimal \bac\ for a feasible co-culture system only brings benefit to the carbon capture device.  Carbon benefit includes the total carbon yield harvest from the system during the service life and the carbon content in the disposed device.  However organic carbon can be consumed and respired by decomposers in nature and release back to the atmosphere.  Only non-bioavailable carbon can be considered as ``carbon benefit" of the device.

Method of carbon harvest is an important factor contributing to the carbon footprint \autocite{fuentes2016impact}.  Continuous harvest can reduce as much as a quarter of the carbon benefit gained by a feasible co-culture \autocite{mata2010microalgae}.  Existing harvesting methods are using ultrafiltration membranes \autocite{zhang2010harvesting}, electrocoagulation and centrifugation (currently not deployable at large scale) \autocite{wijffels2010outlook} and carbon adsorption \autocite{mata2010microalgae,wang2012novel,lee2014repeated}.  These methods should be incorporated in an LCA and a cost-benefit analysis.  Although automated harvesting may be costly in funds and carbon footprint, we can always design and encourage harvesting by hand.  Harvesting by hand can reduce this harvest method cost to zero (because human effort is negligible in an LCA); the only requirement is to construct an invasion-resistant co-culture which would not genetically pollute the environment.  The co-culture should be invasion-resistant because the tools harvesting carbon by hand is likely to be contaminated by environmental microorganisms.  Resistance is a key to preserve the expected carbon yield; co-cultures might have this property already \autocite{fuentes2016impact,seyedsayamdost2011roseobacticides}.  Handling the co-culture will also bring genetic materials out from the artificial co-culture which are not native to the area of installation.  If non-native genetic material is leaked from the co-culture to the environment during carbon harvest or container leakage, native microorganisms may have access to these environmental DNA and may fuse these material in their genomes.  If these recombined organisms win over native microorganisms, the native microorganism balance may shift and affect the higher level organisms.  The leakage of genetic material is called a ``genetic pollution".  It is an important concern if genetic modified \phy\ or \bac\ is used in the co-culture.

\end{document}