% 20200713
\documentclass[../thesis.tex]{subfiles} %% use packages & commands as this main file
\begin{document}
\section{Introduction}

Climate change is affecting not only the ecosystem, but also human society \autocite{notz2016observed,schuur2015climate}.  Extreme weather, season regime shifts and tropical disease have spread towards temperate regions have intensified.  Although we know carbon emissions are major climate change contributors \autocite{notz2016observed}, we currently need fossil fuels to power our cities and industries \autocite{ferguson2000electricity}.  Minimise or even revert the damages to our climate is a urgent task to complete.

For decades, experts have proposed geo-engineering plans for climate mitigation and adaptation \autocite{farrelly2013carbon,yang2008progress,boyd2008ranking,boettcher2019high,vaughan2011review}.  Two major categories were distinguished \autocite{boyd2008ranking}, one on carbon capture and storage and the other one on solar radiation reduction.  Debates and reviews are still ongoing because we do not have a standard scheme and/or model to judge which method(s) work or not \autocite{boyd2008ranking,boettcher2019high,boyd2016development,oschlies2017indicators,gattuso2018ocean}.

Fertilising planktons \autocite{gnanadesikan2008export,lovelock2007ocean,lawrence2014efficiency} or large autotrophs \autocite{duarte2017can,johannessen2016geoengineering,krause2016substantial} can compensate carbon absorption from bulldozed ecosystems.  But the expensive operation may have a much lower yield than expected \autocite{boyd2008implications,gnanadesikan2008export,oschlies2010side}.  Fertilisation may cause water column acidification \autocite{oschlies2010side} and imbalanced local resource competition \autocite{chung2011using,thiele2012microbial,batten2014did}, destroying local ecosystems.  Piping deep ocean water to the surface may also disrupt atmospheric pressure balance \autocite{kwiatkowski2015atmospheric}.  In high-nitrate-low-chlorophyll areas, fertilisation may even cause toxic algal blooms \autocite{trick2010iron}.

Releasing particles in high atmosphere and clouds as sunshield \autocite{latham2008global,stjern2018response,xia2016stratospheric} can reduce incoming solar radiation which theoretically halt greenhouse effect \autocite{williamson2012impacts}.  Yet the expensive blanketing \autocite{boyd2008implications} may cause unpredictable global primary production feedback in the short term and weaken the deep ocean thermal circulation in the long term \autocite{lauvset2017climate,williamson2012impacts}.  Rainwater chemistry and natural climate patterns can also be altered for a long time with unpredictable impacts on local terrains and ecosystems \autocite{bala2011albedo}, such as salt damages to rainforests and soil \autocite{muri2015tropical}.  Thermal-sensitive species and associated food webs may collapse due to rapid changes in light intensity and local temperatures \autocite{williamson2012impacts}.  In simulations, such global operations can bring ongoing extreme weather to further extremes in both human settlements and natural ecosystems \autocite{jones2009climate}.

Fixing carbon using rocks \autocite{stephens2008assessing,gunnarsson2018rapid} or ocean \autocite{salter200920,sant2014offshore} can amplify the current rate of carbon storage in natural sinks.  But sudden increase in local carbon dioxide concentration will acidify the seafloor, dissolving major alkaline rock basalt and may destroy local ecology \autocite{trias2017high,ridgwell2011geographical}.  The injection may also alter long-term groundwater chemistry in wider area \autocite{trias2017high}.

Directly filtering atmospheric carbon dioxide \autocite{broecker2007co2,williamson2016emissions} can make the smallest change in natural environment.  Yet carbon footprints from manufacturing the chemicals used for carbon capture are not taken into consideration \autocite{stephens2008assessing}.

Public perception on the above proposals are also obstacles for these methods to be tested \autocite{kamishiro2009public}.  Difference in benefit-cost judgement, community ethics and faith on the organisations carrying out such operations cause oppositions.  When the scientific community is still debating on the geo-engineering plans, it is logical that the general public do not support any of them.  Hence in recent years scientists are still urging for emergency actions for our climate \autocite{lenton2019foresight,van2017open}.

Geo-engineering plans usually require a lot of power and raw materials \autocite{boyd2008ranking,boyd2008implications,mcclellan2012cost}.  Electricity and material extraction requirements symbolise the extensive use of fossil fuels.  Calculation of the overall benefits, carbon footprints and environmental damages within the lifetime of a device is a ``life-cycle assessment” (LCA).  Through LCA, most of the geo-engineering mitigation plans and renewable energies are unsustainable \autocite{abdussalam2020green}.  For example, manufacturing solar panels requires different chemicals and rare earth metals.  Material extraction, material delivery, panel assembly and all associated machines use a large amount of materials and power.  Hence the expected lifespan of the solar panel cannot cover all the environmental debt \autocite{martinopoulos2020rooftop}.  For a proper LCA, chemical pollution caused by the disposal of solar panels should be also taken into consideration.  Hence a potential sustainable solution is to use the natural methods, for example the \phy\ \autocite{farrelly2013carbon}.

\Phy\ can capture around double the amount of carbon than that of land plants \autocite{SCHLESINGER2013341} because of the high growth rate.  Fluid plankton cultures can also be easily shaped by different containers \autocite{evanson_2019}.  Although the solution sounds ideal, the \phy\ carbon capture device is exposed in air and \bac\ can invade into the system.  Hence \bac l damages are an important problem to mitigate.  On the other hand, \bac\ is useful.  Different types of \bac\ are currently used in industries for pollution control \autocite{dash2013marine,naik2013lead}, medicine production \autocite{huang2012industrial} and electricity generation \autocite{songera2012electricity}.  Currently microbial engineering studies are focusing on pure cultures and the usage of single species.  Hence our project set foot on the non-ideal situations, when \bac\ and \phy\ coexist in the same culture for harvesting atmospheric carbon.

However, where to start?  Billions of \phy\ and \bac\ exist around the world and hence trial-and-error would take forever and be unaffordable.  In this case, we can seek for a simpler alternative by using mathematical models.  There are dozens of ecosystem models published in the past few decades.  Yet they are constructed based on nutrient cycles (nitrogen and/or phosphorus) in their respective project sites (Table \ref{modComp}).  So these published models do not fit our purpose because they cannot represent general \phy-\bac\ relationships and/or do not resemble the carbon flow within the system.  So to answer our question, we have proposed a new mathematical model.  The model simulates how carbon flows and recycles in a simple \pbs\ when we harvest carbon from it.

Using the model, we want to address three aims listed below.

Can \bac\ coexist with \phy\ without significantly changing the system’s maximum carbon harvest ability?  Carbon harvest in this context can be interpreted as either ``destructive harvest” or ``continuous harvest”.  Destructive harvest means we wait for some time, take down and replace the whole \pbs\ and extract all carbon content (total carbon) from the system.  Continuous harvest means we use some methods (e.g. filtration) to take a portion of organic carbon out from the ongoing system in regular time intervals (yield flux).  The question is targeted on organic carbon capture devices.  In those devices, a pure \phy\ culture is used \autocite{evanson_2019} and hence bacterial invasion can destroy its function.  We would also want to apply the result in industrial bacterial cultures, like those for production of medicine and electricity.  The main reason is because this model can suggest whether adding \phy\ as a carbon source for these \bac\ would be beneficial for the culture.  If this is true, adding \phy\ also means the factories do not need to buy sugar for the \bac\ and thus bring down their production costs.

In what conditions would a \pbs\ prefer destructive or continuous harvest?  As defined above, destructive harvest is taking all carbon content out from the stabilised system and replacing it with a fresh culture.  This method is experimentally supported by the production of bacterial cellulose, which a semi-continuous growth media replacement increases the product yield within the same period of time \autocite{aytekin2016statistical}.  However the production of harvestable dead cellular content in a stable \pbs\ is theoretically high because the death rate for both \phy\ and \bac\ are at their maxima.  So we don’t need to wait for a new system to grow to its stable position before destroying it for carbon.

Are biological features of \phy\ and/or \bac\ significantly affecting the system’s carbon harvest function?  For this question we use destructive harvest systems and see how \bac\ influence the carbon yield.  We also compare \pbs s to investigate how harvest mode influence the carbon yield.  All result reflect on how theoretical resource usage efficiencies, growth rates and death rates influence the carbon yield.  Industries usually design a recombinant \bac\ and optimise its growth media for high productivity \autocite{dash2013marine,naik2013lead,huang2012industrial,evanson_2019}.  However the choice of \bac\ is often the one we understand the most \autocite{huang2012industrial}, which may be not the best candidate for the job.  Deciphering the intertwining effect between \bac, harvest mode and biological features in this project might provide insight on industrial optimisation of the \bac\ on top of culture refinements.  This is potentially a way to enhance productivity and lower production cost.

%Our literature review on relevant fields (i.e. ecosystem modeling, biofuel engineering and carbon sequestration science) reveal an empty field of research addressed by our exploratory questions.  Publications on ecosystem modeling are site-specific as discussed above (Table \ref{modComp}).  Because of their specificity, it is impossible to use the same models in other situations.  Apart from the above, there is a published microbial community model and a biofuel model.  With the community model, users can predict microbial community successions in laboratory settings using whole genome data \autocite{harcombe2014metabolic}.  Using the biofuel model, users can estimate the continuous biofuel extraction rate by inputting molecular features of a microbial strain \autocite{kirthiga2014mathematical}.  Both models are not suitable for our questions because they focus on well-known species while we want to explore the unknowns.  A brief search in Google Scholar on our focus (with keyphrase ``carbon sequestration" OR ``biofuel" ``yield" AND ``model" AND ``ordinary differential" AND ``equation" AND ``ecological" -``policy" in incognito mode excluding patents and citations from all time) resulted in 139 entries.  Yet none was adopting our method of using an ecological viewpoint to script a model simulating carbon capture.  Our focus is currently an unexplored interdisciplinary topic.

A new general \phy-\bac\ model resembles a simple real-life situation was developed.  This model is a set of ordinary differential equations (ODE) describing an open system (matter can flow into and out from the defined system).  Integrating the model can simulate a forward-time prediction of carbon densities stored in the system while solving the equation set analytically can pinpoint the stable position of carbon densities in the system.  The mathematical form and a graphical representation of the model is shown in the Methods section.

\end{document}