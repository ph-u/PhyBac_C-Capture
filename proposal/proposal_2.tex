
% Author: PokMan Ho pok.ho19@imperial.ac.uk
% Script: proposal.tex
% Desc: `LaTex` proposal for MRes 9-months project ver 3
% Input: none
% Output: pdf proposal in same directory
% Arguments: 0
% Date: Dec 2019

\documentclass[a4paper, 11pt]{article}

%% packages
\usepackage[margin=2cm]{geometry}

\usepackage{graphicx}
\graphicspath{{graphics/}}

\usepackage{fontspec}
\setmainfont{Arial}

\usepackage{hyperref, setspace, lineno, longtable, amsmath, amssymb}

%% hyperlinks
\hypersetup{
	colorlinks=true,
	linkcolor=blue,
	filecolor=blue,      
	urlcolor=blue,
	citecolor=blue
}

%% test insert variables
\newcommand{\ReportTitle}{Proposal for MRes CMEE Dissertation Project}
\newcommand{\ReportAuthor}{PokMan HO (CID: 01786076)}
\newcommand{\ReportAffil}{Department of Life Sciences, Faculty of Natural Sciences,\\Imperial College London}
\newcommand{\Disclaim}{\textbf{A Proposal submitted in partial fulfilment of the requirements for the degree of Master of Research at Imperial College London\\\vspace{1cm}
		Formatted in the journal style of the \textit{Nature} Journal\\
		Submitted for the MRes in Computational Methods of Ecology and Evolution}}
\newcommand{\As}{\textit{Anabaena sp.}}
\newcommand{\Ct}{\textit{Chroococcidiopsis thermalis}}
\newcommand{\Cs}{\textit{Chroococcidiopsis sp.}}

%% tabular colouring <https://tex.stackexchange.com/questions/94799/how-do-i-color-table-columns>
\usepackage{xcolor, colortbl}
\definecolor{grey90}{rgb}{.9,.9,.9}
\definecolor{lorange}{rgb}{1,.8,0}

%% 90-deg tabular tab rotation
%\usepackage{rotating, arraycols, booktabs, array}
%\newcolumntype{R}{1}{>{\begin{turn}{90}\begin{minipage}{#1}\scriptsize}l
%			<{\end{minipage}\end{turn}}}}

\title{\ReportTitle}
\author{\ReportAuthor}
\date{}

%% citation
\usepackage[%
%autocite 	= superscript,
backend 	= bibtex,
sortcites 	= true,
style 		= authoryear %nature
]{biblatex}
\bibliography{../reference/proposal.bib}

%% set as required
\onehalfspacing %% <https://tex.stackexchange.com/questions/30073/why-is-the-linespread-factor-as-it-is>
\linenumbers

\begin{document}
	\begin{center}
		\Huge\textbf{\ReportTitle}\\
		\LARGE\ReportAuthor\\
		\Large\ReportAffil\\
		\Large{Last modified: Dec 2019}
	\end{center}
	\begin{figure}[h]
		\centering\includegraphics[width=\linewidth]{icl.jpg}
	\end{figure}
Primary supervisor:\\
\indent Dr. James Rosindell\\
\indent Department of Life Sciences (Silwood Park), Faculty of Natural Sciences, Imperial College London\\
\indent email: \href{mailto:j.rosindell@imperial.ac.uk}{j.rosindell@imperial.ac.uk}\\
Secondary supervisor:\\
\indent Dr. Samraat Pawar\\
\indent Department of Life Sciences (Silwood Park), Faculty of Natural Sciences, Imperial College London\\
\indent email: \href{mailto:s.pawar@imperial.ac.uk}{s.pawar@imperial.ac.uk}
\clearpage
\section{Keywords}
ODE model, ecological photovoltaics, electricity, carbon sequestration, co-existence, nutrient cycling
\section{Introduction}
Two main substrates were able to print cyanobacteria for electricity:  \textit{Synechocystis sp.} on carbon nanotube-coated paper \autocite{sawa2017electricity} and \As on mushrooms \autocite{joshi2018bacterial}.  Both successfully generate small photocurrents (micro-ampere on paper \autocite{sawa2017electricity} and nano-ampere on mushrooms \autocite{joshi2018bacterial} respectively).  Under the urge of clean energy due to climate change \autocite{schuur2015climate}, our civilization needs electricity as well as carbon sequestration services simultaneously and quickly.  We can augment photosynthesis by inserting carbon nanotubes inside phytocells \autocite{giraldo2014plant}, or incorporate biodiversity.  Primitive cyanobacteria \Cs\ were found extremely tough on different temperatures \autocite{baque2013boss} and \Ct\ was confirmed able to utilize extra far red spectrum as photosynthetic light source \autocite{nurnberg2018photochemistry}.  If \Cs\ lineage is able to provide photocurrent, it would be a step towards the solar-powered civilization.\\

With ecology in mind, this project aims at finding the theoretical recipe to make a multi-spectra bio-solar panel with carbon sequestration ability.  The primitive idea is mushrooms holding \As\ and \Cs\ at their umbrella-shaped pileus.  All photocells thrive until the mushroom dies, which the whole batch will be dumped into a separate Lactobacillus fermentation chamber for production of irrespirable lactates \autocite{senthuran1997lactic}.  Since new mushrooms keep replacing old ones, electricity generation continues.\\

By assuming domestic wastewater as an all-rounded fertilizer \autocite{markou2014microalgal} for this system, the research questions are:
\begin{itemize}
	\item When would the coexistence system be stabilized after initialisation?
	\item How do \As\ and \Ct\ interact?
	\item What is the expected performance on electricity generation based on this ecosystem?
	\item What is the expected performance on carbon sequestration based on this ecosystem? (if time allows)
\end{itemize}

\section{Proposed methods}
A set of ordinary differential equations (ODEs) would be implemented as a Lotka-Volterra model (LVM).  The cyanobacteria-mushroom system \autocite{joshi2018bacterial} would be the stem of this model.  Nutrients for all players were assumed to be unlimited due to the universal fertilizer of domestic wastewater \autocite{markou2014microalgal}.  The main equations included were:\\
%% eq 1,2 	: rate = growth - natural death - competition - substrate death
%% eq 3 	: rate = growth - natural death
For \As:
\begin{equation}
	dA/dt = [r_A A] - [k_{A1} A] - [k_{A2} A C] - [K_M]
\end{equation}
For \Ct:
\begin{equation}
	dC/dt = [r_C C] - [k_{C1} C] - [k_{C2} A C] - [K_M]
\end{equation}
For mushroom:
\begin{equation}
	dM/dt = [r_M M] - [k_M M]
\end{equation}

$dA/dt, dC/dt$ and $dM/dt$ were instantaneous growth rates of representative populations.  $[r_A A], [r_C C]$ and $[r_M M]$ were natural growth rates.  $[k_{A1} A], [k_{C1} C]$ and $[k_M M]$ were natural death rates.  $[k_{A2} A C]$ and $[k_{C2} A C]$ were symbolizing growth rates hinder by competition of space.  $[K_M]$ was symbolizing batch removal when mushrooms died.  By modelling how this system react to insolation cycles would provide insight on future engineering design of the true panel.\\
Natural growth and death rates for all three bio-players could be extracted in literature.  Assuming both cyanobacteria lineages only compete by growth, competition coefficient can be estimated by their differences in light harvesting ability.  This ratio would be a fraction further modified by their growth rates:
\begin{tabular}{cc}
	$k_{A2} = \dfrac{J_{Ct}}{J_{As} + J_{Ct}}\cdot r_A$ \&& $k_{C2} = \dfrac{J_{At}}{J_{As} + J_{Ct}}\cdot r_C$
\end{tabular}, which energy harvested by \As\ per cell per unit time was $J_{As}$ and that by \Ct\ was $J_{Ct}$.\\
Competition coefficients $k_{A2}$ \& $k_{C2}$ of one species was depending on how better the other one did from its (i.e. the fraction).  This coefficient was also positively depending on its own natural growth rate.  Logically higher rate, bigger hinder.\\
By assuming \As\ layer printed on mushroom was a single cell layer \autocite{joshi2018bacterial}, ratio between given light energy and photocurrent per photocell of \As\ could be calculated because of known cell size.  Photocurrent strength per cell could also be calculated for given solar wavelength.  By linking photocurrent size with population, expected electricity output per time step could be obtained.  On the other hand, carbon sequestration ability could also be estimated through biomass growth \autocite{markou2014microalgal} and Lactobacillus fermentation efficiency \autocite{senthuran1997lactic}.  This numeric description would be carried out if time allows.

\section{Anticipated outputs and outcomes}
From the simulation of successful coexistence, there would be a graph of population sizes for all ecological parties upon solving the equations.  Model will be validated if population fluctuations within the ecosystem followed (daily,) monthly, seasonal and annual insolation cycles for chosen locations.

\section{Project feasibility}
Parameter data was available in literature.  Hence completing the project within time limit is considered positive.

Key:
\begin{tabular}{lll}
	section & target & action\\\hline
	1 = abstract & m = model & s = scripting\\
	2 = introduction & a = analysis & d = debug\\
	3 = methodology & p = parameters & v = validation\\
	4 = results && \cellcolor{grey90}work\\
	5 = discussion && \cellcolor{lorange}buffer / if any\\
\end{tabular}\\
Gantt chart:
\begin{longtable}{p{.1\linewidth}|p{.1\linewidth}|p{.1\linewidth}|p{.1\linewidth}|p{.1\linewidth}|p{.1\linewidth}|}
	Month	&write					&design					&script					& \begin{tabular}{c}MSc\\lecture\end{tabular}	& \begin{tabular}{c}model\\run\end{tabular}\\\hline
	Dec		&						&\cellcolor{grey90}m,a,p	&\cellcolor{grey90}m,s	&\cellcolor{lorange}							&										\\
	Jan		&\cellcolor{grey90}3	&\cellcolor{grey90}m	&\cellcolor{grey90}m,d,v	&\cellcolor{lorange} 							&\cellcolor{grey90}						\\
	Feb		&\cellcolor{grey90}2	&\cellcolor{grey90}a	&						&\cellcolor{lorange} 							&\cellcolor{grey90}						\\
	Mar		&\cellcolor{lorange}3	&\cellcolor{lorange}	&\cellcolor{grey90}a,s	&\cellcolor{lorange} 							&\cellcolor{grey90}						\\
	Apr		&\cellcolor{grey90}4,5	&						&\cellcolor{grey90}		&												&\cellcolor{grey90}						\\
	May		&\cellcolor{grey90}		&						&\cellcolor{grey90}		&												&\cellcolor{grey90}						\\
	Jun		&\cellcolor{grey90}		&						&\cellcolor{lorange}	&												&\cellcolor{lorange}					\\
	Jul		&\cellcolor{grey90}1	&						&						&												&\cellcolor{lorange}					\\
	Aug		&\cellcolor{lorange}	&						&\cellcolor{lorange}	&												&\cellcolor{lorange}					\\
\end{longtable}
\section{Budget}
This project does not require any funds.  Round trip travelling costs to advisers for meetings (via public transport standard class) are approx GBP20.00 per trip.  Five meetings were expected throughout the project (a total of GBP100.00 is expected).
\section*{Acknowledgement}
The author would like to thank \href{mailto:wei.huang@eng.ox.ac.uk}{Prof. Wei Huang} for agreeing to be the external adviser for this project.
%\clearpage
\nocite{*}\printbibliography
\end{document}
