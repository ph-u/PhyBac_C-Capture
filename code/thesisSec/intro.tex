% Author: PokMan Ho pok.ho19@imperial.ac.uk
% Script: intro.tex
% Desc: MRes thesis introduction section
% Input: none
% Output: none
% Arguments: 0
% Date: Jan 2020

\documentclass[../thesis.tex]{subfiles} %% use packages & commands as this main file

\begin{document}
\section{Introduction}
%this is an introduction
Forests are displaced by human settlements, which means carbon absorption components are replaced by the opposite.  Urbanization, deforestation and the reliance of fossil fuels are increasing the atmospheric carbon content\autocite{ferguson2000electricity,schuur2015climate}.  A solution is to make the current heterotrophic settlements autotrophic, just like plants, using solar energy to sequester carbon and generate energy.

There are attempts to use cyanobacteria\autocite{joshi2018bacterial,mccormick2015biophotovoltaics,sawa2017electricity} and \textit{Escherichia coli}.\autocite{songera2012electricity}. They successfully generated bio-electricity yet these structures are unsustainable and short-lived.  The major reason is due to the use of single species, which the population can exist within a limited period.  With ecology in mind, these generators can potentially be having a longer service life if a small self-sustaining ecosystem can be established.  In this project, this type of electricity generators are called ``eco-bioelectric cell" (EBC).

Empirical trials can be carried out to make these self-sustaining ecosystems, but ``trial and error" approach can be resource-hungry and time-consuming.  An alternative approach is to narrow down the trial possibilities through theoretical trials.  Using mathematical models can effectively rule out trivial factors and impossible parameter combinations.  It can potentially make empirical trials more cost-effective with higher successful chances.  In this thesis, I propose the research question: How can an EBC serve carbon sequestration and electricity generation purposes sustaining for multiple years for domestic uses?

Through addressing the question, I propose the following hypotheses:
\begin{enumerate}
    \item EBC can sustain for several years
    \item Wiring EBCs can add up power output to meet household electricity demands
    \item Carbon sequestration ability of EBC is biologically-significant
\end{enumerate}

\end{document}