
% Author: PokMan Ho pok.ho19@imperial.ac.uk
% Script: proposal.tex
% Desc: `LaTex` proposal for MRes 9-months project ver 3
% Input: none
% Output: pdf proposal in same directory
% Arguments: 0
% Date: Dec 2019

\documentclass[a4paper, 11pt]{article}

%% packages
\usepackage[margin=2cm]{geometry}

\usepackage{graphicx}
\graphicspath{{graphics/}}

\usepackage{fontspec}
\setmainfont{Arial}

\usepackage{hyperref, setspace, lineno, longtable}

%% hyperlinks
\hypersetup{
	colorlinks=true,
	linkcolor=blue,
	filecolor=blue,      
	urlcolor=blue,
	citecolor=blue
}

%% test insert variables
\newcommand{\ReportTitle}{Proposal for MRes CMEE Dissertation Project}
\newcommand{\ReportAuthor}{PokMan HO (CID: 01786076)}
\newcommand{\ReportAffil}{Department of Life Sciences, Faculty of Natural Sciences,\\Imperial College London}
\newcommand{\Disclaim}{\textbf{A Proposal submitted in partial fulfilment of the requirements for the degree of Master of Research at Imperial College London\\\vspace{1cm}
		Formatted in the journal style of the \textit{Nature} Journal\\
		Submitted for the MRes in Computational Methods of Ecology and Evolution}}

%% tabular colouring <https://tex.stackexchange.com/questions/94799/how-do-i-color-table-columns>
\usepackage{xcolor, colortbl}
\definecolor{grey90}{rgb}{.9,.9,.9}
\definecolor{lorange}{rgb}{1,.8,0}

%% 90-deg tabular tab rotation
%\usepackage{rotating, arraycols, booktabs, array}
%\newcolumntype{R}{1}{>{\begin{turn}{90}\begin{minipage}{#1}\scriptsize}l
%			<{\end{minipage}\end{turn}}}}

\title{\ReportTitle}
\author{\ReportAuthor}
\date{}

%% citation
\usepackage[%
%autocite 	= superscript,
backend 	= bibtex,
sortcites 	= true,
style 		= authoryear %nature
]{biblatex}
\bibliography{../reference/00_2019ICLMResProposal.bib}

%% set as required
\onehalfspacing %% <https://tex.stackexchange.com/questions/30073/why-is-the-linespread-factor-as-it-is>
\linenumbers

\begin{document}
	\begin{center}
		\Huge\textbf{\ReportTitle}\\
		\LARGE\ReportAuthor\\
		\Large\ReportAffil\\
		\Large{Last modified: Dec 2019}
	\end{center}
	\begin{figure}[h]
		\centering\includegraphics[width=\linewidth]{icl.jpg}
	\end{figure}
Primary supervisor:\\
\indent Dr. James Rosindell\\
\indent Department of Life Sciences (Silwood Park), Faculty of Natural Sciences, Imperial College London\\
\indent email: \href{mailto:j.rosindell@imperial.ac.uk}{j.rosindell@imperial.ac.uk}\\
Secondary supervisor:\\
\indent Dr. Samraat Pawar\\
\indent Department of Life Sciences (Silwood Park), Faculty of Natural Sciences, Imperial College London\\
\indent email: \href{mailto:s.pawar@imperial.ac.uk}{s.pawar@imperial.ac.uk}
\clearpage
\section{Keywords}
ODE model, ecological photovoltaics, electricity, carbon sequestration, co-existence, nutrient cycling
\section{Introduction}
Fossil fuels are still the dominant energy source for civilization \autocite{yang2008progress,ferguson2000electricity}.  However this lifestyle is not sustainable and consequences are appearing \autocite{schuur2015climate}.  Solar energy is the ultimate unlimited source of bioenergy in shallow Earth biosphere.  Utilization of this clean energy source is the only way to take out the largest greenhouse gas emitter (i.e. the energy industry) from the system.  Only by having highly negative carbon footprint could there be a chance to restore biogas balance for the carbon cycle.\\

This project aims at searching for the minimal ingredients for the most efficient self-sustaining biovoltaics ecosystem based on an existing biovoltaic cell \autocite{mccormick2015biophotovoltaics}.  The minimal ecosystem should be able to sequester atmospheric carbon via photosynthesis and generate electricity.  The aim is to build a Ordinary Differential Equation (ODE) model to address the following questions:
\begin{itemize}
	\item What ingredients do a self-sustaining biovoltaics ecosystem need to initialize?
	\item How much ingredients do such a system need for initialization?
	\item Which organisms can coexist and coorporate to maximize electricity and carbon sequestration output by finding niches from multiple solar bands?
\end{itemize}

In order to build an ecosystem, one must know the flux of different biochemicals in different biological players and nutrient cycles of the artificial ecosystem.  The type and concentration of chemical and biological species have to be accurate to minimize probability of having unforeseen inpracticalities (e.g. the oxygen loss in ``Biosphere2 project", 1996).  This project will hence provide numerical basis for the future experimental testing of the eco-biovoltaics system.

%Humans are destabilizing our climate \autocite{schuur2015climate}.  It is easy to create atmospheric carbons but hard to reverse \autocite{yang2008progress}.  Biological electricity generator is available \autocite{mccormick2015biophotovoltaics} yet still lagging behind the global energy thirst \autocite{ferguson2000electricity}.  This project aims at modelling an ecosystem which can extract energy from multiple solar bands.\\
%Proposed research question: What is the recipe for making a multi-banded self-standing ecological solar panel?
\section{Proposed methods}
A continuous-time resource-consumer model will be scripted in R (ver 3.6.0).  It would be run in coarse and fine resolutions with different raw material combinations.  For successful attempts, non-linear-least-squares (NLLS) method would be used to optimize the parameters.  Boundary parameter values would be tested to get the successful range within the parameter space.
\section{Anticipated outputs and outcomes}
Initial parameter(s) for simulations achieving dynamic equilibria would be exported as csv and cluster graphs.  Respective biochemical and population flux would be shown in networks.

\section{Project feasibility}
Most details on the nutrient conditions for producers \autocite{duarte2009microbial,markou2014microalgal} and consumers \autocite{neff1957purification,mooshammer2014stoichiometric} were published.  Successful cyanobacteria biovoltaics panel details were also published \autocite{mccormick2015biophotovoltaics}.  Hence the project would be successful once data was included correctly in the model.\\
Key:
\begin{tabular}{lll}
	section & target & action\\\hline
	1 = abstract & m = model & s = scripting\\
	2 = introduction & a = analysis & d = debug\\
	3 = methodology & p = parameters & v = validation\\
	4 = results && \cellcolor{grey90}work\\
	5 = discussion && \cellcolor{lorange}buffer / if any\\
\end{tabular}\\
Gantt chart:
\begin{longtable}{p{.1\linewidth}|p{.1\linewidth}|p{.1\linewidth}|p{.1\linewidth}|p{.1\linewidth}|p{.1\linewidth}|}
	Month	&write					&design					&script					& \begin{tabular}{c}MSc\\lecture\end{tabular}	& \begin{tabular}{c}model\\run\end{tabular}\\\hline
	Dec		&						&\cellcolor{grey90}map	&\cellcolor{grey90}ms	&\cellcolor{lorange}							&										\\
	Jan		&\cellcolor{grey90}3	&\cellcolor{grey90}m	&\cellcolor{grey90}mdv	&\cellcolor{lorange} 							&\cellcolor{grey90}						\\
	Feb		&\cellcolor{grey90}2	&\cellcolor{grey90}a	&						&\cellcolor{lorange} 							&\cellcolor{grey90}						\\
	Mar		&\cellcolor{lorange}3	&\cellcolor{lorange}	&\cellcolor{grey90}as	&\cellcolor{lorange} 							&\cellcolor{grey90}						\\
	Apr		&\cellcolor{grey90}45	&						&\cellcolor{grey90}		&												&\cellcolor{grey90}						\\
	May		&\cellcolor{grey90}		&						&\cellcolor{grey90}		&												&\cellcolor{grey90}						\\
	Jun		&\cellcolor{grey90}		&						&\cellcolor{lorange}	&												&\cellcolor{lorange}					\\
	Jul		&\cellcolor{grey90}1	&						&						&												&\cellcolor{lorange}					\\
	Aug		&\cellcolor{lorange}	&						&\cellcolor{lorange}	&												&\cellcolor{lorange}					\\
\end{longtable}
\section{Budget}
This project is free
\section*{Acknowledgement}
The author would like to thank \href{mailto:wei.huang@eng.ox.ac.uk}{Prof. Wei Huang} for agreeing to be the external adviser for this project.
\clearpage
\nocite{*}\printbibliography
\end{document}
