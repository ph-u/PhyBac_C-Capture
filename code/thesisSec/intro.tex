% Author: PokMan Ho pok.ho19@imperial.ac.uk
% Script: intro.tex
% Desc: MRes thesis introduction section
% Input: none
% Output: none
% Arguments: 0
% Date: Jan 2020

\documentclass[../thesis.tex]{subfiles} %% use packages & commands as this main file

\begin{document}
\section{Introduction}
%this is an introduction

%% 20200324
Solar energy is a type of renewables that has the potential of powering human settlements due to the high flux to Earth surfaces daily.  Solar panels are attempting to make use of this energy but production of these chemical-based panels can produce different types of pollution.  The production process also involves a non-trivial carbon footprint that may not be compensated throughout its service life.  Hence the use of these solar panels might not be environmentally-friendly.

The loss of green surfaces in planetary scale is observable.  The use of fossil fuels and decreasing photosynthetic cover are undeniable facts to the contribution of anthropogenic carbon dioxide production.  Based on the above reality, can we construct a photosynthetic solar panel with small carbon footprint during panel production?

Phytoplanktons are important autotrophic players contribute to around 50\% carbon fixation service.  Some microbial detritivores, such as \textit{Escherichia coli}, are found to leak electricity naturally along with their metabolism machinary.  Hence logically if one use phytoplankton biomass to feed these detritivores in a stable ecosystem, this ecosystem can sequester carbon dioxide and generate electricity.  If the system can generate biomass faster than consumption, this ecosystem is a net carbon sink with energy storage (in the form of organic matter) and electricity production ability.  In that case, this system is a solar-rechargeable battery.

Due to the huge microbial biodiversity available, an empirical pairwise study of phytoplanktons and detritivores is unfeasible.  So a theoretical study to trim down the number of possible candidates makes this idea tangible.  Through a minimal mathematical model on a phytoplankton-detritivore couple, stability of such system can be examined and parameter ranges can be searched.

%% 20200327
Published microbial ecological models were found either statistical or complex.  Statistical models can provide predictions on multiple factors in different scales while complex models can explore parameter spaces more thoroughly.  Yet a simple interactive model providing both functions were lacked.  Hence this project was designed as the first step to tackle the multi-layered battery problem by designing a simple model describing biotic and abiotic carbon pool fluxes.  Since biotic carbon pools are related to carbon sequestration, carbon use efficiency and possible electricity output, an accurate mathematical model exploring the biotic parameter spaces can narrow down candidates for downstream empirical trials.

With the above motivation in mind, I proposed the following research question: how can a minimal artificial ecosystem with phytoplankton and detritivore co-exist and accumulate carbon in the system?

Several hypotheses can also be proposed as follow:
\begin{itemize}
    \item There is a non-trivial carbon balance in the artificial ecosystem;
    \item Intraspecific interference in phytoplankton can be the only hindrance factor leading to ecosystem stability; and
    \item There are possible microbial candidates meeting the parameter ranges achieving system stability.
\end{itemize}

%% 20202328
A simple ecosystem interactive model on carbon density is beneficial for evaluating and assessing different aspects of an ecosystem.  It can provide an approximate framework even when only overview ecosystem (such as a lake) parameters or partial ecosystem (such as a forest) data are available or collected.  Using a simple autotroph-heterotroph mechanistic model can also provide guidance on significant and important factors when given few but accurate information.  This model is potentially important not only within the field of quantitative ecology across biomes but also inter-relating fields such as ecosystem engineering and environmental impact assessments.

\end{document}