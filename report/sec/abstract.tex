% 20200813
\documentclass[../thesis.tex]{subfiles} %% use packages & commands as this main file
\begin{document}

Climate is changing due to the rocketing atmospheric carbon content and we need an effective solution to capture and store these carbon.  Many carbon capture solutions were proposed but we lack a consensus on which proposals truly work.  One of these plans is to use a pure \phy\ culture to capture and store atmospheric carbon.  However these \phy\ cultures cannot tolerate the presence of \bac\ or else \bac\ would destroy the yield.  To see whether \bac\ is truly forbidden, I have simulated the co-culture with a new mathematical model.  The model simulates the carbon flow in a \pbs\ with one \phy\ and one \bac\ under two different yield harvesting methods -- destructive and continuous.  Destructive harvest is collecting all carbon content in the system at once; continuous harvest is filtering out some organic carbon from the system daily.  The result shows \Rn{1}) destructive harvest is superior when \bac l life history traits are uncontrollable; \Rn{2}) continuous harvest makes most \pbs s unfeasible but the feasible ones have an expected yield higher than pure \phy\ systems; \Rn{3}) high yield is possible when an optimal \bac\ is co-cultured with \phy.  This proof-of-concept study shows that \bac\ can be beneficial to \phy\ cultures when the right \bac\ is used, although most \bac\ would destroy the yield.  Further experiments are needed on which \bac\ is the optimal candidate, how to construct such co-culture and how high is the potential benefit.  If \bac\ is experimentally proven beneficial to capture atmospheric carbon, these optimal co-cultures can be proposed as an effective carbon capture solution to combat the current human-induced climate change.

\end{document}