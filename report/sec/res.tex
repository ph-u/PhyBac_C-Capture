% 20200816
\documentclass[../thesis.tex]{subfiles} %% use packages & commands as this main file
\begin{document}

Biomass densities were stable around 10 days in destructive systems (Fig.\ref{f:destCarbon}).  \Phy\ biomass dropped during system developments in \PoN\ \& \PBN\  (Fig.\ref{f:destCarbon}B).  \Bac\ density increased beyond and dropped back to the equilibrium density (Fig.\ref{f:destCarbon}C).  Organic carbon in both \PoN\ \& \PBN\ behaved differently (Fig.\ref{f:destCarbon}A).  It accumulated linearly in \PoN\ but dropped to equilibrium density in \PBH.

Average carbon yield from feasible scenarios of \PoN\ was significantly higher than that of \PBN\ (Fig.\ref{f:ydByHarv}A; pairwise Wilcox p$\ll$0.01).  Daily carbon yield from feasible scenarios of \PoH\ was significantly lower than that of \PBH\ (Fig.\ref{f:ydByHarv}B; pairwise Wilcox p$\ll$0.01).  Average carbon yield for \PoN\ had a significant increase (Fig.\ref{f:ydByHarv}A\&\ref{f:ydDaily}A) between day 0.5 and 5 (pairwise Wilcox p$\ll$0.01) but not between day 5 and 50 (pairwise Wilcox p$>$0.1).  Daily carbon yield for \PoH\ had no significant different across harvest rates (Fig.\ref{f:ydByHarv}B; pairwise Wilcox p$>$0.1) and no significant different from average carbon yield of equivalent harvest intervals in \PoN\ (Fig.\ref{f:ydDaily}\&\ref{f:harvPo}).  \PBN\ average carbon yield had a significant drop on median yield from the selected intervals (Fig.\ref{f:ydByHarv}A\&\ref{f:ydDaily}A).  Daily carbon yield for \PBH\ had a significant increase across selected harvest rates (Fig.\ref{f:ydByHarv}B\&\ref{f:ydDaily}B; pairwise Wilcox p$\ll$0.01); number of feasible scenarios dropped with increasing harvest rate (Fig.\ref{f:ydByHarv}B; $x$=100 \dayU, n=184; $x$=1000 \dayU, n=31; $x$=10K \dayU, n=3).

Harvest interval/rate containing the maximum carbon yield were different (\PBN\ $<$ \PBH\ $\ll$ \PoH\ $<$ \PoN).  They had different number of feasible scenarios (\PoH=\PoN\ $>$ \PBN\ $\gg$ \PBH) and average/daily carbon yield (\PBH\ $\gg$ \PoH/\PoN\ $>$ \PBN).  Maximum carbon yield of \phy-only systems were almost the same, higher than that of \PBH\ for around 50 \dxdt\ and \PBN\ for 100 \dxdt\ (\PoH=\PoN\ $>$ \PBH\ $>$ \PBN).  Interquartile range of average/daily carbon yield was also different (\PBH\ $>$ \PoH\ $>$ \PoN\ $\gg$ \PBN).  There were 61.6\% (n=3388) scenarios in \PBN\ with average carbon yield $>$ 0 \dxdt.  It showed that with destructive harvest mode on a \pbs, the system would have a 0.616 chance of having a beneficial yield return after 90 days if the \phy\ and \bac\ life history trait values were unknown.  There were 0.3\% (n=19) scenarios in \PBH\ with daily carbon yield $>$ 0 \dxdt.  It showed that finding a \pbs\ with continuous harvest mode having a feasible carbon yield was very hard if the life history trait values were unknown.  Yet each feasible scenarios of \PBH\ yielded much higher than other alternatives (without \bac\ and/or using destructive harvest mode).  Details of average/daily carbon yield were summarized in Table \ref{t:feasDist}.  In short, \phy-only systems had similar carbon yield regardless of the harvest mode; carbon yield for \pbs s depend heavily on the choice of organisms, harvest mode and harvest interval/rate.

\begin{table}[H]
    \centering
    \caption[Feasible carbon yield distribution summary]{Feasible carbon yield distribution summary of the harvest interval/rate with maximum yield}
    \begin{tabular}{ccrrcccc}\hline
        System & max & scenarios & \% & lower quartile & median & upper quartile & max \\
        & $x$ or $T$ & (N=5500) && \dxdt & \dxdt & \dxdt & \dxdt \\\hline
        \PBH & 2101 \dayU & 19 & 0.3 & 12.4 & 25.2 & 47.8 & 285 \\
        \PBN & 90 days & 5346 & 97.2 & -0.012 & 0.019 & 0.141 & 243 \\
        \PoH & 19501 \dayU & 5500 & 100.0 & 0.039 & 0.371 & 1.853 & 346 \\
        \PoN & 19900 days & 5500 & 100.0 & 0.045 & 0.379 & 1.851 & 346 \\
    \hline\end{tabular}
    \label{t:feasDist}
\end{table}

In destructive harvest systems, \phy\ life history traits values significantly determined the effectiveness of the system (Fig.\ref{f:bacEffect}; Wilcox p$\ll$0.01 between the two parameter extremes for each \phy\ parameter in both \PoN\ \& \PBN).  Effect of \phy\ life history traits were similar between \PoN\ and \PBN: increase of \phy\ non-respired carbon fraction ($\ePR$), \phy\ assimilated carbon fraction ($\eP$) or \phy\ growth rate led to a significant increase of average carbon yield; increase of \phy\ intraspecific interference led to a significant decrease of average carbon yield.  \Bac\ life history traits values had no significant influence (Fig.\ref{f:bacEffect2}; Wilcox p$>$0.1 between the two parameter extremes for each \bac\ parameter in both \PoN\ \& \PBN).  In short, yield distribution maximized when \phy\ had high carbon-to-biomass ratio (high $\ePR$ and $\eP$), high growth rate (high $\gP$) and low intraspecific interference (low $\aP$).

Life history traits for \phy\ and \bac\ determined both feasibility and effectiveness in \PBH\ systems.  Although influence trends for \phy\ life history traits had a similar effect on the effectiveness, feasible scenarios for \PBH\ observed when $\ePR$ was above average (Fig.\ref{f:harvPB}A), $\eP$ was not low (Fig.\ref{f:harvPB}B), $\gP$ was not low (Fig.\ref{f:harvPB}C), $\aP$ was low (Fig.\ref{f:harvPB}D).  For \bac\ life history traits, feasible scenarios for \PBH observed when \bac\ non-respired carbon fraction ($\eBR$) was in a mid-upper range (Fig.\ref{f:harvPB}E), \bac\ assimilated carbon fraction ($\eB$) was not around 40\% (Fig.\ref{f:harvPB}F), \bac\ clearance rate was either below average or high (Fig.\ref{f:harvPB}G) and with specific \bac\ death rates (Fig.\ref{f:harvPB}H).  \Bac\ life history traits had significant influence on \PBH\ daily yield (Wilcox p$\ll$0.01 between the two parameter extremes within feasibility limit for each \bac\ parameter).  $\eBR$, $\eB$ and $\gB$ all led to a decreasing daily carbon yield while $\mB$ led to an increasing daily carbon yield.  In short, a \PBH\ system had similar \phy\ requirements with other systems; in addition, yield distribution maximized with \bac\ had moderately-high carbon leakage (moderate $\eBR$ and low $\eB$), above average growth rate (moderate-high $\gB$) and a specific low death rate (discrete low $\mB$ values).

\end{document}