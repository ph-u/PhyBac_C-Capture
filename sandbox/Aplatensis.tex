% Author        : PokMan Ho pok.ho19@imperial.ac.uk
% Script        : Aplatensis.tex
% Desc          : reasons for including A. platensis as food source
% Input         : none
% Output        : pdf report in same directory
% Arguments     : 0
% Date          : Feb 2020

\documentclass[a4paper,11pt]{article}
\usepackage[margin=2cm]{geometry}
\usepackage[english]{babel}
\usepackage{graphicx, longtable, amsmath, amssymb, csquotes}
\graphicspath{{graph/}{graph/}}

\usepackage{xcolor,colortbl}
\definecolor{green}{rgb}{0,.4,0}

\usepackage{hyperref}
\hypersetup{
	colorlinks=true,
	linkcolor=green,
	filecolor=red,      
	urlcolor=blue,
	citecolor=orange
}

%% citation
\usepackage[%
autocite 	= superscript,
backend 	= bibtex,
sortcites 	= true,
style 		= nature
]{biblatex}
\bibliography{Aplatensis.bib}

\title{\textit{A. platensis} in Eco-Bioelectric Cell}
\author{PokMan Ho}
\date{09 Feb 2020}

\begin{document}
    \maketitle
    \tableofcontents
    \clearpage
    
    \section{lineage description}
    \href{https://microbewiki.kenyon.edu/index.php/Arthrospira_platensis}{\textit{Arthrospira platensis}} (\href{https://microbewiki.kenyon.edu/index.php/Spirulina}{Spirulina})
    
    \section{assumptions}
    \begin{itemize}
        \item 
    \end{itemize}
    
    \section{electricigen potential}
    
    \section{digestion efficiency to metabolic energy}
    
    \section{conclusion}
    
    \nocite{*}\printbibliography
\end{document}