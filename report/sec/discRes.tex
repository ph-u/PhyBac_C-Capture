% 20200816
\documentclass[../thesis.tex]{subfiles} %% use packages & commands as this main file
\begin{document}

Engineering with bio-components have less potential to harm nature than geo-engineering proposals.  This is because the former can be installed in cities and on rooftops; the device is isolated from natural communities.  Geo-engineering proposals however directly perturbing nature; there are risks of unintended consequences.  This study proposes a new model incorporating artificial and life history trait constraints with applications in designing an effective carbon capture device.  The model incorporates \phy\ in a system with the option to add \bac\ and harvest.  The result show the importance of optimising \phy, \bac\ and harvest parameters for maximum carbon harvest efficiency.  The result also show there is a trade-off between feasibility and efficiency; \phy\ cultures have high feasibility but low efficiency while a \pbs\ have low feasibility but high efficiency when optimised.  The choice of system is dependent on the goals of the users and whether the life history traits of \phy\ and \bac\ are controllable.

\subsection{Feasibility of \pbs s}
My model shows \bac\ lowers feasibility of \pbs s (0.3\% \PBH, 97.2\% \PBN) compare with \phy-only systems (100\% \PoH\ \& \PoN; Table \ref{t:feasDist}).  This means a random choice of \phy-\bac\ pair has a lower chance of system establishment.  Substantial work has suggested the importance of chemical mediators in establishment of a high carbon yield co-culture (also known as consortia) and about culture medium optimisations \autocite{rivas2010interactions,amin2009photolysis,fuentes2016impact}.  This study suggests an alternative view on optimising bio-mediators (life history traits).  The results show life history trait values are at least as important as (if not more) environmental factors leading to an establishment of a high carbon yield co-culture; bio-mediators have not been studied in the context of carbon capture yet.  The small feasibility (0.3\%) may explain the few success cases in the experimental tests of high carbon yield \phy-\bac\ pairs \autocite{fuentes2016impact}.  Unfortunately those experiments do not measure life history traits of their organisms of choice; the measurements count on future work for confirmation on whether the experimental life history trait values are those predicted by this model (Fig. \ref{f:harvPB}).

\subsection{Optimal harvest of \phy\ and \phy-\bac\ \pbs s}
Previous work has mixed views on whether destructive or continuous harvest is a better carbon capture approach.  Destructive harvest is proposed by many aquaculture and geo-engineering projects \autocite{lawrence2014efficiency,krause2016substantial}.  Co-culture experiments often adopt continuous harvest methods without considering destructive harvest as an alternative \autocite{kim2008anaerobic,kazamia2012mutualistic}.  My model incorporates both continuous and destructive harvest options which enables direct comparison of the two harvest methods.

The results show there is no optimal harvest for \phy-only systems (Fig. \ref{f:ydDaily}).  This is because \phy\ captures carbon at a constant speed regardless of the carbon density in the model after the system stabilizes (Fig. \ref{f:destCarbon}).  The density-independent carbon capture feature from \phy\ causes a linear accumulation of organic carbon independent from time (Fig. \ref{f:ydDaily}A).  Hence carbon yield for \phy-only systems at equivalent harvest interval/rate have almost the same numerical value; any numerical difference is likely explained by the technical error of the computational methods I used.  The results agree with geo-engineering proposals that adopting destructive harvest by convention is an effective carbon capture method because the alternative harvest method does not have significant carbon yield advantages (Fig. \ref{f:harvPo}).

Continuous harvest is the optimal harvest for \pbs s within feasibility limits (Fig. \ref{f:harvPB}).  This is because \bac\ stabilizes equilibrium carbon densities during development of \pbs s (Fig. \ref{f:destCarbon}A).  As a result longer harvest intervals in destructive harvest have no effect on the carbon content accumulation in the systems; longer waiting time lowers the average carbon yield instead (Fig. \ref{f:ydDaily}A).  Continuous harvest consistently extracts organic carbon out from the system without altering biomass content.  Thus the carbon capture ability from \phy\ is not affected by the harvest.  On the other hand, \bac l growth is limited by the availability of organic carbon; future carbon consumption from \bac\ is also limited.  With smaller \bac l biomass, more carbon can be retained as organic carbon and benefit future harvest of carbon yield.  So a high harvest rate benefits carbon yield from \pbs s (Fig. \ref{f:ydDaily}B).  The results agree with experimental observations of continuous harvest enhances carbon/product yield from \pbs s \autocite{kim2008anaerobic,choix2012enhanced1,choix2012enhanced2,leyva2014accumulation} by providing a mechanistic explanation using ecological interactions between \phy\ and \bac.


The result shows no optimal harvest for \phy-only systems (Fig. \ref{f:ydDaily}).  The reason is because \phy\ is a carbon pump in the model (Fig. \ref{f:model}).  The rate of carbon capture by \phy\ is depending on rate of photosynthesis, respiration, leakage, biomass growth and intraspecific interference.  These rates in the model are either fixed values (resource allocation, biomass growth and interspecific interference) or a combination of fixed parameters (photosynthesis, respiration and leakage rates).  Equilibrium positions of \phy\ biomass density should be similar between destructive and continuous harvest modes (Fig. \ref{f:ydByHarv} \& \ref{f:harvPo}).  The result cannot falsify continuous harvest is an effective method for carbon harvest \autocite{fuentes2016impact}, although destructive harvest is proposed by many aquaculture and geo-engineering projects \autocite{lawrence2014efficiency,krause2016substantial}.

Continuous harvest mode is necessary for \phy-\bac\ \pbs s.  The result shows continuous harvest is significantly higher than destructive harvest among the feasible scenarios across all harvest intervals/rates (Fig. \ref{f:ydDaily}).  The effect of high product yield is also observed from the production of hydrogen \autocite{kim2008anaerobic}, carbohydrates \autocite{choix2012enhanced1,choix2012enhanced2} and fatty acids \autocite{leyva2014accumulation}.  The reason for the lack of consensus support is because of the low feasibility of \pbs s (Table \ref{t:feasDist}).  In publications reporting high yields, chemical analysis of the \pbs s is a popular methodology \autocite{santos2014microalgal,rivas2010interactions,leyva2014accumulation,amin2009photolysis}.  Without indications on life history traits of the organisms they have selected, it is hard to construct an effective carbon capture engine with living components.  Nevertheless, this study was able to provide strong direct and quantitative evidence that continuous harvest is more superior than the destructive under a set of defined ranges of \phy\ and \bac\ life history trait values.

\subsection{Importance of life history traits for the organic carbon yield}
Life history traits of \phy\ are important for \phy-only systems. An optimised \phy\ on life history trait values produces a significant higher average/daily organic carbon yield (Fig. \ref{f:bacEffect}).  In the model, a \phy-only system contains only the \phy\ half of Fig. \ref{f:model}.  Destructive carbon harvest both organic carbon ($C$) and \phy\ biomass ($P$) pool after days of accumulation while continuous harvest daily extract only the $C$ pool.  However content of $P$ is non-significant when compared to the $C$ pool (Fig. \ref{f:ydByHarv}).  Yield maximisation should focus on expanding the $C$ pool, which can be achieved through \Rn{1}) increase the rate of photosynthesis and \Rn{2}) reduce the rate of respiration.  Rate of photosynthesis can be increased by having more chlorophyll content for carbon capture.  This is achieved by applying fertilisers in geo-engineering proposals \autocite{gnanadesikan2008export,lawrence2014efficiency,trick2010iron,kwiatkowski2015atmospheric,lovelock2007ocean} or using a \phy\ with higher carbon-to-biomass ratio (Fig. \ref{f:bacEffect}B) or higher growth rate (Fig. \ref{f:bacEffect}C).  Rate of respiration can be reduced by lowering the \phy\ respiration carbon fraction (Fig. \ref{f:bacEffect}A), which is a factor currently lack of studies.  Low intraspecific interference ($\aP$) yield higher in general (Fig. \ref{f:bacEffect}D).  The result agrees with Fig. \ref{f:ydByHarv}, in which \phy\ biomass are insignificant between \PoH\ and \PoN\ systems (Wilcox p$>$0.1).  This shows \phy\ biomass is functionally more important than its literal carbon content.  The reason is because carbon in the form of biomass enhances carbon capture from the atmosphere.  Because the ratio of carbon allocations are fixed, more carbon can be fixed to the $C$ pool and allocated to \phy\ biomass.  The extra biomass due to low $\aP$ brings a positive feedback in carbon yield.

%[  Intraspecific interference ($\aP$) is beneficial in smaller values (Fig. \ref{f:bacEffect}D).  In destructive system a low $\aP$ has pure benefit because carbon in the system are used to absorb more carbon into the system.  However in continuous harvest systems a high $\aP$ brings direct benefit to the $C$ pool (as \phy\ biomass contribute to the $C$ pool upon death) while a low $\aP$ only bring indirect benefit by contributing to \phy\ biomass.  More \phy\ can absorb more carbon into the system for either respiration, leakage (carbon directly into the $C$ pool) or into \phy\ biomass (further increasing photosynthesis rate).  The result shows that \phy\ biomass is functionally more important than its literal carbon content (Fig. \ref{f:ydByHarv}).

Life history traits of \phy\ and \bac\ are vital for feasible \pbs s.  Destructive harvest mode brings most carbon yield down to near zero (Table \ref{t:feasDist}) and has no functional importance in this study.  Continuous harvest mode can have high carbon yield with suitable \phy\ and \bac\ combinations (Fig. \ref{f:harvPB}), and has the highest expected yield among all systems in this study.  \PBH\ systems are Fig. \ref{f:model}, which the $C$ pool is supplied with leakages and dead cells from both \phy\ and \bac.  The result shows the best \phy-\bac\ \pbs\ maximises \phy\ and minimises \bac l carbon use efficiencies.  Model parameters can then be regrouped into two functional categories: consumption and production.

``Consumption" parameters reduce the \pbs's future carbon capture ability which should be minimised.  $\aP$ is the short-term increasing $C$ pool at the expense of future carbon capture (Fig. \ref{f:harvPB}D).  $\eBR$ and $\eB$ are the \bac l carbon consumption parameters which directly reduce the carbon density in $C$ pool (Fig. \ref{f:harvPB}E-F).  ``Production" parameters enhance the \pbs's future carbon capture ability which should be maximised.  $\mB$ is the short-term increase of $C$ pool and hinder the future \bac\ consumption on organic carbon; yet only a low death rate can support a sustainable population of \bac\ in a \pbs. $\ePR$ and $\eP$ are the \phy\ carbon capture parameters which directly enhances the carbon content in the \pbs s.  $\gP$ and $\gB$ are two special parameters which conditionally belong to ``consumption" or ``production".  Values lower than the optimal have positive effects (the ``production" group) while those higher than the optimal have negative effects (the ``consumption" group).  Reasons for $\gP$ having positive and $\gB$ having negative effects are logical because of their functional roles -- \phy\ is the producer while \bac\ is the consumer.  $\gP$ higher than optimal may analogous to a \phy\ bloom.  A bloom in the model system brings more carbon into the system but space is limited.  The extra \phy\ cannot contribute to the $C$ pool and cannot be harvested.  This situation may provide a reason for a lower-than-expected yield in ocean fertilisation geo-engineering proposals \autocite{boyd2008implications,gnanadesikan2008export,oschlies2010side} because the space factor limits the maximum capacity of \phy.  $\gB$ lower than optimal may suggest a reason for the importance of \bac\ to enhance carbon yield in \pbs s \autocite{fuentes2016impact,santos2014microalgal}.

\end{document}