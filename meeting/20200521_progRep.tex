% Author        : PokMan Ho
% Script        : 20200521_progRep.tex
% Desc          : MRes progress report tex
% Input         : none
% Output        : pdf report in same directory
% Arguments     : 0
% Date          : May 2020

\documentclass[a4paper,11pt]{article}
\usepackage[margin=2cm]{geometry}
\usepackage[utf8]{inputenc}
\usepackage[english]{babel}
\usepackage{graphicx, hyperref, longtable, amsmath, amssymb}
\graphicspath{{figure/}{figure/}}

\hypersetup{
	colorlinks=true,
	linkcolor=blue,
	filecolor=blue,      
	urlcolor=blue,
	citecolor=blue
}

\title{Thursday mornings MRes project progress report}
\author{PokMan Ho}
\date{}

\begin{document}
    \maketitle
    
    \begin{tabular}{rl}
        Date Time: & 21-May-2020 09:00 \\
        Location: & MS Teams \\
        People: & James, Samraat, Emma, PokMan \\
    \end{tabular}
    
    \section{Topics discussed}
    \begin{enumerate}
        \item R vs Julia-py3 using same equation settings
        \item R initial carbon density scanning
    \end{enumerate}
    
    \section{Consensus}
    \begin{enumerate}
        \item Do the same scan using Julia-py3 to confirm numerical stability (pie charts)
        \item if scipy.integrate.odeint not having numerical stability, try other solvers (default: LSODA; alt: LSODE, LSODES)
        \item do a stability test in Julia-py3 numerical solution using analytical solution as initial values
        \item (James) do very small initial values practically meaningful?
        \begin{itemize}
            \item ecosystem stochasticity may take over
            \item microbial cells are easy to obtain
        \end{itemize}
    \end{enumerate}
\end{document}
