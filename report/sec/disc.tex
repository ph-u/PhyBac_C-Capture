% Author: PokMan Ho
% Script: disc.tex
% Desc: MRes thesis discussion section
% Input: none
% Output: none
% Arguments: 0
% Date: Jan 2020

\documentclass[../thesis.tex]{subfiles} %% use packages & commands as this main file

\begin{document}
\section{Discussion}
%this is the discussion
\subsection{Overview}
All three hypotheses proposed are conditional.  Hypothesis 1 is true for harvest rate $>$ 1/day (Fig.\ref{fig:wilcox}).  Hypothesis 2 is accurate for harvest rate $>$ 8/day (Fig.\ref{fig:wilcox}).  Hypothesis 3 correctly describes all parameters except $\eP$ (Fig.\ref{fig:v2}B).  Bacterial parameters, however, only affect P+B systems as deduced (eqm 3 in Table \ref{tab:eqm}, Fig.\ref{fig:v2}E-H).

The result implies an ideal microbial carbon capture (MCC) system has a phytoplankton strain with high carbon capture ability (high $\ePR$) that focus on cell growth instead of division (moderately low $\gP$) and low density-dependent death rate (low $\aP$).  It does not matter much whether the captured carbon is leaked or in biomass (stable $\eP$) under the fourth model assumption ``no carbon preference for bacteria".  The ideal MCC system should also put in a bacterial strain with moderately high carbon retain ability (moderately high $\eBR$) but high leakage (low $\eB$).  Slow growers (low $\gB$) with high density-independent death rate (high $\mB$) strains is preferred.  This statement, however, is potentially involving a trade-off between strains with high longevity (i.e. slow growers) and high turnover rate (i.e. high death rate).  Benefits brought to the P+B system by the bacteria are more subtle than that of phytoplankton because the effects are observable only for the ``log yield flux" (Fig.\ref{fig:v2}).  This means these P+B systems look similar although their functions have difference in log-scale manner.  Two interesting points to note is phytoplankton strains that grow too slow or bacteria being too efficient in carbon retaining will lead to a crash on such system (Fig.\ref{fig:v2}).  System crashing means bacteria is eliminated from the system at steady state.

This study shows that a P-only system is for the people prefer feasibility while a P+B system is for the ones prefer efficiency.  The reason is because for every biological parameter combinations P-only systems have 100\% feasibility (Fig.\ref{fig:wilcox}).  Yet only a fraction of them are feasible for P+B systems.  Yet these possible combination yielded generally higher than that of its P-only version (Fig.\ref{fig:wilcox}).

\subsection{Strengths and Constrains}
ODE developed for this study is reflecting a general reality.  The model is general because it is spatial, temporal and temperature independent.  A minimal number of terms are used to describe a simple coupling system for one phytoplankton and one bacteria.  This system is ``real" because each parameter carry realistic biological meaning (Table \ref{varInTab}) which is measurable through experiments.  The ODE only carries one mechanistic component, the intraspecific interference ($\aP$).  This allows users to generate testable hypotheses on phytoplankton-bacteria ecological relationships, for both intraspecific and interspecific aspects.  The system is also mathematically simple, enhancing readability and adaptability into more complex systems.  Complicated systems with multiple trophic levels can either use an enhanced version of this model or a set of generalised parameters to represent real system's autotrophic and resource cycling ability.

Since this study is a pilot theoretical study, blanketed assumptions (described in the first methodology subsection) are needed to initiate a testable model.  For example, many ecosystems have light and/or nutrient limitation(s).  Light attenuation and nitrogen availability in aquatic ecosystems are also common parameters in location-specific models which shapes temporal primary productivity and population distributions in water columns (Table \ref{modComp}).  In this study, we have assumed an unlimited resource supply except space.  This ideal state can be constructed through a continuous supply of nutrients to the system with a shallow water column to ensure light and nutrients can reach every part of the system.  On the other hand, these blanketed assumptions are testable null hypotheses.  With more observations on these nulls, they can either be falsified, restricted or kept.  Model variations can also be scripted based on different sets of environmental conditions.

Literature also reveal that some phytoplanktons can reabsorb dissolved organic carbon from the environment.\autocite{j1989respiration,samejima1958heterotrophic}  Yet in this model we assume carbon only pass through phytoplanktons in one-way direction (hence no reabsorption).  The purpose is to keep the model a minimalist fashion preventing overparameterisation.  The minimal model hence detaches phytoplankton's dependency on available carbon pool (Table \ref{tab:eqm}).  Hence the role of phytoplankton in the model is a biological carbon pump to supply organic carbon to the C pool (Fig.\ref{modelInWord}).

Parameter ranges in this study are not restricted by the described species.  It is beneficial in exploring biological possibilities yet fall short on realism.  As a result, there are no lists on the possible microbial candidates for the P+B system.  The reason is due to the lack of biological data.  Most of the parameters (except phytoplankton growth rate $\gP$ and bacterial clearance rate $\gB$) can only be extracted from papers published decades ago.  Details such as environmental pH, oxygen availability and apparatus preparation hence cannot be standardised.  Bacterial clearance rate was modified from growth rates assuming these two quantities have an arbitrary linear relationship and growth rates of microbes (both phytoplankton and bacteria) have similar ranges.  For the data collected, there is no species with information for all four of its parameters (either for the set [$\ePR$, $\eP$, $\gP$, $\aP$] or [$\eBR$, $\eB$, $\gB$, $\mB$]).  Some bacterial data was also recorded as ``bacterial community" in the papers,\autocite{cochran1988estimation} making analyses only achievable in a community level.

\subsection{Applications and Future Directions}
%% theoretical and applicable
The first line of application is constructing carbon capture devices.  Because of the general reality nature, users can measure the parameters from existed microbes comply with its role in the system and have a quick glance on the feasibility of the proposed system.  Under the increasing risk of climate destabilisation,\autocite{notz2016observed,schuur2015climate} efficiency in both time and cost is generally preferred.

Further modification can also be incorporated into microbial carbon-capture cells (MCC),\autocite{neethu2018enhancement,pandit2012microbial}

This flexible model can also be modified or generalised into different scenarios involving autotrophs and chemoorganotrophs.  For example, environmental impacts can be modeled by generalising the ecosystem-of-interest into overviewing parameters.  Consumers can be mathematically hidden into the death rates of producers.  Alternatively, a consumer equation can be explicitly plug into the model to enhance the compatibility to the site.  By integrating the ODE system, the ecological impacts can be better quantified in the environmental impact assessment (EIA).

\end{document}
