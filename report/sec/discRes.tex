% 20200713, 20200720
\documentclass[../thesis.tex]{subfiles} %% use packages & commands as this main file
\begin{document}
Our model quantitatively shows \bac\ mainly determines whether a carbon harvest system will work and \phy\ mainly determines how efficient is the system (Fig.\ref{f:bacEffect}).  Linear accumulation of organic carbon achieved in a few days when system stabilised (Fig.\ref{f:ydDaily}A).  \Bac\ in destructive systems stabilise the system carbon content (Fig.\ref{f:ydDaily}A).  Harvesting mainly determines whether \pbs s can be successfully established (Fig.\ref{f:ydByHarv}).  Destructive harvest is better when microbial biology cannot be controlled (Fig.\ref{f:ydDaily}A).  Yet continuous harvest is better when microbial biology can be designed (Fig.\ref{f:harvPB}); a \PBH\ system with maximised yield yields better than the optimised \PBN\ system.

Parameter ranges are guided by published data (Table \ref{t:ranges}) cited in the appendix.  Parameters definitions are key biological processes in microbes.  Hence yield distributions reflect simplistic expectations of \phy-only and \pbs s under destructive and continuous harvest.

%% biofuel summary
Destructive harvest benefits systems with uncontrollable biology.  Time is allowed for accumulation of products with quick rotation.  The method is currently used in biofuel \autocite{robertson2017cellulosic} and \bac l cellulose \autocite{aytekin2016statistical} cultivation.  Fast-growing species minimise the waiting time (Fig.\ref{f:ydDaily}A) because rotation cycles can be completed quicker \autocite{robertson2017cellulosic}.  Maximum harvesting cycle is around 3 days for \phy-only systems and 20 days for \pbs s (Fig.\ref{f:ydDaily}A).

%% recombinant bacteria summary
Continuous harvest benefits systems with controllable biology (e.g. genetic modified organisms).  Nutrients and bio-products can be flowing into and out from the \bac l culture continuously, keeping the population density (hence productivity) at its maximum capacity \autocite{huang2012industrial,matassa2016autotrophic}.  The method can minimise cost of recombinant technology by maximising yield per \bac.  Mixed microbial communities can produce high quality microbial proteins from carbon dioxide and ammonia using domestic wastewater as nutrient source \autocite{matassa2016autotrophic}.  Yield can be hugely enhanced to more than 200\dxdt\ of organic carbon by optimising the harvest rate and \phy\ and \bac l biology (Fig.\ref{f:ydByHarv}B \& \ref{f:harvPB}).

%% ecosystem stability maintained by perturbations
Continuous harvest gives resilience to \phy\ cultures against \bac l invasion.  The invasion-resistance \phy\ culture resembles a frequently-perturbed ecosystem, such as savannas and grasslands, which stability maintains by the fire cycle \autocite{dass2018grasslands,trollope1984fire}.  Fire in these ecosystems preserves fast-growing communities \autocite{trollope1984fire} from succession to more mature ecosystems.  Continuous harvest in a \phy-only culture has a similar effect \autocite{sharp2017robust}, which disadvantages \bac l invasions and protects the \phy\ culture.

For example, \bac l invasion into a destructive \phy\ culture resembles a transition from a \PoN\ to \PBN\ system.  In Fig.\ref{f:ydByHarv} \& \ref{f:bacEffect}, yield expectancy (i.e. median) drops to nearly zero.  If the initial \phy\ culture is harvested continuously (i.e. \PoH), then \bac l invasion is mostly unfeasible.  \Bac\ will be naturally eliminated after invasion and the pure \phy\ culture remains.

Death-related parameters ($\aP$ \& $\mB$) are hardly changeable; it depends on the selected microbial strains.  Harvest rate ($x$) is a changeable artificial factor.  \Phy\ growth rate ($\gP$) and bacterial clearance rate ($\gB$) are moderately changeable \autocite{park2020potential}.  They associate with environmental and genetic factors such as temperature, acidity, light intensities, metabolites concentration, carbon use efficiencies (i.e. $\ePR$ \& $\eP$ for \phy; $\eBR$ \& $\eB$ for \bac) and bio-resource allocation (such as quorum sensing).  Factor parameters can be changed by genetic modification \autocite{moniruzzaman1996ethanol} but unchangeable by switching nutrients because of phenotypic plasticity \autocite{j1989respiration,bratbak1985phytoplankton,samejima1958heterotrophic} and interspecies coordinations \autocite{beliaev2014inference,amin2012interactions}.

\end{document}