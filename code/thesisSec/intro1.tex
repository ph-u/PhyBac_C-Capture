% Author: PokMan Ho pok.ho19@imperial.ac.uk
% Script: intro.tex
% Desc: MRes thesis introduction section
% Input: none
% Output: none
% Arguments: 0
% Date: Jan 2020

\documentclass[../thesis.tex]{subfiles} %% use packages & commands as this main file

\begin{document}
\section{Introduction}
%this is an introduction
%% 20200324

Solar energy is a type of renewables that has the potential of powering human settlements due to the high flux to Earth surfaces daily.  Solar panels are attempting to make use of this energy but production of these chemical-based panels can produce different types of pollution.  The production process also involves a non-trivial carbon footprint that may not be compensated throughout its service life.  Hence the use of these solar panels might not be environmentally-friendly.

The loss of green surfaces in planetary scale is observable.  The use of fossil fuels and decreasing photosynthetic cover are undeniable facts to the contribution of anthropogenic carbon dioxide production.  Based on the above reality, can we construct a photosynthetic solar panel with small carbon footprint during panel production?

Phytoplanktons are important autotrophic players contribute to around 50\% carbon fixation service.  Some microbial detritivores, such as \textit{Escherichia coli}, are found to leak electricity naturally along with their metabolism machinary.  Hence logically if one use phytoplankton biomass to feed these detritivores in a stable ecosystem, this ecosystem can sequester carbon dioxide and generate electricity.  If the system can generate biomass faster than consumption, this ecosystem is a net carbon sink with energy storage (in the form of organic matter) and electricity production ability.  In that case, this system is a solar-rechargeable battery.

Due to the huge microbial biodiversity available, an empirical pairwise study of phytoplanktons and detritivores is unfeasible.  So a theoretical study to trim down the number of possible candidates makes this idea tangible.  Through a mathematical model on a phytoplankton-detritivore couple, stability of such system can be examined and parameter ranges can be searched.

From the above motivation and research question, I have formulated the following hypotheses:
\begin{enumerate}
    \item There are parameter spaces for a stable artificial ecosystem serving as carbon sequester and electricity generator;
    \item There are lists of possible microbial candidates match parameter requirements in the model; and
    \item the artificial ecosystem leave smaller carbon footprint than common commercial solar panels with similar service quality.
\end{enumerate}

%% pre-March
%Forests are displaced by human settlements, which means carbon absorption components are replaced by the opposite.  Urbanization, deforestation and the reliance of fossil fuels are increasing the atmospheric carbon content\autocite{ferguson2000electricity,schuur2015climate}.  A solution is to make the current heterotrophic settlements autotrophic, just like plants, using solar energy to sequester carbon and generate energy.

%There are attempts to use cyanobacteria\autocite{joshi2018bacterial,mccormick2015biophotovoltaics,sawa2017electricity} and \textit{Escherichia coli}.\autocite{songera2012electricity}. They successfully generated bio-electricity yet these structures are unsustainable and short-lived.  The major reason is due to the use of single species, which the population can exist within a limited period.  With ecology in mind, these generators can potentially be having a longer service life if a small self-sustaining ecosystem can be established.  In this project, this type of electricity generators are called ``eco-bioelectric cell" (EBC).

%Empirical trials can be carried out to make these self-sustaining ecosystems, but ``trial and error" approach can be resource-hungry and time-consuming.  An alternative approach is to narrow down the trial possibilities through theoretical trials.  Using mathematical models can effectively rule out trivial factors and impossible parameter combinations.  It can potentially make empirical trials more cost-effective with higher successful chances.  In this thesis, I propose the research question: How can an EBC serve carbon sequestration and electricity generation purposes sustaining for multiple years for domestic uses?

%Through addressing the question, I propose the following hypotheses:
%\begin{enumerate}
%    \item EBC can sustain for several years
%    \item Wiring EBCs can add up power output to meet household electricity demands
%    \item Carbon sequestration ability of EBC is biologically-significant
%\end{enumerate}

\end{document}