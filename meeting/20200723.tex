\documentclass[a4paper,11pt]{article}
\usepackage[margin=2cm]{geometry}
\usepackage[english]{babel}
\usepackage{graphicx, hyperref, longtable, amsmath, amssymb}
\graphicspath{{figure/}{../result/}{../report/media/}}

\hypersetup{
	colorlinks=true,
	linkcolor=blue,
	filecolor=blue,      
	urlcolor=blue,
	citecolor=blue
}

\title{Thursday mornings MRes project progress report}
\author{PokMan Ho}
\date{}

\begin{document}
    \maketitle
    
    \begin{tabular}{rl}
        Date Time: & 23-Jul-2020 09:00 \\
        Location: & MS Teams \\
        People: & James, Samraat, Emma, PokMan \\
    \end{tabular}
    
    \section{Topics discussed}
    \begin{enumerate}
        \item result graph: comparing harvest modes
        \item reason for independency of phytoplankton-only systems from harvest interval/rate
        \item literature review sufficiency and bold claims
    \end{enumerate}
    
    \section{Corrections}
    \begin{enumerate}
        \item make a new plot showing a selected biological parameter combination
        \item literature search on a new keyword: ``geoengineering"
        \item Do not need to explicit put a paragraph on literature review; expand the second paragraph talking about the diversity of past research
    \end{enumerate}
    
\end{document}