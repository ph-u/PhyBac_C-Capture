% 20200713
\documentclass[../thesis.tex]{subfiles} %% use packages & commands as this main file
\begin{document}
\section{Discussion}
The model shows a general reality of possibilities in using a coexistence system of bacteria and phytoplankton to continuously harvest carbon from the atmosphere with a biological meaningful quantity.  We define eight biological parameters in this general model and use published experimental measurements to root the basis of the parameter ranges (Table \ref{t:ranges}).  Then we use percentage ranges from data collected for some parameters to apply on data deficient parameters of similar biological meaning.  The model captures a few commonly-known important interactions between microbes and its carbon source (Fig.\ref{f:model}).  With minimal parameters and minimal mathematical complexity, this ordinary differential equation (ODE) model should reflect a biological pattern with the use of real-life parameter values.

With real-life scenarios, we explored the effect of having bacteria coexisting with phytoplankton and/or continuous harvest of organic carbon from the system.  The result shows maximum daily carbon density yield can often be achieved by using the right combination of phytoplankton and bacteria.  Although phytoplankton-only systems are easier to be established and potentially having higher maximum yield than the coexistence system, our model shows that among the few right combinations (326 theoretical combinations of \PBH\ systems under the LHS technique), much less external harvesting effort are needed to get a general high return (Fig.\ref{f:ydByPara}A).

Industrial production of biomolecules is a general interest in production maximisation.  Past studies on bacterial cellulose production showed material production by a pure bacterial culture can be maximised.  In the past few years a study established the experimental maximum to around 284 \dxdt\ \autocite{aytekin2016statistical}.  In the study they fed glucose to the bacteria.  Their system can be grouped into the destructive harvest system in this study.  They replace the bacterial culture and seed a new batch for material extraction.  If we apply changes from this model, theoretically a right phytoplankton can be put into the system and yield an average maximum of around 246 \dxdt.  In that case, the material cost for them can be cut by quite a fraction with similar material output.  By knowing the biological properties of the bacteria of interest, this model can also provide information on biological properties of the optimal phytoplankton choice and optimal extraction frequency within seconds.

For this study, we want to investigate whether bacteria can coexist with phytoplankton without significantly changing the system’s maximum carbon harvest ability.  Our result shows that if we are able to get the optimal combination of phytoplankton and bacteria, the \PBH\ system is generally more productive than the others even with low external effort of harvesting (Fig.\ref{f:ydByPara}A).  However in terms of absolute maximum carbon harvest, addition of bacteria is always an disadvantage to the system.  The reason is because carbon fixed into organic form by phytoplankton can be released back to the atmosphere through bacterial respiration (Fig.\ref{f:model}).  Hence in short, an optimised \PBH\ system is better suited than others within its feasibility limits.

However, the model also shows that if a phytoplankton-only system gets bacterial invasion, bacterial involvement can destroy productivity.  In the model, the P-only system resembles the \PoN\ system in our model.  People usually put up the system to let it run for some time and replace the saturated system with a new one.  The bacterial invasion resembles a transition from a \PoN\ to \PBN\ system.  In Fig.\ref{f:ydByPara}, this transition has a top 10\% productivity median theoretical shift from 16.94 to 0.08 \dxdt.  If the phytoplankton is the best-fit one with invasion of the best-fit bacteria, the theoretical maximum still drops around 100 \dxdt, from 345.70 to 246.28 (Fig.\ref{f:ydByPara}).  On a different perspective, if the phytoplankton culture is designed to incorporate external harvest, which resembles the \PoH\ system, then bacterial invasion might either be ecologically unfeasible (Fig.\ref{f:ydByPara}A, and the small feasible sample size of \PBH\ under LHS scenarios) or increase the system’s productivity if bacteria suits the phytoplankton.  Unfeasible \PBH\ means bacteria will be naturally eliminated and hence the system is ``resilient" towards invasions.  This situation shows the importance of considering external harvest in the initial design of ecological carbon capture devices.

Also, we want to explore for a phytoplankton-bacteria system, whether destructive or continuous harvest can maximise the carbon harvest.  Our result shows a clear support towards continuous harvest within feasibility limits.  Logically speaking, continuous harvest is a daily stable output from a mature carbon cycle within the phytoplankton-bacteria system.  On a contrary, biomass needs time to build up in a destructive system before the system gets replaced \autocite{aytekin2016statistical}.  In terms of the rate of cellular biochemistry, the rate is in its full theoretical potential under continuous harvest scenarios but not in destructive harvesting ones.  Therefore to reach high productivity in destructive systems, grow rates (Fig.\ref{f:ydByPara}D,H) and density-independent death rate (Fig.\ref{f:ydByPara}I) have to be higher than that in continuous systems.

On the last point we are curious about whether biological features of phytoplankton and/or bacteria can significantly affect the system’s carbon harvest function.  Our result shows that biological features of phytoplankton and bacteria affect the carbon harvest yield flux.  But only two of the parameters have unidirectional effect on the flux.  Among the feasible scenarios of \PBH\, $\ePR$ and $\gP$ have positive effects on the yield flux (Fig.\ref{f:ydByPara}B,D).  The flux increase for increasing $\ePR$ can be close to 100 \dxdt\ while that for increasing $\gP$ is around 50 \dxdt.  Limitations posed by the intraspecific interference (Fig.\ref{f:ydByPara}E) is the strictest, which only the minimal $\aP$ value can result in feasible \PBH\ scenarios.  It shows that population size insensitivity is a major concern experimentally when identifying possible phytoplankton candidates.

Among all the nine parameters, death-related parameters ($\aP$ \& $\mB$) are hardly changeable under the current technology -- they can only be changed by choosing appropriate microbial candidates.  On a contrary, harvest rate ($x$) is the easiest to change according to needs because it is an external effort working on the system.  Phytoplankton growth rate ($\gP$) and bacterial clearance rate ($\gB$) are changeable with a number of associated factors \autocite{park2020potential}.  These factors may relate to many environmental and genetic factors such as temperature, media acidity, light intensities, metabolites concentration, carbon use efficiencies (i.e. $\ePR$ \& $\eP$ for phytoplankton; $\eBR$ \& $\eB$ for bacteria) and resource allocation to other cellular activities instead of biomass (such as environmental sensing).  For the group of factor parameters, they can be changed by genetic modification \autocite{moniruzzaman1996ethanol} but hard to be changed via growth media because of phenotypic plasticity \autocite{j1989respiration,bratbak1985phytoplankton,samejima1958heterotrophic} and interspecies coordinations \autocite{beliaev2014inference,amin2012interactions}.

\end{document}