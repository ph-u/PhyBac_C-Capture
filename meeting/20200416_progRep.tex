% Author        : PokMan Ho
% Script        : 20200416_progRep.tex
% Desc          : MRes progress report tex
% Input         : none
% Output        : pdf report in same directory
% Arguments     : 0
% Date          : Apr 2020

\documentclass[a4paper,11pt]{article}
\usepackage[margin=2cm]{geometry}
\usepackage[utf8]{inputenc}
\usepackage[english]{babel}
\usepackage{graphicx, hyperref, longtable, amsmath, amssymb}
\graphicspath{{figure/}{figure/}}

\hypersetup{
	colorlinks=true,
	linkcolor=blue,
	filecolor=blue,      
	urlcolor=blue,
	citecolor=blue
}

\title{Progress report 20200416}
\author{PokMan Ho}
\date{16-Apr-2020}

\begin{document}
    \maketitle
    
    \begin{tabular}{rl}
        Date Time: & 16-Apr-2020 09:00 \\
        Location: & Skype \\
        People: & James, Samraat, Emma, PokMan \\
    \end{tabular}
    
    \section{Follow-up items}
    none
    
    \section{Topics discussed}
    \begin{enumerate}
        \item equilibrium (eqm) convergence between numerical and analytical approaches
        \item temperature condition mismatches between published model parameters
        \item pairwise parameter effects on eqm solution based on a rough parameter space scan
        \item biofuel literature review: how they address carbon harvesting challenge in reality and using mathematical terms
        \item thesis write-up
    \end{enumerate}
    
    \section{Consensus}
    \begin{enumerate}
        \item use ``BioTrait" data from CMEE miniproject folder to obtain $P_0$ \& $B_0$ for Arrhenius equation [$A=A_0exp(E_a/(kT_K))$]
        \item use Arrhenius equation to standardize temperature-dependence of rate terms (i.e. growth rates for photocell \& bacterial decomposer, photocell intraspecific interference \& bacterial decomposer death rate)
        \item write full introduction section \& send out for feedback
        \begin{enumerate}
            \item big question to tackle: how to make an artificial ecosystem of photocell-bacterial decomposer couple to sequester carbon in theory?
            \item importance of carbon sequestration in modern day society
            \item existing methods of carbon sequestration description and critique
            \item knowledge gap in the field of carbon sequestration from existing models (i.e. models comparison)
            \item project novelty: 3-part system using carbon pool density of organic matter, photocell and bacterial decomposer
            \item aim of model: max carbon pool size at given temperature (mid-20$^oC$, as fractions are all given at 23 or 25$^oC$)
        \end{enumerate}
    \end{enumerate}
\end{document}
