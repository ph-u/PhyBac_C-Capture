% 20200816
\documentclass[../thesis.tex]{subfiles} %% use packages & commands as this main file
\begin{document}

Engineering with bio-components have less potential to harm nature than geo-engineering proposals.  This is because the former can be installed in cities and on rooftops; the device is isolated from natural communities.  Geo-engineering proposals however directly perturbing nature; there are risks of unintended consequences.  This study proposes a new model incorporating artificial and life history trait constraints with applications in designing an effective carbon capture device.  The model incorporates \phy\ in a system with the option to add \bac\ and harvest.  The results show the importance of optimising \phy, \bac\ and harvest parameters for maximum carbon harvest efficiency.  The results also show there is a trade-off between feasibility and efficiency; \phy\ cultures have high feasibility but low efficiency while a \pbs\ have low feasibility but high efficiency when optimised.  The choice of system is dependent on the goals of the users and whether the life history traits of \phy\ and \bac\ are controllable.

\subsection{Feasibility of \pbs s}
My model shows \bac\ lowers feasibility of \pbs s (0.3\% \PBH, 97.2\% \PBN) compare with \phy-only systems (100\% \PoH\ \& \PoN; Table \ref{t:feasDist}).  This means a random choice of \phy-\bac\ pair has a lower chance of system establishment.  Substantial work has suggested the importance of chemical mediators in establishment of a high carbon yield co-culture (also known as consortia) and about culture medium optimisations \autocite{rivas2010interactions,amin2009photolysis,fuentes2016impact}.  This study suggests an alternative view on optimising bio-mediators (life history traits).  The results show life history trait values are at least as important as (if not more) environmental factors leading to an establishment of a high carbon yield co-culture; bio-mediators have not been studied in the context of carbon capture yet.  The small feasibility (0.3\%) may explain the few success cases in the experimental tests of high carbon yield \phy-\bac\ pairs \autocite{fuentes2016impact}.  Unfortunately those experiments do not measure life history traits of their organisms of choice; the measurements count on future work for confirmation on whether the experimental life history trait values are those predicted by this model (Fig. \ref{f:harvPB}).

\subsection{Optimal harvest of \phy\ and \phy-\bac\ \pbs s}
Previous work has mixed views on whether destructive or continuous harvest is a better carbon capture approach.  Destructive harvest is proposed by many aquaculture and geo-engineering projects \autocite{lawrence2014efficiency,krause2016substantial}.  Co-culture experiments often adopt continuous harvest methods without considering destructive harvest as an alternative \autocite{kim2008anaerobic,kazamia2012mutualistic}.  My model incorporates both continuous and destructive harvest options which enables direct comparison of the two harvest methods.

The results show there is no optimal harvest for \phy-only systems (Fig. \ref{f:ydDaily}).  This is because \phy\ captures carbon at a constant speed regardless of the carbon density in the model after the system stabilizes (Fig. \ref{f:destCarbon}).  The density-independent carbon capture feature from \phy\ causes a linear accumulation of organic carbon independent from time (Fig. \ref{f:ydDaily}A).  Hence carbon yield for \phy-only systems at equivalent harvest interval/rate have almost the same numerical value; any numerical difference is likely explained by the technical error of the computational methods I used.  The results agree with geo-engineering proposals that adopting destructive harvest by convention is an effective carbon capture method because the alternative harvest method does not have significant carbon yield advantages (Fig. \ref{f:harvPo}).

Continuous harvest is the optimal harvest for \pbs s within feasibility limits (Fig. \ref{f:harvPB}).  This is because \bac\ stabilizes equilibrium carbon densities during development of \pbs s (Fig. \ref{f:destCarbon}A).  As a result longer harvest intervals in destructive harvest have no effect on the carbon content accumulation in the systems; longer waiting time lowers the average carbon yield instead (Fig. \ref{f:ydDaily}A).  Continuous harvest consistently extracts organic carbon out from the system without altering biomass content.  Thus the carbon capture ability from \phy\ is not affected by the harvest.  On the other hand, \bac l growth is limited by the availability of organic carbon; future carbon consumption from \bac\ is also limited.  With smaller \bac l biomass, more carbon can be retained as organic carbon and benefit future harvest of carbon yield.  So a high harvest rate benefits carbon yield from \pbs s (Fig. \ref{f:ydDaily}B).  The results agree with experimental observations of continuous harvest enhances carbon/product yield from \pbs s \autocite{kim2008anaerobic,choix2012enhanced1,choix2012enhanced2,leyva2014accumulation} by providing a mechanistic explanation using ecological interactions between \phy\ and \bac.

\subsection{Importance of life history traits for the organic carbon yield}
Experiments on co-cultures focus on chemical and genetic features of \phy\ and \bac\ bringing the enhanced organic carbon yield effect \autocite{seyedsayamdost2011roseobacticides,durham2015cryptic,amin2009photolysis}.  They used various \phy-\bac\ pairs with different chemical mediators to achieve the enhancement, yet there is not a general explanation on what mechanism causes the enhancement effect.  This model uses a new approach to address effect through bio-mediator (life history traits); the results show trait values are as important (if not more) as chemical ones repeatedly emphasized for the carbon yield enhancement \autocite{fuentes2016impact}.  In the model, carbon yield enhancement can be achieved by expanding the organic carbon pool.  The expansion is maximising the supply and/or minimising the consumption of organic carbon (Fig. \ref{f:harvPB}).

Maximising the carbon supply is maximising \phy\ carbon-use efficiency.  This includes increasing the \phy\ non-respired carbon fraction (increasing $\ePR$; Fig. \ref{f:harvPB}A), increasing the \phy\ carbon leakages (increasing $\eP$; Fig. \ref{f:harvPB}B) and decreasing the \phy\ density-dependent death rate (decreasing $\aP$; Fig. \ref{f:harvPB}D).  Increasing $\ePR$ and $\eP$ increases the amount of carbon leaked into the organic carbon pool and directly benefiting the carbon yield in the short-term.  Decreasing $\aP$ benefits the organic carbon pool in the long-term instead of short-term.  This is because a low $\aP$ keep more carbon as \phy\ biomass.  A high \phy\ biomass enhances photosynthesis rate (Fig. \ref{f:model}) and increases the amount of carbon available for all three processes: respiration, leakage and biomass incorporation.  An enhanced \phy\ respiration is not a concern in this study because the external carbon dioxide pool is always unlimited.  The enhanced carbon leakage from \phy\ directly benefits the organic carbon pool.  The increased carbon incorporated into biomass then forms a positive feedback loop as a long-term enhancement of carbon yield.  Thus \phy\ biomass is functionally more important than its literal carbon content in carbon capture devices.  The results provide a direction for future experimental work to confirm whether the bio-mechanisms of these \phy\ carbon-use efficiencies perform in this logic.

Minimising the carbon consumption is minimising \bac\ carbon-use efficiency.  This includes increasing \bac l respired carbon fraction (decreasing $\eBR$; Fig. \ref{f:harvPB}E), decreasing \bac l leakage carbon fraction (decreasing $\eB$; Fig. \ref{f:harvPB}F) and increasing the \bac l density-independent death rate (increasing $\mB$; Fig. \ref{f:harvPB}H).  Decreasing $\eBR$ and $\eB$ raises the carbon requirements of \bac, which suppress the maximum capacity of \bac.  Decreasing $\eBR$ means more carbon has to be released as carbon dioxide.  Decreasing $\eB$ means more carbon has to be released as organic carbon.  So less carbon can be incorporated into the \bac l biomass, and the maximum capacity of \bac\ decreased.  With less \bac, less carbon will be consumed in the long-term and relatively more carbon can be retained in the organic carbon pool for harvest.  The logic is similar to $\aP$ but in reverse.  Increasing $\mB$ has a short-term effect of increasing the organic carbon pool as well as a long-term effect of suppressing \bac l consumption of organic carbon.  Hence more carbon can be retained for harvest.  The results agree with experimental co-cultures on enhancing product yield by the use of \bac\ \autocite{fuentes2016impact,santos2014microalgal}.  It also simultaneously suggests the biological reasons leading to the enhancement of carbon yield.  The results provide a direction for future experimental confirmation on why a \bacm\ is needed for yield enhancement but the \bacm\ has to be suppressed to the maximum extent.

Two life history traits, \phy\ growth rate ($\gP$; Fig. \ref{f:harvPB}C) and \bac l clearance rate ($\gB$; Fig. \ref{f:harvPB}G), have a peak carbon yield within the range of trait values.  This shows increasing those trait values are beneficial to carbon yield enhancement until their optima; further increasing the trait values recede the carbon benefit.  Increasing $\gP$ and decreasing $\gB$ enhance carbon yields agree with the same logic above: increasing the \phy\ biomass density and/or decreasing the \bac l biomass density benefits the long-term organic carbon accumulation.

The model suggests a high $\gP$ beyond the optimal value also recede carbon yield.  This effect may be brought by the limiting living space for \phy; extremely high \phy\ growth rate would make some fraction of \phy\ biomass non-functional but also non-harvestable.  The redundant \phy\ biomass cannot expand further the organic carbon pool due to space limitations but also cannot be harvested because it is not in the organic carbon pool.  This situation may provide a reason for a lower-than-expected yield in ocean fertilisation geo-engineering proposals \autocite{boyd2008implications,gnanadesikan2008export,oschlies2010side}, where fertilisers only cause a localised algal bloom \autocite{gnanadesikan2008export,lawrence2014efficiency,trick2010iron,kwiatkowski2015atmospheric,lovelock2007ocean}.  In those areas, \phy\ are growing so fast that they cannot disperse to other areas where living space can be a limiting factor.  The results provide a direction for experimental confirmation on the mechanism behind receding yield benefit.

The model suggests that a $\gB$ lower than optimal value benefits carbon yield.  This effect is caused by the trade-off between \bac l consumption and death.  In a low $\gB$ situation, \bac l consumption ($\gB CB$) is small, so as the \bac l death ($\mB$; Eq. \ref{eq:PBH}).  An increase of $\gB$ provides more \bac l biomass to enhance future carbon yield through higher amounts of \bac l death with a minor fraction of respiration (leakage does not matter as carbon flow back to the organic carbon pool).  The increase of \bac l biomass hence is a benefit to future carbon harvest.  The benefit recedes when $\gB$ rises beyond the optimal value, which \bac l respiration is a considerable loss of carbon from the \pbs\ and hinders the yield from organic carbon pool.  The results may reinforce the experimental observations on \bac\ enhancing carbon yield in \pbs s \autocite{fuentes2016impact,santos2014microalgal} by providing a mechanistic explanation.  The results provide a direction for future ground-truthing with experimental measurements.

\end{document}