% 202010??
\documentclass[env.tex]{subfiles}
\begin{document}

Rising atmospheric carbon content is changing the climate.  An effective solution for carbon capture and storage is needed.  There are many proposals on carbon capture but there is no consensus on which ones truly work.  A \phy-\bac\ co-culture is one of these proposals but at present only a few combinations work in laboratories.  This study uses several life history traits to investigate why \phy-\bac\ co-cultures are good at carbon capture but rarely discovered.  A new mathematical model is developed, which simulates the carbon flow in a system of one \phy\ coexist with one \bacm.  Carbon in the model is harvested either destructively or continuously.  Destructive harvest collects all carbon content in the system at once; continuous harvest filters out some organic carbon from the system daily.  The results show that \Rn{1}) feasible systems have specific \phy-\bac\ requirements; \Rn{2}) continuous carbon harvest makes feasible systems superior than \phy-only ones in carbon yield; \Rn{3}) feasible systems maximise carbon yield by maximising \phy\ and minimising \bac\ carbon-use efficiencies.  The results provide a quantitative guide on engineering an optimal \phy-\bac\ co-culture to combat climate change.

\end{document}