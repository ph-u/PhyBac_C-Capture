% Author: PokMan Ho pok.ho19@imperial.ac.uk
% Script: intro.tex
% Desc: MRes thesis introduction section
% Input: none
% Output: none
% Arguments: 0
% Date: Apr 2020

\documentclass[../thesis.tex]{subfiles} %% use packages & commands as this main file

\begin{document}
\section{Introduction}

Climate destabilisation is happening.\autocite{notz2016observed,schuur2015climate}  Anthropogenic emission is affecting the environment\autocite{notz2016observed} but the society still depends on fossil fuels to power the economy.\autocite{lotfalipour2010economic}  Since economy has to be developed by using electricity,\autocite{ferguson2000electricity} how can we minimise our impacts to the globe?

Some suggest 

Aquatic photosynthetic communities are guesstimated as powerful as terrestrial ones in carbon capture ability.\autocite{SCHLESINGER2013341}  Since aquatic phytoplanktons are small sized but fast-growing organisms, their communities can be bottled into different forms, such as panels and bulbs\autocite{evanson_2019}.  If this advantage can be better utilised, carbon footprints from cities and industries can potentially be decreased.  Given the fact that bacterial decomposers are everywhere, it is illogical to neglect their existence (and potential impacts) in an open system (i.e. materials can be exchanged between the set-up and the outside world) in open space.  With the above idea in mind, empirical approaches become too complicated because 1. fit phytoplankton candidate identities are unknown; 2. lineages of bacterial decomposers exist in a system are random and their performances are under-investigated; and 3. too many potential biochemical interactions between individuals, populations and lineages are simultaneously happening with many interactions are yet to be discovered.  Hence a mathematical model is useful to generalise the reality and contain the system to a few defined interactions for detailed investigations.

Many interactive models are built for different purposes, mainly grouped by their growth media (aquatic or terrestrial) and number of players (multi-lineages or multi-organs within an individual).  These models are specific to their respective environmental settings and nutrient limitations.  By comparing model features, a new general model should be made (Table \ref{modComp}).  Apart from the above phenotypical models, there is a genomics community\autocite{harcombe2014metabolic} and specific biofuel\autocite{kirthiga2014mathematical} models published.  These two models are too specific because they require prior knowledge and data regarding the microbes of interest.  Hence this new model is aimed at visualising a minimal ecosystem as a whole, which accommodates known and unknown microbes as long as they fit into either the photocell or bacterial decomposer categories.



\end{document}