% 20200713, 20200720
\documentclass[../thesis.tex]{subfiles} %% use packages & commands as this main file
\begin{document}
\section{Discussion}
With minimal number of parameters and mathematical complexity, our model shows feasibility, influential effects of \bac\ and different harvest modes on \phy\ cultures.  We define eight biological parameters and use published experimental measurements to root the basis of the parameter ranges (Table \ref{t:ranges}).  Percentage ranges are for keeping data-deficient parameters bounded within real-life ranges.  The model captures a few commonly-known important interactions between microbes and its carbon source (Fig.\ref{f:model}).

Among destructive systems, \phy-only systems are more effective than \pbs\ for its linear accumulation of harvestable carbon (Fig.\ref{f:destCarbon}A).  Yet \bac l features determine success of \pbs s.  With the few right combinations, high yield can be achieved with much less harvesting effort compared to that of \phy-only systems (Fig.\ref{f:ydByHarv}).

Industrial production of biomolecules is a general interest in optimisation.  Past studies on \bac l cellulose production showed material production by a pure \bac l culture can be maximised.  In the past few years 284 \dxdt\ was the experimental maximum \autocite{aytekin2016statistical}.  In the study they feed glucose to the \bac\ and carry out destructive harvest.  If we adopt a \phy\ recommendation from this model, theoretically they can have the same yield with a daily continuous harvest.  In that case, this approach can keep their yield and cut their cost.  If biological properties of the \bac\ of interest were known, our model can suggest the optimal choice of \phy\ features and harvest frequency within seconds.

Our result shows that \PBH\ system can yield similar to a \phy-only system if an optimal \bac\ is used (Fig.\ref{f:ydByHarv}).  However \phy-only systems always yield higher because \phy\ is a one-way carbon pump in our model (Fig.\ref{f:model}).  Addition of an optimal \bac\ hence mean sacrificing a fraction of carbon yield for a return of other biochemical services from the \bac, such as biomolecules production \autocite{aytekin2016statistical} or pollution control \autocite{dash2013marine,naik2013lead}.

\Bac l invasion into \phy-only systems destroy productivity unless the systems was designed to continuously harvest carbon.  \Phy-only system resembles the \PoN\ system in our model.  \Bac l invasion resembles a transition from a \PoN\ to \PBN\ system.  In Fig.\ref{f:ydByHarv} \& \ref{f:bacEffect}, yield expectancy (i.e. median) drops to zero.  Productivity still around halved (Fig.\ref{f:bacEffect}) if the invaded \bac\ is the best scenario.  If the \phy\ culture is a \PoH\ system, then \bac l invasion is mostly unfeasible.  It means \bac\ will be naturally eliminated and hence the system is ``resilient" towards invasions.  This situation shows the importance of designing carbon harvest in biological carbon capture devices.

Destructive harvest is\ the best choice only when we cannot control microbial biology.  Or else continuous harvest should be preferred.  The reason is because a \PBH\ system can produce a high yield for specific \bac l biological combinations (Fig.\ref{f:harvPB}E-H) but with a small feasibility limit (0.3\% success).  \PBN\ systems have a smaller maximum yield (around 40 \dxdt\ lower) with around 50\% chance of success.  To reach high productivity in destructive systems, grow rates (Fig.\ref{f:harvPB}C,G) and density-independent death rate (Fig.\ref{f:harvPB}H) have to be higher than that in continuous systems.  Hence continuous harvest is better when recombinant \bac\ is a choice because the biological requirement is comparatively relaxed.

Biological features of \phy\ affect carbon yield.  $\ePR$, $\eP$ and $\gP$ have positive influence but $\aP$ has negative impact.  No conclusions can be drawn for \bac l features (Fig.\ref{f:harvPB}E-H) due to the high number of unfeasible scenarios.  That means a fast-growing carbon-efficient \phy\ insensitive to population density is preferred in any systems but \bac\ choice is case-specific (Fig.\ref{f:bacEffect}).

Death-related parameters ($\aP$ \& $\mB$) are hardly changeable under the current technology -- they can only be changed by choosing appropriate microbes.  Harvest rate ($x$) is the easiest to change because it is an artificial factor.  \Phy\ growth rate ($\gP$) and bacterial clearance rate ($\gB$) are moderately changeable \autocite{park2020potential}.  They associate with environmental and genetic factors such as temperature, acidity, light intensities, metabolites concentration, carbon use efficiencies (i.e. $\ePR$ \& $\eP$ for \phy; $\eBR$ \& $\eB$ for \bac) and bio-resource allocation (such as quorum sensing).  Factor parameters can be changed by genetic modification \autocite{moniruzzaman1996ethanol} but unchangeable by switching nutrients because of phenotypic plasticity \autocite{j1989respiration,bratbak1985phytoplankton,samejima1958heterotrophic} and interspecies coordinations \autocite{beliaev2014inference,amin2012interactions}.

\end{document}