% Author: PokMan Ho pok.ho19@imperial.ac.uk
% Script: method.tex
% Desc: MRes thesis methods section
% Input: none
% Output: none
% Arguments: 0
% Date: Jan 2020

\documentclass[../thesis.tex]{subfiles} %% use packages & commands as this main file

\begin{document}
%this is the methodology
%growth rates are related to the light intensity of the measurement location
\subsection{Coexistence Ecosystem Model}
A set of ordinary differential equations (ODEs) were used to describe the coexistence ecosystem.  All units used were SI units used in physics.  The coexistence ecosystem contained two cyanobacteria lineages \As\ (or \Ss) and \Cs\ ink-jet printed on button mushrooms.\autocite{joshi2018bacterial} . The main equations of perpetuation were:\\
For \As:
\begin{equation}
	dA/dt = [r_A A] - [k_{A1} A] - [k_{A2} A C] - [K_M]
\end{equation}
For \Cs:
\begin{equation}
	dC/dt = [r_C C] - [k_{C1} C] - [k_{C2} A C] - [K_M]
\end{equation}
For button mushroom:
\begin{equation}
	dM/dt = [r_M M] - [k_M M]
\end{equation}

$dA/dt, dC/dt$ and $dM/dt$ were instantaneous growth rates.  $[r_A A], [r_C C]$ and $[r_M M]$ were natural growth terms (growth rates \times\ \ps s).  $[k_{A1} A], [k_{C1} C]$ and $[k_M M]$ were natural death terms (death rates \times\ \ps s).  These natural death rates were assumed to be constants only related to expected cell lifespans.  $[k_{A2} A C]$ and $[k_{C2} A C]$ were growth hinder terms on spatial competition (competition death rates \times\ \ps s).  $[K_M]$ was the cohort removals when host mushrooms died.  Mushrooms were considered as having a continuous dying behaviour affecting the coexistence population immobilized by their umbrella-shaped pilei.  This model would be validated by having population cycles following the insolation cycles.

Growth rates of button mushrooms should be a constant given a stable nutrient supply.  Yet the two cyanobacteria lineages would also be affected by the cellular energy stock accumulated through photosynthesis.  As wavelength of the incident light was positively affecting the energy budget of the cell (hence the energy available for growth), $[r_A A], [r_C C]$ should be considered as a ratio related to the published experimental conditions and the simulated conditions:\\
For \As:
\begin{equation}
    r_A = \dfrac{r_A|_{expt}}{J|_{expt}}\cdot J_{As}
\end{equation}
For \Cs:
\begin{equation}
    r_C = \dfrac{r_C|_{expt}}{J|_{expt}}\cdot J_{Cs}
\end{equation}

$r_A|_{expt}$ and $r_C|_{expt}$ were the growth rates measured experimentally in literature.  $J|_{expt}$ was the respective energy supplied per cell per unit time to the experiment system in respective papers.  $J_{As}$ and $J_{Cs}$ were the energy supply per cell per unit time in the simulated ecosystem.

Competition death rates were assumed only related to energy budgets of the two populations and both populations only compete for space with each other by growth.  Hence these terms were also calculated using ratio of energy gained between the two cyanobacteria lineages.  The calculation assumed the whole area of all mushrooms pilei were covered by either \As\ or \Cs.  The equations were:\\
For \As:
\begin{equation}
    k_{A2} = \dfrac{J_{Ct}}{J_{As} + J_{Ct}}\cdot r_A
\end{equation}
For \Cs:
\begin{equation}
    k_{C2} = \dfrac{J_{At}}{J_{As} + J_{Ct}}\cdot r_C
\end{equation}

The model were validated by using mean hourly, annual and eleven-year cycles.  Hourly means were to validate the daily cycle, annual means were to validate the seasonal cycle and the eleven-year means were to validate the solar cycle.  The data (1947-2019) was requested from Centre of Environmental Data Analysis (CEDA) Archive.\autocite{solarData}

\subsection{Solar Data for Major Cities} %% global solar irradiation amount "glbl_irad_amt", KJ/m^2
Solar data was formatted as csv files (one year per file) in 22 \href{https://artefacts.ceda.ac.uk/badc_datadocs/ukmo-midas/RO_Table.html#definition}{columns} and huge variable number of rows, one hour per available station a row.  Three columns (column 3, 5, 7 and 9; ``," as separator) were extracted representing ``record date and time" (DateTime), ``Met office quality confirmed" (Mqc), ``solar station identifier" (Ssi) and ``global solar irradiation amount (10$^{3}$J/m$^{2}$)" (Gsia) data respectively.  Only data with Mqc column value ``1" was used as instructed.  DateTime column was the hourly timestamp for modelling. Ssi column was used to map the Gsia data with geo-location for modelling.  Gsia column was the solar irradiation data with UV-infrared spectrum ($\lambda$ = 10$^{-9}$-10$^{-3}$ m, or 1nm-1000$\mu$m).  For every yearly-recorded file, the three columns were extracted and concatenated with previous years in a long-table form as a csv file with headers.  Then only the Mqc value =1 rows were extracted out as the cleaned raw data.

Data for global major cities were collected based on ``2019 Global Power City Index"\autocite{GPCI2019_summary} (GPCI).  It was assumed that power consumption positively correlated with city ranks.  Geo-location and area covered for the top 48 cities listed on p.7 of GPCI was gathered through Google Earth Pro\autocite{gEarth} and Google web (search phrase: ``<city> city area") incognito searches respectively.  This information was the reference spatial frame of the model.

Solar station geo-location was formatted in Keyhole Markup Language (kml) file upon unzipping the downloaded kmz file.  Among all details, only four information sets were extracted and rearranged in csv format: ``solar station identifier" (Ssi, ``src\_id" in data) and the 3D coordinates (``coordinates" in data) latitude (lat), longitude (lon) and altitude (alt).  The data was the linker data set between the city location and the hourly insolation data.

%% Data descriptions
Only cities had solar data records were considered in the model.  The three data described above were put together and only the insolation data with geo-location within considered cities were extracted.  Mean values of hourly insolation per area data for cities with more than one insolation record.  The final solar data was containing 
%%insert_num_here
 rows and 6 columns (``Year", ``Month", ``Day", ``Hour", ``Location", ``Mean solar irradiation amount (10$^{3}$J/m$^{2}$)").  It contained solar irradiation data of UV to infrared spectrum timed from 
%%insert_num_here
 stations globally across year 
%%insert_num_here
 to 
%%insert_num_here
 in 
%%insert_num_here
 cities.  This data was the geo-located energy per unit area (10$^{3}$J/m$^{2}$) time series input for the calculation of actual energy budget for cyanobacteria.

``Reference Air Mass 1.5 Spectra" was downloaded from the US Department of Energy \href{https://www.nrel.gov/grid/solar-resource/spectra-am1.5.html}{website}.  This was the current standard curve for solar panels to calculate expected energy output.  The spectra was incorporated into the energy budget calculation within the model.  Gsia column in the solar data was the area-under-curve of the reference spectra within the UV-infrared range.  Energy budget of each cyanobacteria lineage was a proportion of this ranged area which intersect its reported photosynthesis spectra.  Due to the standard spectra, the model was referenced at sea-level.  The altitude effect was not reflected by the solar spectra but only indirectly addressed through the Gsia values.

\subsection{Calculations}
\subsubsection{Electricity generating ability}

\subsubsection{Carbon sequestering service efficiency}

\end{document}