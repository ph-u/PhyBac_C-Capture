% Author: PokMan Ho
% Script: disc.tex
% Desc: MRes thesis discussion section on the application
% Input: none
% Output: none
% Arguments: 0
% Date: Jan 2020

\documentclass[../thesis.tex]{subfiles} %% use packages & commands as this main file

\begin{document}
%\section{Discussion}
\subsection{Strengths and Constrains}
The ordinary differential equation (ODE) model is reflecting a general reality.  It is general due to its spatial, temporal and temperature independence.  A minimal number of terms are used on one \phy\ and one \bac (Eq.\ref{eq:PBH}).  This system is real because each parameter carry realistic biological meaning (Table \ref{t:ranges}) measured through experiments.  The ODE carries one mechanistic component, the intraspecific interference ($\aP$), which makes the ecological relationship testable.  The system is mathematically simple, enhancing readability and adaptability into more complex systems.  Complicated systems with multiple trophic levels can either use an enhanced version of this model or a set of generalised parameters to represent real system's autotrophic and resource cycling ability.

Blanketed assumptions are needed to initiate a testable model in this pilot study.  For example, we have assumed an unlimited resource supply except space.  Yet light attenuation and nitrogen availability are common parameters in location-specific aquatic models (Table \ref{modComp}).  Our assumption is to test whether space sensitivity of \phy\ can be the sole factor for a \pbs\ to achieve stability.  This ideal state is possible through a continuous supply of nutrients with a shallow water column to ensure light and nutrients can reach every part of the system.  These blanketed assumptions are also testable null hypotheses.  With more observations, these assumptions can either be falsified, restricted or kept.  Model variations can also be scripted based on different sets of environmental conditions.

Parameter ranges in this study are not limited to the described species.  It is beneficial in exploring biological possibilities yet fall short on realism.  Due to the lack of biological data, possible microbial candidates for the \pbs\ are not provided.  Most of the parameter values (except \phy\ growth rate $\gP$ and bacterial clearance rate $\gB$) used in our model were from publications decades ago.  Experimental details such as environmental pH, oxygen availability and apparatus preparation hence cannot be standardised.  \Bac l clearance rate was modified from growth rates assuming these two quantities have an arbitrary linear relationship and growth rates of microbes (both \phy\ and \bac) have similar ranges.  For the data collected, there is no species with a full set of information (either [$\ePR$, $\eP$, $\gP$, $\aP$] or [$\eBR$, $\eB$, $\gB$, $\mB$]).  Some \bac l data was also recorded as ``\bac l community" \autocite{cochran1988estimation}, making analyses only achievable in a community level.

\subsection{Applications and Future Directions}
%% theoretical and applicable
The first line of application is constructing carbon capture devices.  Because of the general reality nature, feasibility can be calculated by plugging in biological features in a proposed system.  Under the increasing risk of climate change \autocite{notz2016observed,schuur2015climate}, efficiency in both time and cost is generally preferred.  Since our model uses carbon density as units, it is easy to compare its carbon footprint with other proposals in a LCA.  The only additional carbon footprint from the system is the container.  With a high carbon yield within months after implementation, our \pbs\ is a competitive proposal comparing to the geo-engineering plans.

Further modification can also be incorporated into cathodes of microbial carbon-capture cells (MCC) \autocite{varanasi2020improvement,mohamed2020bioelectricity,neethu2018enhancement,pandit2012microbial}, by using the \bac l component as the electron generator.  \Bac l biomass density (the $B$ pool) can be mathematically translated into electric potential via statistical model-fitting.  Once the relationship is established, bioelectricity from MCC can be potentially maximised by modeling the potential candidates using this ODE.  In that case, MCC can further be elaborated into a microbial battery with continuous wastewater treatment \autocite{mohamed2020bioelectricity} and organic carbon harvest ability.

This flexible model can also be modified or generalised into different scenarios involving autotrophs and chemoorganotrophs.  For example, environmental impacts can be modeled by generalising the ecosystem-of-interest into general parameter values.  Consumers can be mathematically incorporated into other terms (such as the death rates of producers) \autocite{hurtt1996pelagic}.  Alternatively, a consumer equation can be explicitly added into the model to enhance the compatibility to the target system.  By integrating the ODE system, the ecological impacts can be better quantified in the environmental impact assessment (EIA).

Confirmation of whether carbon reabsorption by \phy\ \autocite{j1989respiration,bratbak1985phytoplankton,samejima1958heterotrophic} is an important factor is needed, especially when the organic carbon pool is the sole carbon source for \bac\ in a carbon harvest device.  If reabsorption is important, it will change the relationship of \phy\ and \bac\ from commensalism (this study) to competition \autocite{bratbak1985phytoplankton}.  Hence the equilibrium positions may be shifted.

Intraspecific interference is the density-dependent mechanism for \phy\ to reach population stability \autocite{o2017unexpected,savage2004effects,allen2007recasting,bernhardt2018metabolic}.  In our model, \bac\ has a density-independent death rate.  It will be crucial to investigate whether \bac\ also has a density-dependent death rate as carbon capture ability is potentially be modified if the intraspecific relationship has changed.

All types of organic carbon in our model system is assumed as one.  Yet different classes of biomolecules can have different nutritional values and handling difficulties \autocite{amon1996bacterial}.  So it is important to show whether carbon source preference by \bac\ alter carbon yield.

To further complicate the issue, \phy\ and \bac\ communicate with biochemical signals \autocite{beliaev2014inference,amin2012interactions}.  Communication between these two microbes will raise a question of ``how important do communication influence the carbon yield".  Records on nutrient limitation show \bac\ can switch its nutrient source and compete with \phy\ \autocite{danger2007bacteria}.  This raises a question on how well do wastewater resembles an ``unlimited nutrient source" in reality.

\end{document}
