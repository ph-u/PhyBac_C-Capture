% 20200817
\documentclass[../thesis.tex]{subfiles} %% use packages & commands as this main file
\begin{document}

\subsection{Strengths and Constraints}
This model is spatial, temporal independent and ignores temperature variation; it can be applied across latitudes and longitudes where \phy\ and \bac\ can thrive.  The results provide a general overview on the carbon capture ability of \pbs s within feasibility limitations.  This provides a framework on how the model data can be interpreted and suggests potential criteria on selecting the optimal harvest parameters and life history trait values of \phy\ and \bac.

Enhanced co-cultures were reported resilient towards \bac l invasion experimentally \autocite{fuentes2016impact,seyedsayamdost2011roseobacticides}.  The model in this study may have implications on invasion tolerance for \pbs s; the low feasibility (0.3\%) may indicate but not directly address this issue.  An invasion analysis is needed for confirmation of this \pbs\ property.

The model can be engineered as a panel of co-culture with an optimal \phy-\bacm\ pair; this design is applicable in urban area and on rooftops as a carbon capture device.  A life cycle assessment (LCA) should be needed to test whether the design is environmentally-friendly.  Important factors in the LCA should include resource footprints of materials, logistics, construction, maintenance and disposal.  Carbon debt of the panel includes the logistics for raw materials (\phy, \bac, chemicals and disposable laboratory apparatus), genetic modification materials and power used (if needed), co-culture assembly, device manufacture, delivery of the device to the installation location, installing the device, consistence use of machinery and maintenance for carbon harvest and finally the disposal of the device.  It is worth noting that using \bac\ in the design has a minor impact on the overall carbon footprint but a huge benefit on the carbon yield (Fig. \ref{f:ydDaily}).  Carbon benefit includes the total carbon yield harvest from the system during the service life of the device and the internal carbon content when disposed.  However organic carbon can be consumed and respired by decomposers in nature and release back to the atmosphere.  Only carbon that are not easily consumable by organisms is the ``carbon benefit" of the device.

Method of carbon harvest is an important factor contributing to the carbon footprint \autocite{fuentes2016impact}.  Continuous harvest can reduce as much as a quarter of the carbon benefit from a feasible co-culture \autocite{mata2010microalgae}.  Current harvesting methods include ultrafiltration membranes \autocite{zhang2010harvesting}, electrocoagulation and centrifugation (currently not deployable at large scale) \autocite{wijffels2010outlook} and carbon adsorption \autocite{mata2010microalgae,wang2012novel,lee2014repeated}.  These methods should be incorporated in an LCA and a cost-benefit analysis.  Automated harvesting may be costly in expenses and carbon footprint, so harvesting by hand should be considered in the design.  Harvesting by hand can reduce both harvest costs to zero (because human effort is neglected); the only requirement is to construct an invasion-resistant co-culture which would not genetically pollute the environment.  This is because the hand-held harvesting tools is likely to be contaminated by environmental microorganisms.  Co-culture resistance to invasion is a key to preserve the expected carbon yield; some co-cultures might already have this ecological property \autocite{fuentes2016impact,seyedsayamdost2011roseobacticides}.  Handling the co-culture by hand will also bring genetic materials out of the device.  If non-native genetic material is leaked, native microorganisms may have access to recombine with these DNAs.  Native microbiome may change and affect the ecosystems.  This potential risk is an important concern if genetic modified \phy\ or \bac\ is used.

Broad assumptions are the main constraints for this model.  For example, unlimited nutrient may not always be a valid assumption because some \phy\ have specific nutrient requirements \autocite{kazamia2012mutualistic}.  Nutrient specificity of a \phy\ may require specific \bac\ or artificial fertilisers.  These additional criteria may cause the use of sub-optimal \bac, or even lowers the overall carbon yield from the \pbs\ because production of artificial fertilisers require energy (hence translate to carbon footprints in an LCA analysis if the energy is generated from fossil fuels).  Genetic modification may provide a solution but there is a practical risk to the native environment.

\end{document}