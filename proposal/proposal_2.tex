
% Author: PokMan Ho pok.ho19@imperial.ac.uk
% Script: proposal.tex
% Desc: `LaTex` proposal for MRes 9-months project ver 3
% Input: none
% Output: pdf proposal in same directory
% Arguments: 0
% Date: Dec 2019

\documentclass[a4paper, 11pt]{article}

%% packages
\usepackage[margin=2cm]{geometry}

\usepackage{graphicx}
\graphicspath{{graphics/}}

\usepackage{fontspec}
\setmainfont{Arial}

\usepackage{hyperref, setspace, lineno, longtable, amsmath, amssymb}

%% hyperlinks
\hypersetup{
	colorlinks=true,
	linkcolor=blue,
	filecolor=blue,      
	urlcolor=blue,
	citecolor=blue
}

%% test insert variables
\newcommand{\ReportTitle}{Proposal for MRes CMEE Dissertation Project}
\newcommand{\ReportAuthor}{PokMan HO (CID: 01786076)}
\newcommand{\ReportAffil}{Department of Life Sciences, Faculty of Natural Sciences,\\Imperial College London}
\newcommand{\Disclaim}{\textbf{A Proposal submitted in partial fulfilment of the requirements for the degree of Master of Research at Imperial College London\\\vspace{1cm}
		Formatted in the journal style of the \textit{Nature} Journal\\
		Submitted for the MRes in Computational Methods of Ecology and Evolution}}

%% tabular colouring <https://tex.stackexchange.com/questions/94799/how-do-i-color-table-columns>
\usepackage{xcolor, colortbl}
\definecolor{grey90}{rgb}{.9,.9,.9}
\definecolor{lorange}{rgb}{1,.8,0}

%% 90-deg tabular tab rotation
%\usepackage{rotating, arraycols, booktabs, array}
%\newcolumntype{R}{1}{>{\begin{turn}{90}\begin{minipage}{#1}\scriptsize}l
%			<{\end{minipage}\end{turn}}}}

\title{\ReportTitle}
\author{\ReportAuthor}
\date{}

%% citation
\usepackage[%
%autocite 	= superscript,
backend 	= bibtex,
sortcites 	= true,
style 		= authoryear %nature
]{biblatex}
\bibliography{../reference/proposal.bib}

%% set as required
\onehalfspacing %% <https://tex.stackexchange.com/questions/30073/why-is-the-linespread-factor-as-it-is>
\linenumbers

\begin{document}
	\begin{center}
		\Huge\textbf{\ReportTitle}\\
		\LARGE\ReportAuthor\\
		\Large\ReportAffil\\
		\Large{Last modified: Dec 2019}
	\end{center}
	\begin{figure}[h]
		\centering\includegraphics[width=\linewidth]{icl.jpg}
	\end{figure}
Primary supervisor:\\
\indent Dr. James Rosindell\\
\indent Department of Life Sciences (Silwood Park), Faculty of Natural Sciences, Imperial College London\\
\indent email: \href{mailto:j.rosindell@imperial.ac.uk}{j.rosindell@imperial.ac.uk}\\
Secondary supervisor:\\
\indent Dr. Samraat Pawar\\
\indent Department of Life Sciences (Silwood Park), Faculty of Natural Sciences, Imperial College London\\
\indent email: \href{mailto:s.pawar@imperial.ac.uk}{s.pawar@imperial.ac.uk}
\clearpage
\section{Keywords}
ODE model, ecological photovoltaics, electricity, carbon sequestration, co-existence, nutrient cycling
\section{Introduction}
Two main substrates were able to print cyanobacteria for electricity:  \textit{Synechocystis sp.} on carbon nanotube-coated paper \autocite{sawa2017electricity} and \textit{Anabaena spp.} on mushrooms \autocite{joshi2018bacterial}.  Both successfully generate small photocurrents (micro-ampere on paper \autocite{sawa2017electricity} and nano-ampere on mushrooms \autocite{joshi2018bacterial} respectively).  Under the urge of clean energy due to climate change \autocite{schuur2015climate}, our civilization needs electricity as well as carbon sequestration services simultaneously and quickly.  We can augment photosynthesis by inserting carbon nanotubes inside phytocells \autocite{giraldo2014plant}, or incorporate biodiversity.  Freshwater themo-extremophile \textit{Chroococcidiopsis thermalis} were found to utilize far red spectrum as light source to initialize photosynthesis apart from ordinary ones \autocite{nurnberg2018photochemistry}.  If this lineage is able to provide photocurrent, it would be a step towards the solar-powered civilization.\\

With ecology in mind, this project aims at finding the theoretical recipe to make a multi-spectra bio-solar panel with carbon sequestration ability.  The research questions are:

\begin{itemize}
	\item What ingredients are necessary to build a healthy minimal ecosystem panel?
	\item How much ingredients do such a system need for initialization?
	\item Which organisms can coexist and contribute in both electricity generation and carbon sequestration aspects?
\end{itemize}

\section{Proposed methods}
A set of ordinary differential equations (ODEs) would be implemented as a Lotka-Volterra model (LVM).  The cyanobacteria-mushroom system \autocite{joshi2018bacterial} would be the stem of this model.  The main equations included would be:\\
\begin{tabular}{r|rl}
	content...
\end{tabular}
Within a annual insolation cycle, ordinary differential equations (ODEs) would be scripted in C and solved in Python (ver 3.7.3).  Using insolation and sea-level temperature data from different latitudes, the system would be expected to achieve a resource-consumer population balance assuming light and thermal input are the only external independent variables for this ecosystem.  Atmospheric compositions were assumed to be constant and at sea-level.  Physical and chemical parameters will be obtained from general literatures describing habitats of considered lineages.  All cyanobacteria were assumed having same behaviour towards photocurrent stimuli.\\\\
%Solved ODE parameters would be analysed by principal component analysis (PCA) in R (ver 3.6.0) to identify important ingredients for the system.  If more than one optimal system statuses were founded, Kruskal-Wallis tests with posthoc Nemenyi pairwise comparisons would be carried out to identify the best system among all possibilities.

\section{Anticipated outputs and outcomes}
A animated model simulation of the best panel achieving ecosystem equilibrium would be expected.  Initializing chemicals and environmental factors would also be listed in a file.

\section{Project feasibility}
Past publications have many considered parameters.  Hence completing the project is considered positive.

Key:
\begin{tabular}{lll}
	section & target & action\\\hline
	1 = abstract & m = model & s = scripting\\
	2 = introduction & a = analysis & d = debug\\
	3 = methodology & p = parameters & v = validation\\
	4 = results && \cellcolor{grey90}work\\
	5 = discussion && \cellcolor{lorange}buffer / if any\\
\end{tabular}\\
Gantt chart:
\begin{longtable}{p{.1\linewidth}|p{.1\linewidth}|p{.1\linewidth}|p{.1\linewidth}|p{.1\linewidth}|p{.1\linewidth}|}
	Month	&write					&design					&script					& \begin{tabular}{c}MSc\\lecture\end{tabular}	& \begin{tabular}{c}model\\run\end{tabular}\\\hline
	Dec		&						&\cellcolor{grey90}map	&\cellcolor{grey90}ms	&\cellcolor{lorange}							&										\\
	Jan		&\cellcolor{grey90}3	&\cellcolor{grey90}m	&\cellcolor{grey90}mdv	&\cellcolor{lorange} 							&\cellcolor{grey90}						\\
	Feb		&\cellcolor{grey90}2	&\cellcolor{grey90}a	&						&\cellcolor{lorange} 							&\cellcolor{grey90}						\\
	Mar		&\cellcolor{lorange}3	&\cellcolor{lorange}	&\cellcolor{grey90}as	&\cellcolor{lorange} 							&\cellcolor{grey90}						\\
	Apr		&\cellcolor{grey90}45	&						&\cellcolor{grey90}		&												&\cellcolor{grey90}						\\
	May		&\cellcolor{grey90}		&						&\cellcolor{grey90}		&												&\cellcolor{grey90}						\\
	Jun		&\cellcolor{grey90}		&						&\cellcolor{lorange}	&												&\cellcolor{lorange}					\\
	Jul		&\cellcolor{grey90}1	&						&						&												&\cellcolor{lorange}					\\
	Aug		&\cellcolor{lorange}	&						&\cellcolor{lorange}	&												&\cellcolor{lorange}					\\
\end{longtable}
\section{Budget}
This project is free
\section*{Acknowledgement}
The author would like to thank \href{mailto:wei.huang@eng.ox.ac.uk}{Prof. Wei Huang} for agreeing to be the external adviser for this project.
\clearpage
\nocite{*}\printbibliography
\end{document}
