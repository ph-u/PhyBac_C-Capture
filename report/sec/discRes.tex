% 20200816
\documentclass[../thesis.tex]{subfiles} %% use packages & commands as this main file
\begin{document}

Engineering with bio-components have less potential harm comparing with geo-engineering proposals because the former does not perturb nature.  Since the engineered devices with bio-components can be installed in cities and on rooftops, natural processes are kept untouched.  Hence natural balances and ecosystem feedback systems are not affected and the engineering with bio-components bear small risks of having large scale unintended consequences.  This study proposed a new model on laying out artificial and life history traits constrains when we want to engineer these carbon capture devices with living components.  The model quantitatively shows the importance of choosing the optimal \phy, \bac, harvest mode and harvest interval/rate when engineering an efficient carbon capture device.  The result shows there is not a single systems serves all purposes; the choice of system depends on what the users want to achieve and whether we can control the life history traits of \phy\ and \bac.

\subsection{Feasibility of \pbs s}
Feasibility of \pbs s are always lower than that of \phy-only systems (Table \ref{t:feasDist}).  The lower feasibility means a wild search of a \phy-\bac\ pair has a probability of system establishment.  The result shows the probability is 0.3\% using a continuous harvest mode and 97.2\% using a destructive harvest mode (Table \ref{t:feasDist}).  The small probability can potentially explain the few success cases in the experimental search of high carbon yield \phy-\bac\ pairs \autocite{fuentes2016impact}.  Life history traits of the \phy\ and \bac\ can be measured for these published pairs in co-cultures.  Such measurements can confirm whether these existing efficient co-cultures bear the trait values within the feasible ranges quantified in this study (Table \ref{t:ranges}).  The measurements can also test the hypothesis of whether chemical mediators from \phy\ and/or \bac\ are important for an effective carbon harvest engine in general \autocite{rivas2010interactions,amin2009photolysis,fuentes2016impact}.

\subsection{Harvest preferences on different \phy\ and \phy-\bac\ \pbs s}
The result shows no preferences for \phy-only systems (Fig.\ref{f:ydDaily}).  The reason is because \phy\ is a carbon pump in the model (Fig.\ref{f:model}).  The rate of carbon capture by \phy\ is depending on rate of photosynthesis, respiration, leakage, biomass growth and intraspecific interference.  These rates in the model are either fixed values (resource allocation, biomass growth and interspecific interference) or a combination of fixed parameters (photosynthesis, respiration and leakage rates).  Hence equilibrium positions of \phy\ biomass density should be similar between destructive and continuous harvest modes (Fig.\ref{f:ydByHarv}\&\ref{f:harvPo}).  The result cannot falsify that continuous harvest is a growing concept as an effective carbon harvest method \autocite{fuentes2016impact} but in practice aquaculture and geo-engineering proposals adopt the destructive harvest mode by default \autocite{lawrence2014efficiency,krause2016substantial}.

Continuous harvest mode is necessary for \phy-\bac\ \pbs s.  The result shows continuous harvest is significantly higher than destructive harvest among the feasible scenarios across all harvest intervals/rates (Fig.\ref{f:ydDaily}).  The effect of high product yield is also observed from the production of hydrogen \autocite{kim2008anaerobic}, carbohydrates \autocite{choix2012enhanced1,choix2012enhanced2} and fatty acids \autocite{leyva2014accumulation}.  The reason for the lack of consensus support is because the feasibility

\subsection{Importance of life history traits on the organic carbon yield}

\end{document}