% 20200814, 20200815
\documentclass[../thesis.tex]{subfiles} %% use packages & commands as this main file
\begin{document}

Climate change is affecting not only biosphere, but also human society \autocite{notz2016observed,schuur2015climate}; there are intensified extreme weather events, season shifts and tropical diseases spreading poleward.  Although carbon emissions are major contributors of climate change \autocite{notz2016observed}, fossil fuels are still powering cities and industries \autocite{ferguson2000electricity}.  It is urgent to minimise and revert climate damages.

For decades, experts keep proposing geo-engineering plans for climate mitigation and adaptation \autocite{farrelly2013carbon,yang2008progress,boyd2008ranking,boettcher2019high,vaughan2011review}.  There are two major categories \autocite{boyd2008ranking}: carbon capture and storage (CCS) and solar radiation reduction.  Without standards and consensuses, debates and reviews on the benefits, costs and risks of the proposals are still ongoing with different criteria and metrics \autocite{boyd2008ranking,boettcher2019high,boyd2016development,oschlies2017indicators,gattuso2018ocean}.

An CCS example is fertilising open ocean \autocite{gnanadesikan2008export,lovelock2007ocean,lawrence2014efficiency} or intertidal autotrophs \autocite{duarte2017can,johannessen2016geoengineering,krause2016substantial}.  Utilizing artificial fertilisers \autocite{gnanadesikan2008export,lawrence2014efficiency,trick2010iron} or piping nutritious deep ocean water \autocite{kwiatkowski2015atmospheric,lovelock2007ocean} are the mainstreams.  However, the extensive fertilisation may have low carbon yield \autocite{boyd2008implications,gnanadesikan2008export,oschlies2010side}, regional ocean acidification \autocite{oschlies2010side}, local ecological resource competition imbalances \autocite{chung2011using,thiele2012microbial,batten2014did} and regional atmospheric pressure anomalies \autocite{kwiatkowski2015atmospheric}.  Fertilising high-nitrate-low-chlorophyll water may also trigger toxic algal blooms which harms the regional ecosystem \autocite{trick2010iron}.

A solar radiation reduction example is injecting reflective particles to the high atmosphere as sun-shields \autocite{latham2008global,stjern2018response,xia2016stratospheric}.  The expensive sun-shields \autocite{boyd2008implications} can theoretically pause greenhouse effect by blocking sunlight \autocite{williamson2012impacts}.  Sunlight reduction, however, can collapse thermal-sensitive food webs \autocite{williamson2012impacts}, aggravating extreme weather events \autocite{jones2009climate} and weakening the long-term deep ocean thermal circulation \autocite{lauvset2017climate,williamson2012impacts}.  Additionally, salt particles and soluble aerosols can dissolve in rainwater and make the rain salty \autocite{bala2011albedo}.  High salt concentration in rain put rainforests at risk \autocite{muri2015tropical}.

Geo-engineering plans for climate mitigation and adaptation are expensive because they require lots of electricity and chemicals \autocite{boyd2008ranking,boyd2008implications,mcclellan2012cost}.  Electricity generation and chemical extraction currently consume fossil fuels, which contribute to climate change.  A ``life-cycle assessment" (LCA) is a numerical summary of an engineering proposal, calculating the overall environmental benefits, carbon footprints and environmental damages associated with it.  Many geo-engineering proposals are unsustainable under LCA \autocite{abdussalam2020green}, including fertilisation and particle injection plans mentioned above.

A potential sustainable CCS proposal is using \phy\ \autocite{farrelly2013carbon} or co-cultures (also known as consortia) of \phy\ and \bac\ \autocite{fuentes2016impact}.  Several \phy-\bac\ co-cultures have been discovered for biomass and biomolecules production in the last decade \autocite{fuentes2016impact,santos2014microalgal}.  In these publications, only chemical/genetics \autocite{amin2009photolysis,durham2015cryptic,leyva2014accumulation,rivas2010interactions,seyedsayamdost2011roseobacticides} or phenomenological \autocite{choix2012enhanced1,choix2012enhanced2,kazamia2012mutualistic,santos2014microalgal} aspects were recorded.  Underlying bio-mechanisms (e.g. life history traits) are not the mainstream approach, making room for investigations explaining why these combinations are special.

For example, chemical mediators are popular targets investigating \phy\ growth and carbon capture enhancement \autocite{fuentes2016impact}.  These biochemicals are signalling molecules to enhance carbon capture of a co-culture.  Genetic approach understands what sequence encode the chemical while phenomenological approach understands the level of carbon capture enhancement.  But neither of the information provide evidence on how the enhancement is carried out, which relates to the life history traits of \phy\ and \bacm\ in the co-culture.  This study uses the life history trait values of \phy\ and \bac\ to fill in this knowledge gap on bio-mechanisms.  The method of carbon harvest is also investigated in this study, which is a crucial practical factor when deploying co-culture systems to capture atmospheric carbon.

In the past few decades, dozens of mechanistic ecosystem models are published but these models are unsuitable for this study.  They are based on site-specific nutrient (nitrogen and/or phosphorus) cycles (Table \ref{modComp}) focusing on trophic level interactions on multiple resource pools \autocite{llebot2010role,mitra2009closure,findlay2006modelling} with thermocline interventions \autocite{anderson2015empower,kidston2013phytoplankton,llebot2010role}.  They are unsuitable because I focus on a single resource (carbon dioxide) pool in a co-culture.  Life history traits such as death rate \autocite{anderson2015empower,kidston2013phytoplankton} and resource allocation \autocite{xiao1996relative} were rarely addressed mechanistically in those ecosystem models.  Combining models are impossible because the death rate was in aquatic models but resource allocation was in terrestrial ones.  A new mechanistic model is necessary which simulates carbon flow in a \pbs\ with one \phy, one \bacm\ and one organic carbon pool.

Using the new model, I address \Rn{1}) the feasibility of \pbs s, \Rn{2}) optimal harvest on different \phy\ and \phy-\bac\ \pbs s and \Rn{3}) importance of life history traits of \phy\ and \bac\ on the organic carbon yield.

\textbf{How common are feasible \pbs s in different harvest modes?}  Feasibility is defined as finite non-negative carbon densities in each system component in a given scenario.  The harvest mode is defined as either ``destructive” or ``continuous”.  Destructive harvest extracts all carbon from a system after letting it run for some time.  Continuous harvest extracts organic carbon at a steady rate while the system is running.  Knowing the feasibility limits can aid investigations on the fragmented understanding of \pbs s \autocite{fuentes2016impact}.  The information can show whether \pbs s is an effective method for CCS.

\textbf{In what conditions do \pbs s yield optimally from the two harvest modes?}  Destructive harvest is a common practice in agricultures and is extended to aquacultures of macroalgae \autocite{duarte2017can} and \phy\  \autocite{evanson_2019}.  Continuous harvest is a growing concept for product yield enhancement in various microbial cultures \autocite{aytekin2016statistical,fuentes2016impact}.  At present, there is insufficient information to judge which harvesting mode is superior under what conditions.  Evaluating both harvest modes quantitatively can give direction on designing protocols to construct, deploy and maintain \pbs s for practical CCS purposes.

\textbf{To what extent do life history traits influence organic carbon yield from \phy\ cultures and \pbs?}  At present, only a handful of co-cultures are discovered experimentally \autocite{fuentes2016impact,santos2014microalgal}.  Because of rarity, experimental methodologies are case-specific.  There are no checklists on general features of a suitable \phy/\bacm\ for CCS co-cultures.  In practice, the co-cultures are typically optimised by growth media \autocite{aytekin2016statistical,fuentes2016impact} and interspecies signalling \autocite{fuentes2016impact} for high carbon yield.  The optimisations show triggers of the enhancement effect but not bio-mechanisms (i.e. life history traits) directly behind the enhancement.  This project deciphers the life history traits of \phy\ and \bac, showing why \pbs s are an effective CCS option.

The new model of this study resembles an open system (material exchange across system boundaries is allowed) of carbon capture engineered with one \phy\ and one \bacm.  Carbon captured is harvested by either destructive or continuous manner.  The model provides a quantitative guide on choosing the optimal \phy\ and \bac\ accompanying with a suitable harvest method.

\end{document}