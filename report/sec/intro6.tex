% 20200713
\documentclass[../thesis.tex]{subfiles} %% use packages & commands as this main file
\begin{document}

Climate change is affecting not only the ecosystem, but also human society \autocite{notz2016observed,schuur2015climate}.  Extreme weather, season regime shifts and tropical disease have spread towards temperate regions have intensified.  Although we know carbon emissions are major climate change contributors \autocite{notz2016observed}, we currently need fossil fuels to power our cities and industries \autocite{ferguson2000electricity}.  Minimise or even revert the damages to our climate is a urgent task to complete.

For decades, experts have proposed geo-engineering plans for climate mitigation and adaptation \autocite{farrelly2013carbon,yang2008progress,boyd2008ranking,boettcher2019high,vaughan2011review}.  Two major categories were distinguished \autocite{boyd2008ranking}, one on carbon capture and storage and the other one on solar radiation reduction.  Debates and reviews are still ongoing because we do not have a standard scheme and/or model to judge which method(s) work(s) or not \autocite{boyd2008ranking,boettcher2019high,boyd2016development,oschlies2017indicators,gattuso2018ocean}.

Fertilising planktons \autocite{gnanadesikan2008export,lovelock2007ocean,lawrence2014efficiency} or large autotrophs \autocite{duarte2017can,johannessen2016geoengineering,krause2016substantial} can compensate carbon absorption from bulldozed ecosystems.  But the expensive operation may have a much lower yield than expected \autocite{boyd2008implications,gnanadesikan2008export,oschlies2010side}.  Fertilisation may cause water column acidification \autocite{oschlies2010side} and imbalanced local resource competition \autocite{chung2011using,thiele2012microbial,batten2014did}, destroying local ecosystems.  Piping deep ocean water to the surface may also disrupt atmospheric pressure balance \autocite{kwiatkowski2015atmospheric}.  In high-nitrate-low-chlorophyll areas, fertilisation may even cause toxic algal blooms \autocite{trick2010iron}.

Releasing particles in high atmosphere and clouds as sunshield \autocite{latham2008global,stjern2018response,xia2016stratospheric} can reduce incoming solar radiation which theoretically halt greenhouse effect \autocite{williamson2012impacts}.  Yet the expensive blanketing \autocite{boyd2008implications} may cause unpredictable global primary production feedback in the short term and weaken the deep ocean thermal circulation in the long term \autocite{lauvset2017climate,williamson2012impacts}.  Rainwater chemistry and natural climate patterns can also be altered for a long time with unpredictable impacts on local terrains and ecosystems \autocite{bala2011albedo}, such as salt damages to rainforests and soil \autocite{muri2015tropical}.  Thermal-sensitive species and associated food webs may collapse due to rapid changes in light intensity and local temperatures \autocite{williamson2012impacts}.  In simulations, such global operations can bring ongoing extreme weather to further extremes in both human settlements and natural ecosystems \autocite{jones2009climate}.

Fixing carbon using rocks \autocite{stephens2008assessing,gunnarsson2018rapid} or ocean \autocite{salter200920,sant2014offshore} can amplify the current rate of carbon storage in natural sinks.  But sudden increase in local carbon dioxide concentration will acidify the seafloor, dissolving major alkaline rock basalt and may destroy local ecology \autocite{trias2017high,ridgwell2011geographical}.  The injection may also alter long-term groundwater chemistry in wider area \autocite{trias2017high}.

Directly filtering atmospheric carbon dioxide \autocite{broecker2007co2,williamson2016emissions} can make the smallest change in natural environment.  Yet carbon footprints from manufacturing the chemicals used for carbon capture are not taken into consideration \autocite{stephens2008assessing}.

Public perception on the above proposals are also obstacles for these methods to be tested \autocite{kamishiro2009public}.  Difference in benefit-cost judgement, community ethics and faith on the organisations carrying out such operations cause oppositions.  When the scientific community is still debating on the geo-engineering plans, it is logical that the general public do not support any of them.  Hence in recent years scientists are still urging for emergency actions for our climate \autocite{lenton2019foresight,van2017open}.

Geo-engineering plans usually require a lot of power and raw materials \autocite{boyd2008ranking,boyd2008implications,mcclellan2012cost}.  Electricity and material extraction requirements symbolise the extensive use of fossil fuels.  Calculation of the overall benefits, carbon footprints and environmental damages within the lifetime of a device is a ``life-cycle assessment” (LCA).  Through LCA, most of the geo-engineering mitigation plans and renewable energies are unsustainable \autocite{abdussalam2020green}.  For example, manufacturing solar panels requires different chemicals and rare earth metals.  Material extraction, material delivery, panel assembly and all associated machines use a large amount of materials and power.  Hence the expected lifespan of the solar panel cannot cover all the environmental debt \autocite{martinopoulos2020rooftop}.  For a proper LCA, chemical pollution caused by the disposal of solar panels should also be taken into consideration.  Hence a potential sustainable solution is to use the natural methods, for example the \phy\ \autocite{farrelly2013carbon}.

\Phy\ can capture around double the amount of carbon than that of land plants \autocite{SCHLESINGER2013341} because of the high growth rate.  Fluid plankton cultures can also be easily shaped by different containers \autocite{evanson_2019}.  Although the solution sounds ideal, the \phy\ carbon capture device is exposed in air and \bac\ can invade into the system.  Hence \bac l damages are an important problem to mitigate.  On the other hand, \bac\ is useful.  Different types of \bac\ are currently used in industries for pollution control \autocite{dash2013marine,naik2013lead}, medicine production \autocite{huang2012industrial} and electricity generation \autocite{songera2012electricity}.  Currently microbial engineering studies are focusing on pure cultures and the usage of single species.  Hence our project set foot on the non-ideal situations, when \bac\ and \phy\ coexist in the same culture.  We use carbon harvest as our investigation target as it is a popular application of \phy.

However, where to start?  Billions of \phy\ and \bac\ exist around the world and hence trial-and-error would take forever and be unaffordable.  In this case, we can seek for a simpler alternative by using mathematical models.  There are dozens of ecosystem models published in the past few decades.  Yet they are constructed based on nutrient cycles (nitrogen and/or phosphorus) in their respective project sites (Table \ref{modComp}).  So these published models do not fit our purpose because they cannot represent general \phy-\bac\ relationships and/or do not resemble the carbon flow within the system.  So to answer our question, we have proposed a new mathematical model.  The model simulates how carbon flows and recycles in a simple \pbs\ when we harvest carbon from it.

Using the model, we want to address three aims listed below.

Can \bac\ coexist with \phy\ while preserving the function of carbon harvest?  Harvest mode can be interpreted as either ``destructive” or ``continuous”.  Destructive harvest means we extract all carbon from a system after letting it run for some time.  Continuous harvest means we keep extracting carbon at a precise rate when the system is running.  Two applications can be applied with either \phy\ or \bac\ as the organism of interest.  When pure \phy\ culture is used for capturing atmospheric carbon \autocite{evanson_2019}, \bac l invasion can destroy its function.  Knowing the optimal harvest strategy can help mitigate potential \bac l damages and enhance feasibility of these devices.  Also, \bac\ is currently used in medical and electrical industries.  Feasibility of \pbs s hence provide insights on cost reduction by using \phy\ as carbon sources for \bac.

In what conditions would a \pbs\ prefer destructive or continuous harvest?  Destructive harvest in our definition is experimentally supported by the production of bacterial cellulose.  A semi-continuous growth media replacement in the study raises the average yield \autocite{aytekin2016statistical}.  Continuous harvest can also be preferred because cell death accumulates carbon.  In a stable \pbs, death rates for both \phy\ and \bac\ are at maxima.  Hence there should also be conditions for continuous harvest being the optimal strategy but the topic is not investigated in publications.

Are biological features of \phy\ and/or \bac\ influencing carbon yield?  Organic carbon capture devices and industrial \bac\ systems are the two application targets.  Influence of \bac l biology can inform whether strains of invading \bac\ can cause difference in carbon yield in carbon capture and influence of \phy\ biology can inform the method of carbon source optimisation for industrial \bac\ cultures.  Most productivity optimisation focus on growth media \autocite{dash2013marine,naik2013lead,huang2012industrial,evanson_2019} but neglecting the possibility that the organism using might not be the biologically best-fit species \autocite{huang2012industrial}.  Knowing the optimal combination of key biological features can reflect on how to enhance the current productivity optimisation strategy.

Our new model resembles a real-life \pbs\ because it is an open system (material exchange across system boundary is allowed).  Prediction on yield at different times and stable positions of the systems are possible.  Hence this model is useful in understanding the optimal strategies for organic carbon capture devices and industrial \bac l cultures.

\end{document}