% 20200814, 20200815
\documentclass[../thesis.tex]{subfiles} %% use packages & commands as this main file
\begin{document}

Climate change is affecting not only the ecosystem, but also human society \autocite{notz2016observed,schuur2015climate}.  Extreme weather, seasonal regime shifts and tropical disease have spread towards temperate regions have intensified.  Although we know carbon emissions are major climate change contributors \autocite{notz2016observed}, we currently need fossil fuels to power our cities and industries \autocite{ferguson2000electricity}.  Minimising or even reverting the damages to our climate is a urgent task to complete.

For decades, experts have proposed geo-engineering plans for climate mitigation and adaptation \autocite{farrelly2013carbon,yang2008progress,boyd2008ranking,boettcher2019high,vaughan2011review}.  Two major categories were distinguished \autocite{boyd2008ranking}, one on carbon capture and storage and the other one on solar radiation reduction.  Debates and reviews are still ongoing because we do not have a standard scheme and/or model to judge which method(s) work(s) or not \autocite{boyd2008ranking,boettcher2019high,boyd2016development,oschlies2017indicators,gattuso2018ocean}.  More importantly, intervening natural processes may cause unintended consequences.

One intervention example is large scale fertilisation of open ocean \autocite{gnanadesikan2008export,lovelock2007ocean,lawrence2014efficiency} or intertidal autotrophs \autocite{duarte2017can,johannessen2016geoengineering,krause2016substantial}.  Fertilisation can be done using artificial fertilisers \autocite{gnanadesikan2008export,lawrence2014efficiency,trick2010iron} or deep ocean water \autocite{kwiatkowski2015atmospheric,lovelock2007ocean}.  However the expensive operation may have lower carbon yield than expected \autocite{boyd2008implications,gnanadesikan2008export,oschlies2010side} while risking regional acidification \autocite{oschlies2010side}, imbalance ecological resource competition \autocite{chung2011using,thiele2012microbial,batten2014did} and atmospheric pressure anomaly \autocite{kwiatkowski2015atmospheric}.  Toxic algal blooms are also a risk in high-nitrate-low-chlorophyll water \autocite{trick2010iron}.

Another intervention example is large scale reflective particles injection to high atmosphere as sun-shields \autocite{latham2008global,stjern2018response,xia2016stratospheric}.  The expensive blanketing \autocite{boyd2008implications} can theoretically pause greenhouse effect by a global dimming \autocite{williamson2012impacts}.  But dimming can also collapse thermal-sensitive food webs \autocite{williamson2012impacts}, making extreme weather events even more extreme \autocite{jones2009climate} and weaken the deep ocean thermal circulation for a long time \autocite{lauvset2017climate,williamson2012impacts}.  Salt particles and soluble aerosols can also fall back to Earth through precipitations and deal salt damages to ecosystems \autocite{bala2011albedo}.  Injection of salt to high atmosphere hence makes high rainfall regions vulnerable, such as rainforests \autocite{muri2015tropical}.

Geo-engineering plans are expensive because they require lots of power and chemicals \autocite{boyd2008ranking,boyd2008implications,mcclellan2012cost}.  Power generation and chemical extractions rely on the supply of fossil fuels, which is the main contributor of climate change.  Calculation of the overall benefits, carbon footprints and environmental damages within the lifetime of a device is a ``life-cycle assessment" (LCA).  Most geo-engineering plans are unsustainable under LCA \autocite{abdussalam2020green}.  Hence a potential sustainable solution for carbon capture and storage is using \phy\ \autocite{farrelly2013carbon} or co-cultures of \phy\ and \bac\ \autocite{fuentes2016impact}, using nature to cure natural problems.

Several \phy-\bac\ co-cultures were found beneficial in biomass and biomolecules production in the last decade \autocite{fuentes2016impact,santos2014microalgal}.  Although yield production from these co-cultures are high, the mechanistic reasons for the production level were not investigated.  Biochemical interactions between \phy\ and \bac\ were recorded and explained by chemical/genetics \autocite{amin2009photolysis,durham2015cryptic,leyva2014accumulation,rivas2010interactions,seyedsayamdost2011roseobacticides} or phenomenological \autocite{choix2012enhanced1,choix2012enhanced2,kazamia2012mutualistic,santos2014microalgal} reasons.  These information is yet insufficient to show mechanistically why these special \phy-\bac\ combinations have enhanced carbon capture ability.

For example, siderophores are a popular chemical mediator from \bac\ from publications in enhancing \phy\ growth and carbon yield \autocite{fuentes2016impact}.  Yet what bio-mechanical pathways are up-/down-regulated by siderophores are not mentioned.  Biological importance of such biochemical regulations on respective pathways are hence never discussed, not to mention whether siderophores are necessary to achieve high yield scenarios.  Hence this study set foot on mechanistically addressing the importance of life history trait values of \phy\ and \bac.  Method of carbon harvest is also a main investigation target in this study.  Harvest method is a major human-related factor when deploying co-culture system in real life.

Dozens of \phy-\bac\ related mechanistic models are published in the past few decades.  Yet these models are based on site-specific nutrient (nitrogen and/or phosphorus) cycles (Table \ref{modComp}).  Examples of site-specific components are zooplankton, functional responses on grazing pressure and biomass batch removal by seasonal thermocline depth shifts.  Nutrient cycles in these models define multiple chemical resource pools.  This study focus on a single resource (carbon dioxide) pool in a co-culture engineering context, hence these models are unsuitable.  A new mechanistic model is hence developed, which simulates how carbon flows in a \pbs\ with one \phy, one \bac\ and artificial organic carbon harvest.

Using the new model, I address \Rn{1}) feasibility of \pbs s, \Rn{2}) harvest preferences on different \phy\ and \phy-\bac\ \pbs s and \Rn{3}) importance of life history traits of \phy\ and \bac\ on the organic carbon yield.

\textbf{How common is a \pbs\ being feasible under different harvest modes?}  Feasible is interpreted as finite non-negative carbon densities in each system component in a scenario.  Harvest mode can be interpreted as either ``destructive” or ``continuous”.  Destructive harvest means we extract all carbon from a system after letting it run for some time.  Continuous harvest means we keep extracting carbon at a precise rate when the system is running.  Knowing the feasibility limits can expand the knowledge on explaining the reason(s) for the fragmented understanding \autocite{fuentes2016impact} on \pbs\ and why this field is not currently addressed as a carbon capture and storage method.

\textbf{In what conditions does a \pbs\ prefer destructive or continuous harvest?}  Destructive harvest is a common practice in farms and is extended to aquacultures of macroalgae \autocite{duarte2017can} and \phy\  \autocite{evanson_2019}.  Continuous harvest is a growing concept for yield enhancement in various microbial cultures \autocite{aytekin2016statistical,fuentes2016impact}.  At present, there is insufficient information to conclude which harvesting method is superior than the other under what conditions.  Evaluating both harvest methods quantitatively can give direction on designing protocols to construct, deploy and maintain \pbs\ as a carbon harvest engine.

\textbf{To what extent do life history traits influence organic carbon yield from \phy\ cultures and \pbs?}  At present, experimental searching effort on possible co-cultures in the past decade has resulted a handful of examples \autocite{fuentes2016impact,santos2014microalgal}.  Because \pbs\ are rare, published methodologies in these \pbs\ investigations are case-specific.  Hence future searching directions on \pbs\ are unclear.  Co-culture carbon yields are typically optimised by growth media optimisation \autocite{aytekin2016statistical,fuentes2016impact} and interspecies signalling \autocite{fuentes2016impact}.  The optimisations can provide indirect inference on how high do the carbon yield can be but do not have sufficient mechanistic information showing how do these high yields are achieved.  Directly addressing the lack of mechanistic information is necessary to show the key indicators contributing to the success in carbon harvest using \phy-\bac\ \pbs s.

The new model resembles an open system (material exchange across system boundary is allowed) of carbon capture engineered with one \phy\ and one \bac.  Prediction on organic carbon yield at different times and stable positions of the systems are possible.  The predictions provide quantitative guides on which life history trait values of \phy\ and \bac\ should be considered and what type of harvesting method is the optimal for the selected \phy-\bac\ couple.

\end{document}