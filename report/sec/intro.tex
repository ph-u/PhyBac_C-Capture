% Author: PokMan Ho
% Script: intro.tex
% Desc: MRes thesis introduction section
% Input: none
% Output: none
% Arguments: 0
% Date: Jul 2020

\documentclass[../thesis.tex]{subfiles} %% use packages & commands as this main file

\begin{document}
\section{Introduction}

Climate destabilisation is happening.\autocite{notz2016observed,schuur2015climate}  Anthropogenic emission is affecting the environment\autocite{notz2016observed} but the society still depends on fossil fuels to power the economy.\autocite{lotfalipour2010economic}  Since economy has to be developed by using electricity,\autocite{ferguson2000electricity} how can we minimise our impacts to the globe?

In the past decades, scientists and engineers have proposed numerous methods and measures to capture and sequester carbon.\autocite{farrelly2013carbon,yang2008progress}  There are three main approaches: abiotic (chemical or mechanical) capture and isolation, artificial fertilization and plantation.  From the aspect of life-cycle assessment (LCA), the only way to achieve a net negative carbon footprint is to minimize anthropogenic intervention.  Most of the methods reviewed cannot logically achieve expected effect because they fail to address carbon emitted from associated materials they use.  For example, chemicals, adsorbents and fertilizers used for carbon harvest require electricity to produce.  The use of this material can only be justified being carbon negative if electricity is produced in a (at most) carbon-neutral way under LCA.  Hence it is nearly unfeasible to use artificial non-waste materials as ingredients for carbon harvest.  One exceptional case is the cultivation of fast-growing photosynthetic organisms,\autocite{farrelly2013carbon} if they can survive without human cultivation effort.  This shines light on ecology, which building a sustainable artificial microbial ecosystem is logically feasible.

Aquatic photosynthetic communities are guesstimated as powerful as terrestrial ones in carbon capture ability.\autocite{SCHLESINGER2013341}  Since aquatic phytoplanktons are small sized but fast-growing organisms, their communities can be bottled into different forms, such as panels and bulbs\autocite{evanson_2019}.  If this advantage can be better utilised, carbon footprints from cities and industries can potentially be reduced.  Given the fact that bacterial decomposers are everywhere, it is illogical to neglect their existence (and potential impacts) in an open system (i.e. materials can be exchanged between the set-up and the outside world) in open space.  With the above idea in mind, empirical approaches become too complicated because 1. many phytoplanktons are unknown, hence optimal candidates are unlikely to be identified and listed; 2. lineages of bacterial decomposers exist in a system are random and their performances are under-investigated; and 3. too many known and unknown biochemical interactions are happening simultaneously between individuals, populations and lineages in real life.  Hence a mathematical model is useful to generalise the reality and contain the system to a few defined interactions for detailed investigations.

Many interactive models are built for different purposes, mainly grouped by their growth media (aquatic or terrestrial) and number of players (multi-lineages or multi-organs within an individual).  These models are specific to their respective environmental settings and nutrient limitations.  By comparing model features (Table \ref{modComp}), a new general model should be constructed.  Apart from the above phenotypical models, there are two models published, a community model based on whole genomes\autocite{harcombe2014metabolic} and a specific model for biofuel\autocite{kirthiga2014mathematical}.  These two models are too specific because they require prior knowledge and data regarding to the microbes of interest.  Hence this new ordinary differential equation (ODE) model is aimed at visualising a minimal system as a whole, which accommodates known and unknown microbes as long as they fit into either the phytoplankton (P) or bacterial decomposer (B) categories.

Features of this model/system are fitted between the previous two known clusters of models (i.e. aquatic nutrient and terrestrial carbon-based).  It is a three-part system modeling carbon density from start to equilibrium.  The open system is modeled under the research idea ``Do bacteria benefit carbon harvest in a phytoplankton culture?"  Under this motivation, three exploratory questions are proposed:

%% new
Current phytoplankton and bacteria applications are distinct.  Phytoplankton is mainly used in pollution control\autocite{evanson_2019,mcginn2011integration,rawat2011dual} and biofuel for combustion engines\autocite{mcginn2011integration,rawat2011dual}; some also speculate their potential of being solar panels.\autocite{sawa2017electricity,joshi2018bacterial}  On the other hand, bacteria is commonly used in many fields and industries.  For example bioremediation, artificial endocrine production and 

\begin{itemize}
    \item \textbf{
    Do P+B systems have higher sustainable yield than P-only ones?
    }
\end{itemize}

Sustainable yield is the continuous harvest of organic carbon without damaging the function of the system.  
A P+B system would have more development potential than a P-only system (e.g. microbial carbon capture cells).  Bacterial invasion into artificial phytoplankton cultures is also inevitable.  Hence if bacteria do not hinder the carbon harvest of phytoplankton systems, a P+B model should be considered.  As bacteria utilize organic carbon as resource, a content-restoring tendency in the organic carbon pool might be induced.  This tendency can ultimately be resolved by absorbing more atmospheric carbon into the system.

\begin{itemize}
    \item \textbf{
    Do continuous yield higher than destructive ones?
    } And
\end{itemize}

It is based on the idea that carbon harvest ability of grasslands (i.e. a frequently-disturbed ecosystem by wild fire) may be more reliable than forests (i.e. a stable ecosystem).\autocite{dass2018grasslands}  A possible reason is because short-living individuals maximize growth (i.e. carbon accumulation for biomass) to win over competitors.  Under this idea, the carbon cycle in the ecosystem can potentially be sped up by taking carbon out from the system and conserve this momentum of growth.

\begin{itemize}
    \item \textbf{
    Do continuous yields depend on phytoplankton/bacteria biological features?
    }
\end{itemize}

Since the carbon harvest is depending on the interactions between bio-components in the system (Eq.\ref{eq:ode}), it is logical to speculate whether the choice of organism put into the system is crucial to the success of carbon yield.  The model in this study focus on a general reality, which changes in any parameter value within the parameter hyperspace can potentially be interpreted as a change of the respective organism (either P or B) put into the system.  Whether there are existing strains of microbes with such suit of features, however, require ground-truthing.

\end{document}
